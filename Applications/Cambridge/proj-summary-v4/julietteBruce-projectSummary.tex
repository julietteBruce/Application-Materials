\documentclass[11pt,reqno]{amsart}
\usepackage{amsfonts,amsmath,amssymb,amsbsy,amstext,amsthm,mathtools}
\usepackage{accents,color,enumerate,enumitem,float,fullpage,verbatim}

\usepackage{url}
\usepackage[colorlinks=true,hyperindex, linkcolor=magenta, pagebackref=false, citecolor=cyan]{hyperref}
\usepackage[alphabetic,lite]{amsrefs} 

%\usepackage{eucal,bm,kpfonts,mathbbol}
\usepackage[margin=1.02in,includeheadfoot]{geometry}

\usepackage{tikz,tikz-cd}	
\usetikzlibrary{positioning, matrix, shapes}         								    				
\usetikzlibrary{arrows,calc,matrix}

\usepackage{lscape}

\usepackage{microtype}


\usepackage{titlesec}		
\setcounter{secnumdepth}{4}						     					% Allows one to use nice section titles
\titleformat{\section}[block]{\scshape\bfseries\filcenter}{\thesection.}{1em}{}		% Creates section titles
\titleformat{\subsection}[runin]{\scshape\bfseries}{\thesubsection}{1em}{}			% Creates subsection titles
\titleformat{\subsubsection}[runin]{\scshape\bfseries}{\thesubsubsection}{1em}{}			% Creates subsection titles

\usepackage[titles]{tocloft}								     					% Creates table of fancy contents
\setcounter{tocdepth}{4}
\renewcommand{\contentsname}{}	     					% Renames and centers title of ToC

\usepackage{multirow}
\usepackage{array}
\usepackage{booktabs}
\newcolumntype{M}[1]{>{\centering\arraybackslash}m{#1}}
\newcolumntype{N}{@{}m{0pt}@{}}
\usepackage{diagbox}
\usepackage{cancel}

\newtheorem{lemma}{Lemma}[section]
\newtheorem{theorem}[lemma]{Theorem}
\newtheorem{goalTheorem}[lemma]{Goal Theorem}
\newtheorem{prop}[lemma]{Proposition}
\newtheorem{cor}[lemma]{Corollary}
\newtheorem{conj}[lemma]{Conjecture}
\newtheorem{claim}[lemma]{Claim}
\newtheorem{defn}[lemma]{Definition} 
\newtheorem{notation}[lemma]{Notation} 
\newtheorem{exercise}[lemma]{Exercise}
\newtheorem{question}[lemma]{Question}
\newtheorem*{assumption}{Assumption}
\newtheorem{principle}[lemma]{Principle}
\newtheorem{heuristic}[lemma]{Heuristic}

\newtheorem{theoremalpha}{Theorem}
\newtheorem{corollaryalpha}[theoremalpha]{Corollary}
\renewcommand{\thetheoremalpha}{\Alph{theoremalpha}}

\theoremstyle{remark}
\newtheorem{remark}[lemma]{Remark}
\newtheorem{example}[lemma]{Example}
\newtheorem{cexample}[lemma]{Counterexample}

% Commands
\newcommand{\initial}{\operatorname{in}}
\newcommand{\NF}{\operatorname{NF}}
\newcommand{\HF}{\operatorname{HF}}
\newcommand{\Hilb}{\operatorname{Hilb}}
\newcommand{\depth}{\operatorname{depth}}
\newcommand{\reg}{\operatorname{reg}}
\newcommand{\Span}{\operatorname{span}}
\newcommand{\img}{\operatorname{img}}
\newcommand{\inn}{\operatorname{in}}

\newcommand{\length}{\operatorname{length}}
\newcommand{\coker}{\operatorname{coker}}
\newcommand{\adeg}{\operatorname{adeg}}
\newcommand{\pdim}{\operatorname{pdim}}
\newcommand{\Spec}{\operatorname{Spec}}
\newcommand{\Ext}{\operatorname{Ext}}
\newcommand{\Tor}{\operatorname{Tor}}
\newcommand{\LT}{\operatorname{LT}}
\newcommand{\im}{\operatorname{im}}
\newcommand{\NS}{\operatorname{NS}}
\newcommand{\Frac}{\operatorname{Frac}}
\newcommand{\Khar}{\operatorname{char}}
\newcommand{\Proj}{\operatorname{Proj}}
\newcommand{\id}{\operatorname{id}}
\newcommand{\Div}{\operatorname{Div}}
\newcommand{\Kl}{\operatorname{Cl}}
\newcommand{\tr}{\operatorname{tr}}
\newcommand{\Tr}{\operatorname{Tr}}
\newcommand{\Supp}{\operatorname{Supp}}
\newcommand{\ann}{\operatorname{ann}}
\newcommand{\Gal}{\operatorname{Gal}}
\newcommand{\Pic}{\operatorname{Pic}}
\newcommand{\QQbar}{{\overline{\mathbb Q}}}
\newcommand{\Br}{\operatorname{Br}}
\newcommand{\Bl}{\operatorname{Bl}}
\newcommand{\Kox}{\operatorname{Cox}}
\newcommand{\conv}{\operatorname{conv}}
\newcommand{\getsr}{\operatorname{Tor}}
\newcommand{\diam}{\operatorname{diam}}
\newcommand{\Hom}{\operatorname{Hom}} %done
\newcommand{\sheafHom}{\mathcal{H}om}
\newcommand{\Gr}{\operatorname{Gr}}
\newcommand{\rank}{\operatorname{rank}} 
\newcommand{\codim}{\operatorname{codim}}
\newcommand{\Sym}{\operatorname{Sym}} %done
\newcommand{\GL}{{GL}}
\newcommand{\Prob}{\operatorname{Prob}}
\newcommand{\Density}{\operatorname{Density}}
\newcommand{\Syz}{\operatorname{Syz}}
\newcommand{\pd}{\operatorname{pd}}
\newcommand{\supp}{\operatorname{supp}}
\newcommand{\cone}{\operatorname{\textbf{cone}}}
\newcommand{\Res}{\operatorname{Res}}
\newcommand{\HS}{\operatorname{HS}}
\newcommand{\Cl}{\operatorname{Cl}}
\newcommand{\oO}{\operatorname{O}}

\newcommand{\defi}[1]{\textsf{#1}} % for defined terms

\newcommand{\remd}{\operatorname{remd}}
\newcommand{\colim}{\operatorname{colim}}
\newcommand{\trideg}{\operatorname{tri.deg}}
\newcommand{\indeg}{\operatorname{index.deg}}
\newcommand{\moddeg}{\operatorname{mod.deg}}
\newcommand{\Desc}{\operatorname{Desc}}
\newcommand{\inter}{\operatorname{int}}
\newcommand{\Nef}{\operatorname{Nef}}
\newcommand{\Jac}{\operatorname{Jac}}
\newcommand{\Cox}{\operatorname{Cox}}

\newcommand{\doot}{\bullet}

\newcommand{\Alt}{\bigwedge\nolimits}
\newcommand{\Set}{\text{\bf Set}}										% Category of Sets
\newcommand{\Sch}{\text{\bf Sch}}										% Category of Abelian Groups
\newcommand{\Mod}[1]{\ (\mathrm{mod}\ #1)}




%%%%%%%%%%%%%%%%%%%%%%%%%%%%%% Letters  %%%%%%%%%%%%%%%%%%%%%%%%%%%%%%%%%%%%%%%%%%%%
%%%%%%%%%%%%%%%%%%%%%%%%%%%%%%%%%%%%%%%%%%%%%%%%%%%%%%%%%%%%%%%%%%%%%%%%%%%%%%
\newcommand{\ff}{\mathbf f}
\newcommand{\kk}{\mathbf k}
\renewcommand{\aa}{\mathbf a}
\newcommand{\bb}{\mathbf b}
\newcommand{\cc}{\mathbf c}
\newcommand{\dd}{\mathbf d}
\newcommand{\ee}{\mathbf e}
\newcommand{\vv}{\mathbf v}
\newcommand{\ww}{\mathbf w}
\newcommand{\xx}{\mathbf x}
\newcommand{\yy}{\mathbf y}
\newcommand{\rr}{\mathbf r}
\newcommand{\ii}{\mathbf i}
\newcommand{\nn}{\mathbf n}
\newcommand{\pp}{\mathbf p}
\newcommand{\mm}{\mathbf m}
\newcommand{\fF}{\mathbf F}
\newcommand{\gG}{\mathbf G}
\newcommand{\eE}{\mathbf E}
\newcommand{\qQ}{\mathbf Q}
\newcommand{\tT}{\mathbf T}
\renewcommand{\tt}{\mathbf t}
\newcommand{\one}{\mathbf 1}
\newcommand{\zero}{\mathbf 0}

\renewcommand{\H}{\operatorname{H}}
\newcommand{\OO}{\operatorname{O}}
\newcommand{\oo}{\operatorname{o}}


%%%% Caligraphic Fonts - i.e. ????. %%%%%
\newcommand{\cA}{\mathcal{A}}
\newcommand{\cB}{\mathcal{B}}
\newcommand{\cC}{\mathcal{C}}
\newcommand{\cD}{\mathcal{D}}
\newcommand{\cE}{\mathcal{E}}
\newcommand{\cF}{\mathcal{F}}
\newcommand{\cG}{\mathcal{G}}
\newcommand{\cH}{\mathcal{H}} 
\newcommand{\cI}{\mathcal{I}}
\newcommand{\cJ}{\mathcal{J}}
\newcommand{\cK}{\mathcal{K}}
\newcommand{\cL}{\mathcal{L}}
\newcommand{\cM}{\mathcal{M}}
\newcommand{\cN}{\mathcal{N}}
\renewcommand{\O}{\mathcal{O}}
\newcommand{\cP}{\mathcal{P}}
\newcommand{\cQ}{\mathcal{Q}}
\newcommand{\cR}{\mathcal{R}}
\newcommand{\cS}{\mathcal{S}}
\newcommand{\cT}{\mathcal{T}}
\newcommand{\U}{\mathcal{U}} 		% Notice this is different
\newcommand{\cV}{\mathcal{V}}
\newcommand{\cW}{\mathcal{W}}
\newcommand{\cX}{\mathcal{X}}
\newcommand{\cY}{\mathcal{Y}}
\newcommand{\cZ}{\mathcal{Z}}

%%%% Blackboard Fonts - i.e. Real Numbers, Integers, etc. %%%%%
\newcommand{\A}{\mathbb{A}}
\newcommand{\B}{\mathbb{B}}
\newcommand{\C}{\mathbb{C}}
\newcommand{\D}{\mathbb{D}}
\newcommand{\E}{\mathbb{E}}
\newcommand{\F}{\mathbb{F}}
\newcommand{\G}{\mathbb{G}}
\newcommand{\I}{\mathbb{I}}
\newcommand{\J}{\mathbb{J}}
\newcommand{\K}{\mathbb{K}}
\renewcommand{\L}{\mathbb{L}}
\newcommand{\M}{\mathbb{M}}
\newcommand{\N}{\mathbb{N}}
\newcommand{\bO}{\mathbb{O}}		% Notice this is \bO
\renewcommand{\P}{\mathbb{P}}
\newcommand{\Q}{\mathbb{Q}}
\newcommand{\R}{\mathbb{R}}
\newcommand{\T}{\mathbb{T}}
\newcommand{\bU}{\mathbb{U}}		% Notice this is \bU
\newcommand{\V}{\mathbb{V}}
\newcommand{\W}{\mathbb{W}}
\newcommand{\X}{\mathbb{X}}
\newcommand{\Y}{\mathbb{Y}}
\newcommand{\Z}{\mathbb{Z}}

 %%%% Sarif Fonts - i.e. ???? %%%%%
\newcommand{\sA}{\mathsf{A}}
\newcommand{\sB}{\mathsf{B}}
\newcommand{\sC}{\mathsf{C}}
\newcommand{\sD}{\mathsf{D}}
\newcommand{\sE}{\mathsf{E}}
\newcommand{\sF}{\mathsf{F}}
\newcommand{\sG}{\mathsf{G}}
\newcommand{\sH}{\mathsf{H}} 
\newcommand{\sI}{\mathsf{I}}
\newcommand{\sJ}{\mathsf{J}}
\newcommand{\sK}{\mathsf{K}}
\newcommand{\sL}{\mathsf{L}}
\newcommand{\sM}{\mathsf{M}}
\newcommand{\sN}{\mathsf{N}}
\newcommand{\sO}{\mathsf{O}}
\newcommand{\sP}{\mathsf{P}}
\newcommand{\sQ}{\mathsf{Q}}
\newcommand{\sR}{\mathsf{R}}
\newcommand{\sS}{\mathsf{S}}
\newcommand{\sT}{\mathsf{T}}
\newcommand{\sU}{\mathsf{U}} 
\newcommand{\sV}{\mathsf{V}}
\newcommand{\sW}{\mathsf{W}}
\newcommand{\sX}{\mathsf{X}}
\newcommand{\sY}{\mathsf{Y}}
\newcommand{\sZ}{\mathsf{Z}}
 
 %%%% Fraktur Fonts - i.e. maximal ideals, prime ideals, etc. %%%%%
\newcommand{\cl}{\mathfrak{cl}}
\newcommand{\g}{\mathfrak{g}}
\newcommand{\h}{\mathfrak{h}}
\newcommand{\m}{\mathfrak{m}}
\newcommand{\n}{\mathfrak{n}}
\newcommand{\p}{\mathfrak{p}}
\newcommand{\q}{\mathfrak{q}}
\renewcommand{\r}{\mathfrak{r}}



\newcommand{\juliette}[1]{{\color{red} \sf $\spadesuit\spadesuit\spadesuit$ Juliette: [#1]}}


\title{Juliette Bruce's Research Statement}

%\author{Juliette Bruce}
%\address{Department of Mathematics, University of Wisconsin, Madison, WI}
%\email{\href{mailto:juliette.bruce@math.wisc.edu}{juliette.bruce@math.wisc.edu}}
%\urladdr{\url{http://math.wisc.edu/~juliettebruce/}}

%\thanks{The author was partially supported by the NSF GRFP under Grant No. DGE-1256259 and NSF grant DMS-1502553.}

%\subjclass[2010]{13D02, 14M25}

\begin{document} 

%\maketitle
\begingroup  
  \centering
  \large\scshape\bfseries Juliette Bruce's Project Summary\\[1em]
\endgroup

%\tableofcontents

\setcounter{section}{0}

\noindent \textbf{Overview.} I propose to study questions in commutative algebra, algebraic geometry, and arithmetic geometry. Two projects involve studying the geometry of algebraic varieties via homological commutative algebra. A third project lies in the intersection of algebraic geometry and arithmetic geometry and seeks to extend classical results in algebraic geometry to finite fields. 
\\

\noindent \textbf{Intellectual Merit.} The first project seeks to expand our understanding of syzygies of algebraic varieties from both the asymptotic and computational perspectives. In recent years substantial work has been done studying the syzygies of algebraic varieties as the positivity of the embedding grows. This project builds upon my thesis and seeks to extend these results by weakening the positivity conditions previously considered. In particular, the PI will explore the asymptotic syzygies of varieties in the setting of semi-ample growth. Further, by utilizing new advances in high-performance computing and numerical linear algebra this project will generate new data regarding the syzygies of Hirzebruch surfaces. This data will be publicly disseminated via the online syzygy database \textit{syzygydata.com}, which I maintain. 

A second project proposed will deepen our understanding of curves in toric 3-folds, for example, curves in $\P^1\times\P^2$, by generalizing existing results about the liaison theory of curves in $\P^3$. This project will make use of recent developments in the homological properties of varieties embedded in spaces other than $\P^n$. This project has the potential to yield applications to the study of toric varieties and toric vector bundles, as well as potential applications to unirationality of $\mathcal{M}_g$ or other moduli spaces (similar to~[BS15, Theorem 4.5]).

The third project I am proposing, builds upon recent work generalizing classical Bertini Theorems in algebraic geometry to the setting of finite fields. More specifically, this project hopes to prove more general and uniform Bertini Theorems over finite fields. Such results will shed light on both the geometry and arithmetic of varieties of finite fields, and potentially prove useful in studying things like rational points on varieties. 
\\

\noindent \textbf{Broader Impacts.} As an LGBTQ woman, I have worked hard to promote diversity, inclusivity, and justice in the mathematical community. This proposal will further my work in this direction by my continued involvement  as a mentor to multiple undergraduate women via the Association for Women in Mathematics's mentor network. 

At the 2018 Joint Mathematica Meeting, the I served on a panel organized by \textit{Spectra}, the association of LGBTQ+ mathematicians, titled \textit{Professional Issues Facing LGBTQ Mathematicians}. At the 2020 Joint Meetings, the PI is organizing a \textit{Spectra} panel on transgender inclusion in mathematics. Going forward the I have plans to continue being involved in \textit{Spectra}, potentially stepping into more leadership and organizing roles. I also founded and organized numerous groups for LGBTQ+ students at the University of Wisconsin. Since 2017 I have organized the campus social group for LGBTQ+ graduate students, which has over 350 members. 

I have organized a number of conferences including the \textit{Graduate Workshop in Commutative Algebra for Women and Mathematicians of Minority Genders} (2019), a workshop bringing together algebraic geometers and number theorists \textit{Geometry \& Arithmetic of Surfaces} (2019), and a five day conference dedicated to developing open-source computer software for algebraic geometry and commutative algebra \textit{M2@UW} (2018). I have plans to organize a follow-up to \textit{Graduate Workshop in Commutative Algebra for Women and Mathematicians of Minority Genders} tentatively planned for Spring 2021, as well as a conference for LGBTQ+ mathematicians in algebraic geometry and commutative algebra. 



	

\end{document}