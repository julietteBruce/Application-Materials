\documentclass[11pt]{article}

\usepackage{graphicx}
\usepackage{color}
\usepackage[hidelinks]{hyperref}
\usepackage{fullpage}
\setlength{\topmargin}{-1.0in}
\setlength{\footskip}{0.5in}
\setlength{\textwidth}{6.5in}
\setlength{\textheight}{10.0in}
\setlength{\oddsidemargin}{-0.0in}
\setlength{\evensidemargin}{-0.0in}
\usepackage{parskip}
\usepackage{fancyhdr}
\pagestyle{fancy}
\addtolength{\headheight}{\baselineskip}
\addtolength{\headheight}{1in}
\addtolength{\textheight}{-\baselineskip}
\addtolength{\textheight}{-1in}
\lhead{}
\chead{\includegraphics[height=1in]{color-center-UWlogo-print}}
\rhead{}
\lfoot{}
\cfoot{{\footnotesize
       Department of Mathematics \\
       University of Wisconsin--Madison \hspace{0.1in} 480 Lincoln Dr \hspace{0.1in} Madison, Wisconsin 53706 \\
       Phone: (608) 263--3054 \hspace{0.25in}
       Fax: (608) 263--8891 \hspace{0.25in}
       Web: http://www.math.wisc.edu
       }}
\rfoot{}
\renewcommand{\headrulewidth}{0pt}

\begin{document}

\section*{}

\noindent
\begin{minipage}{0.99\textwidth}
\begin{minipage}{0.69\textwidth}
\textcolor{white}{.}
\end{minipage}
\begin{minipage}{0.29\textwidth}
{
Juliette Bruce \\
Graduate Student \\
Department of Mathematics \\
\href{mailto:juliette.bruce@math.wisc.edu}{juliette.bruce@math.wisc.edu}
%\url{juliettebruce.github.io}
%Phone: (810)--623--7610 
}

\vspace{12pt}
\today
\end{minipage}
\end{minipage}

%\vspace{12pt}
%\noindent
%John Doe \\
%Business, Inc. \\
%1234 Address Avenue \\
%City, State ZZZIP

\vspace{12pt}
\noindent
To the Hiring Committee,

My name is Juliette Bruce, and I am a graduate student at the University of Wisconsin - Madison, working in pure mathematics, namely, algebraic geometry, commutative algebra, and arithmetic geometry under the guidance of my advisor Professor Daniel Erman. I expect to receive my Ph.D. in Mathematics from the University of Wisconsin - Madison in the Spring of 2020. I am writing to apply for the University Research Postdoctoral Fellowship at the University of Kentucky as a member of the Department of Mathematics. Professor Dave Jensen has agreed to be my mentor. 

My application includes:  a curriculum vitae, a description of my career goals, and a research statement. I will have five letters accompanying my application. Three letters of recommendation: Christine Berkesch (\href{mailto:cberkesc@umn.edu}{cberkesc@umn.edu}), David Eisenbud (\href{mailto:de@msri.org}{de@msri.org}), and Daniel Erman (\href{mailto:derman@math.wisc.edu}{derman@math.wisc.edu}), and two letters of support: my proposed mentor Dave Jensen (\href{dave.jensen@uky.edu}{dave.jensen@uky.edu}) and the chair of the Department of Mathematics Uwe Nagel (\href{uwe.nagel@uky.edu}{uwe.nagel@uky.edu}).

I believe strongly in the importance of inclusivity, diversity, and justice, and I am passionate about promoting these values within the mathematical community. As a graduate student at the University of Wisconsin-Madison I worked hard to create a learning community that was as open and inclusive to as many people as possible. By working with outreach programs like the Madison Math Circle I expanded the reach of the university outside the bounds of campus. While on campus I have made our learning community more inclusive and welcoming of people from underrepresented groups; especially LGBTQ+ individuals, through work on the Mathematics Department's Committee on Inclusion and Diversity and by founding oSTEM@UW. Further, to promote the success of mathematicians from minority genders I organized a number of workshops and conferences. Going forward, I am excited to continue working hard to promote these values at the University of Kentucky through my research, teaching, and service. 
\\
\\
\noindent \textbf{Expanding the Learning Community.} The Madison Math Circle (MMC) is an outreach program sponsored by the UW - Madison Math Department. Its goal is to kindle excitement and appreciation of math in middle and high school students. Towards the end of my first semester in graduate school, Fall 2014, I began volunteering with the MMC. At the time, the circle's main programming was a weekly on-campus lecture given by a member of the math department. After volunteering with the MMC for roughly a year, I stepped into the role of student organizer/coordinator. 

During my roughly three years as organizer, I worked to build stronger connections between the Madison Math Circle, local schools and teachers, and other outreach organizations focused on underrepresented groups. These ties helped the weekly attendance of the circle to more than double, and grow substantially more diverse. Additionally, during my time the number of women and undergraduate speakers increased. I also led the creation of a new outreach arm of the MMC, which visits high schools around the state of Wisconsin to better serve students from underrepresented groups. This program has dramatically expanded the reach of the circle, and during my final year as an organizer the circle reached over 300 students.
\\
\\
\noindent \textbf{A More Inclusive Learning Community.} During the Fall of 2016, in response to a growing climate of hate, bias, and discrimination on campus, I led the creation of the Mathematics Department's \textit{Committee on Inclusivity and Diversity}. As a member of this committee I drafted a statement on the department's commitment to inclusivity and non-discrimination that was accepted by the faculty at a department meeting. I also worked to create syllabi statements that let students know about these department polices, and that inform them of other campus resources that may be helpful. Everyone within the department is now encouraged to use these statements. 

More recently, my passion for creating a more inclusive campus has expanded outside of the math department to try and help address inequalities in STEM fields, more generally by founding oSTEM@UW and organizing qGrads. While a large proportion of students at UW - Madison pursue degrees in STEM adjacent fields there are few -- if any -- resources on campus that directly support LGBTQ+ students in STEM. This is despite the fact that many LGBTQ+ students in these fields often feel isolated, feel the need to hide their identity, or even to leave STEM altogether.

In light of this, and my own experiences as an LGBTQ+ person in STEM, over the summer of 2017 I co-founded Out in Science, Technology, Engineering, and Mathematics at UW (oSTEM@UW) as a resource for these students. During my time leading oSTEM@UW, the group grew to over fifty active members. The importance of such a group was made clear by the numerous student comments indicating how helpful and encouraging oSTEM@UW is to them. For example, after a meeting, a student emailed me to say, ``It made me very happy to see other friendly LGBTQ+ faces around ...Thanks so much for organizing this stuff -- it's really helpful for me personally, and I believe it was encouraging for the others attending as well.'' Additionally, I organized and obtained a travel grant for 11 members, including multiple undergrads, to attend the national oSTEM conference. 

%As the vice president of oSTEM@UW I organized for eleven members -- including multiple undergraduates -- to attend the annual national oSTEM Inc. conference. This four day conference with participants from around the world is intended to help individuals learn to build community and unity within the diverse LGBTQ+ family. It also has opportunities for participants to present their research, which a few of our members will be doing. For a couple of those UW -Madison students going, this is their first opportunity to talk about their research. I secured grants from on-campus and off-campus sources to defer the cost of attendance, and give these eleven students this amazing educational and social experience.

Since 2017 I have been the organizer of the campus social organization for LGBTQ+ graduate and post-graduate students, which currently has over 350 members. In this role, I have co-organized a weekly coffee social hour intended to give LGBTQ+ graduate and post-graduate students a place to relax, make friends, and discuss the challenges of being LGBTQ+ at UW - Madison.
\\
\\
\noindent \textbf{Mentoring.} Inspired by the mentoring that helped me navigate the challenges of being a women in mathematics, I have worked hard to mentor people from underrepresented groups. Since the Winter of 2018 I have led reading courses with three undergraduates through the \textit{Wisconsin Directed Reading Program}. One of these students, an undergraduate woman, worked with me for over a year. During this time I helped her through the process of applying for summer research projects. This student is now applying to graduate school to pursue a Ph.D. in math. Working with \textit{Girls' Math Night Out} I lead two girls in high school through a semester long project exploring RSA cryptography. During 2018-2019, I mentored 6 first-year graduate students (all women or non-binary students), advicing them oh how to navigte the program requirements, helping them find advisors, and organizing monthly social dinners. Since 2016 I have volunteered with the AWM's Mentoring Network, and currently I am mentoring two undergraduate women.
\\
\\
\noindent \textbf{Organizational Service.}  In the Spring of 2017 I organized \textit{Math Careers Beyond Academia } (50 participants), a one-day professional development conference on STEM careers outside of academia. In April 2018 I organized \textit{M2@UW} (45 participants), a four-day workshop focused on creating new packages for Macaulay2. In February 2019 I organized \textit{Geometry and Arithmetic of Surfaces} (40 participants), a workshop providing a diverse group of early-career researchers the opportunity to learn about interesting topics in the arithmetic and algebraic geometry. In April 2019 I organized the \textit{Graduate Workshop in Commutative Algebra for Women \& Mathematicians of Other Minority Genders} (35 participants)  focused on forming a community of women and non-binary researchers in commutative algebra, and give young graduate students from minority genders, role models for the next stage in their careers. I organized a \textit{Special Session on Combinatorial Algebraic Geometry} at the AMS Fall 2019 Central Sectional. At the 2020 Joint Mathematics Meetings, I am organizing a panel titled \textit{Supporting Transgender and Non-binary Students}. 

When organizing these conferences I paid particular attention to making them as inclusive of women and non-binary researchers as possible. In particular, I worked hard to make sure there was gender parity among the speakers and participants.  For example, of the five speakers at G\&AoS four were from generally underrepresented groups with three women and one person of color speaking. Additionally, over 30 identify as either female or non-binary researchers. 

I have given substantial thought to how to make many of the smaller aspects of conferences more inclusive. For example, I designed the registration form to be thoughtful of the concerns of transgender researchers, implemented the process of putting pronouns on name tags, highlighted the locations of single occupancy and ADA compliant restrooms. The importance of such efforts was highlighted by the following comment I received from a participant, ``I just wanted to thank you for making this workshop inclusive for people with all gender identifications. As a non-binary biologically female person I have always felt out of place when I participated in conferences/workshops for women when they do not specify that non-binary people are welcome or just assume I am female. I really appreciate those questions you put in the registration form. It means a lot to me.''
\\
\\
As a graduate student I worked hard to develop programs, policies, and practices that promoted diversity, inclusion, and justice within the academic community. As I move forward in my career I hope to continue, and expand upon, this work. I believe that being a University Research Postdoctoral Fellow would give me this opportunity.  In particular, if selected as a fellow, I will work hard to continue promoting these values through my research, teaching, and service. 



%A few highlights in my file are:
%\begin{itemize}
%\item Invited to speak at \textit{CA+} (April 2020), \textit{Foundations of Computing in Mathematics} (June 2020), and \textit{LGBTQ+Math} (July 2020).
%
%\item Elizabeth Hirschfelder Prize (2019) - Awarded by the Department of Mathematics at the University of Wisconsin to an outstanding female student who has demonstrated promise.
%
%\item Co-organizer for three conferences (\textit{M2@UW}, \textit{Geometry \& Arithmetic of Surfaces}, \textit{Graduate Workshop in Commutative Algebra for Women and Mathematicians of Minority Genders}), a AMS special session, and a Spectra panel. 
%
%\item Teaching Assistant Award for Exceptional Service  (2018) - Awarded by the University of Wisconsin to up to 3 teaching assistants campus-wide recognizing their exceptional service.
%\end{itemize}

Please do not hesitate to contact me with any questions, or if there is anything else I can provide, and thank you in advance for your consideration. 

\vspace{24pt}
\noindent
\begin{minipage}{0.99\textwidth}
\begin{minipage}{0.69\textwidth}
\textcolor{white}{.}
\end{minipage}
\begin{minipage}{0.29\textwidth}
Sincerely, 

\vspace{36pt}
Juliette Bruce\\
Graduate Student
\end{minipage}
\end{minipage}

% This command changes the page style to plain from this page onward.
% If your letter is 1 page long, then comment this out.
% If your letter is 2 pages long, then include this command.
% If your letter is longer than 2 pages, then you may need to place
% this command earlier in the document.
%\pagestyle{plain}

\end{document}