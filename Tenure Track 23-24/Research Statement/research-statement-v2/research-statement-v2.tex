\documentclass[11pt,reqno]{amsart}
\usepackage{amsfonts,amsmath,amssymb,amsbsy,amstext,amsthm,mathtools}
\usepackage{accents,color,enumerate,enumitem,float,fullpage,verbatim}

\usepackage{url}
\usepackage[colorlinks=true,hyperindex, linkcolor=magenta, pagebackref=false, citecolor=cyan]{hyperref}
\usepackage[alphabetic,lite]{amsrefs} 

%\usepackage{eucal,bm,kpfonts,mathbbol}

\usepackage{tikz,tikz-cd}	
\usetikzlibrary{positioning, matrix, shapes}         								    				
\usetikzlibrary{arrows,calc,matrix}

\usepackage{lscape}

\usepackage{microtype}


\usepackage{titlesec}		
\setcounter{secnumdepth}{4}						     					% Allows one to use nice section titles
\titleformat{\section}[block]{\scshape\bfseries\filcenter}{\thesection.}{1em}{}		% Creates section titles
\titleformat{\subsection}[runin]{\scshape\bfseries}{\thesubsection}{1em}{}			% Creates subsection titles
\titleformat{\subsubsection}[runin]{\scshape\bfseries}{\thesubsubsection}{1em}{}			% Creates subsection titles

\usepackage[titles]{tocloft}								     					% Creates table of fancy contents
\setcounter{tocdepth}{4}
\renewcommand{\contentsname}{}	     					% Renames and centers title of ToC

\usepackage{multirow}
\usepackage{array}
\usepackage{booktabs}
\newcolumntype{M}[1]{>{\centering\arraybackslash}m{#1}}
\newcolumntype{N}{@{}m{0pt}@{}}
\usepackage{diagbox}
\usepackage{cancel}

\newtheorem{lemma}{Lemma}[section]
\newtheorem{theorem}[lemma]{Theorem}
\newtheorem{goalTheorem}[lemma]{Goal Theorem}
\newtheorem{prop}[lemma]{Proposition}
\newtheorem{cor}[lemma]{Corollary}
\newtheorem{conj}[lemma]{Conjecture}
\newtheorem{claim}[lemma]{Claim}
\newtheorem{defn}[lemma]{Definition} 
\newtheorem{notation}[lemma]{Notation} 
\newtheorem{exercise}[lemma]{Exercise}
\newtheorem{question}[lemma]{Question}
\newtheorem*{assumption}{Assumption}
\newtheorem{principle}[lemma]{Principle}
\newtheorem{heuristic}[lemma]{Heuristic}

\newtheorem{theoremalpha}{Theorem}
\newtheorem{corollaryalpha}[theoremalpha]{Corollary}
\renewcommand{\thetheoremalpha}{\Alph{theoremalpha}}

\theoremstyle{remark}
\newtheorem{remark}[lemma]{Remark}
\newtheorem{example}[lemma]{Example}
\newtheorem{cexample}[lemma]{Counterexample}

% Commands
\newcommand{\initial}{\operatorname{in}}
\newcommand{\NF}{\operatorname{NF}}
\newcommand{\HF}{\operatorname{HF}}
\newcommand{\Hilb}{\operatorname{Hilb}}
\newcommand{\depth}{\operatorname{depth}}
\newcommand{\reg}{\operatorname{reg}}
\newcommand{\Span}{\operatorname{span}}
\newcommand{\img}{\operatorname{img}}
\newcommand{\inn}{\operatorname{in}}

\newcommand{\length}{\operatorname{length}}
\newcommand{\coker}{\operatorname{coker}}
\newcommand{\adeg}{\operatorname{adeg}}
\newcommand{\pdim}{\operatorname{pdim}}
\newcommand{\Spec}{\operatorname{Spec}}
\newcommand{\Ext}{\operatorname{Ext}}
\newcommand{\Tor}{\operatorname{Tor}}
\newcommand{\LT}{\operatorname{LT}}
\newcommand{\im}{\operatorname{im}}
\newcommand{\NS}{\operatorname{NS}}
\newcommand{\Frac}{\operatorname{Frac}}
\newcommand{\Khar}{\operatorname{char}}
\newcommand{\Proj}{\operatorname{Proj}}
\newcommand{\id}{\operatorname{id}}
\newcommand{\Div}{\operatorname{Div}}
\newcommand{\Kl}{\operatorname{Cl}}
\newcommand{\tr}{\operatorname{tr}}
\newcommand{\Tr}{\operatorname{Tr}}
\newcommand{\Supp}{\operatorname{Supp}}
\newcommand{\ann}{\operatorname{ann}}
\newcommand{\Gal}{\operatorname{Gal}}
\newcommand{\Pic}{\operatorname{Pic}}
\newcommand{\QQbar}{{\overline{\mathbb Q}}}
\newcommand{\Br}{\operatorname{Br}}
\newcommand{\Bl}{\operatorname{Bl}}
\newcommand{\Kox}{\operatorname{Cox}}
\newcommand{\conv}{\operatorname{conv}}
\newcommand{\getsr}{\operatorname{Tor}}
\newcommand{\diam}{\operatorname{diam}}
\newcommand{\Hom}{\operatorname{Hom}} %done
\newcommand{\sheafHom}{\mathcal{H}om}
\newcommand{\Gr}{\operatorname{Gr}}
\newcommand{\rank}{\operatorname{rank}} 
\newcommand{\codim}{\operatorname{codim}}
\newcommand{\Sym}{\operatorname{Sym}} %done
\newcommand{\GL}{{\operatorname{GL}}}
\newcommand{\Prob}{\operatorname{Prob}}
\newcommand{\Density}{\operatorname{Density}}
\newcommand{\Syz}{\operatorname{Syz}}
\newcommand{\pd}{\operatorname{pd}}
\newcommand{\supp}{\operatorname{supp}}
\newcommand{\cone}{\operatorname{\textbf{cone}}}
\newcommand{\Res}{\operatorname{Res}}
\newcommand{\HS}{\operatorname{HS}}
\newcommand{\Cl}{\operatorname{Cl}}
\newcommand{\oO}{\operatorname{O}}
\newcommand{\Sp}{\operatorname{Sp}}
\newcommand{\trop}{\operatorname{trop}}

\newcommand{\defi}[1]{\textsf{#1}} % for defined terms

\newcommand{\remd}{\operatorname{remd}}
\newcommand{\colim}{\operatorname{colim}}
\newcommand{\trideg}{\operatorname{tri.deg}}
\newcommand{\indeg}{\operatorname{index.deg}}
\newcommand{\moddeg}{\operatorname{mod.deg}}
\newcommand{\Desc}{\operatorname{Desc}}
\newcommand{\inter}{\operatorname{int}}
\newcommand{\Nef}{\operatorname{Nef}}
\newcommand{\Jac}{\operatorname{Jac}}
\newcommand{\Cox}{\operatorname{Cox}}
\newcommand{\mat}{\operatorname{mat}}
\newcommand{\doot}{\bullet}

\newcommand{\Alt}{\bigwedge\nolimits}
\newcommand{\Set}{\text{\bf Set}}										% Category of Sets
\newcommand{\Sch}{\text{\bf Sch}}										% Category of Abelian Groups
\newcommand{\Mod}[1]{\ (\mathrm{mod}\ #1)}




%%%%%%%%%%%%%%%%%%%%%%%%%%%%%% Letters  %%%%%%%%%%%%%%%%%%%%%%%%%%%%%%%%%%%%%%%%%%%%
%%%%%%%%%%%%%%%%%%%%%%%%%%%%%%%%%%%%%%%%%%%%%%%%%%%%%%%%%%%%%%%%%%%%%%%%%%%%%%
\newcommand{\ff}{\mathbf f}
\newcommand{\kk}{\mathbf k}
\renewcommand{\aa}{\mathbf a}
\newcommand{\bb}{\mathbf b}
\newcommand{\cc}{\mathbf c}
\newcommand{\dd}{\mathbf d}
\newcommand{\ee}{\mathbf e}
\newcommand{\vv}{\mathbf v}
\newcommand{\ww}{\mathbf w}
\newcommand{\xx}{\mathbf x}
\newcommand{\yy}{\mathbf y}
\newcommand{\rr}{\mathbf r}
\newcommand{\ii}{\mathbf i}
\newcommand{\nn}{\mathbf n}
\newcommand{\pp}{\mathbf p}


\newcommand{\qq}{\mathbf q}
\newcommand{\uu}{\mathbf u}

\newcommand{\mm}{\mathbf m}
\newcommand{\fF}{\mathbf F}
\newcommand{\gG}{\mathbf G}
\newcommand{\eE}{\mathbf E}
\newcommand{\qQ}{\mathbf Q}
\newcommand{\tT}{\mathbf T}
\renewcommand{\tt}{\mathbf t}
\newcommand{\one}{\mathbf 1}
\newcommand{\zero}{\mathbf 0}

\renewcommand{\H}{\operatorname{H}}
\newcommand{\OO}{\operatorname{O}}
\newcommand{\oo}{\operatorname{o}}


%%%% Caligraphic Fonts - i.e. ????. %%%%%
\newcommand{\cA}{\mathcal{A}}
\newcommand{\cB}{\mathcal{B}}
\newcommand{\cC}{\mathcal{C}}
\newcommand{\cD}{\mathcal{D}}
\newcommand{\cE}{\mathcal{E}}
\newcommand{\cF}{\mathcal{F}}
\newcommand{\cG}{\mathcal{G}}
\newcommand{\cH}{\mathcal{H}} 
\newcommand{\cI}{\mathcal{I}}
\newcommand{\cJ}{\mathcal{J}}
\newcommand{\cK}{\mathcal{K}}
\newcommand{\cL}{\mathcal{L}}
\newcommand{\cM}{\mathcal{M}}
\newcommand{\cN}{\mathcal{N}}
\renewcommand{\O}{\mathcal{O}}
\newcommand{\cP}{\mathcal{P}}
\newcommand{\cQ}{\mathcal{Q}}
\newcommand{\cR}{\mathcal{R}}
\newcommand{\cS}{\mathcal{S}}
\newcommand{\cT}{\mathcal{T}}
\newcommand{\U}{\mathcal{U}} 		% Notice this is different
\newcommand{\cV}{\mathcal{V}}
\newcommand{\cW}{\mathcal{W}}
\newcommand{\cX}{\mathcal{X}}
\newcommand{\cY}{\mathcal{Y}}
\newcommand{\cZ}{\mathcal{Z}}

%%%% Blackboard Fonts - i.e. Real Numbers, Integers, etc. %%%%%
\newcommand{\A}{\mathbb{A}}
\newcommand{\B}{\mathbb{B}}
\newcommand{\C}{\mathbb{C}}
\newcommand{\D}{\mathbb{D}}
\newcommand{\E}{\mathbb{E}}
\newcommand{\F}{\mathbb{F}}
\newcommand{\G}{\mathbb{G}}
\newcommand{\I}{\mathbb{I}}
\newcommand{\J}{\mathbb{J}}
\newcommand{\K}{\mathbb{K}}
\renewcommand{\L}{\mathbb{L}}
\newcommand{\M}{\mathbb{M}}
\newcommand{\N}{\mathbb{N}}
\newcommand{\bO}{\mathbb{O}}		% Notice this is \bO
\renewcommand{\P}{\mathbb{P}}
\newcommand{\Q}{\mathbb{Q}}
\newcommand{\R}{\mathbb{R}}
\newcommand{\T}{\mathbb{T}}
\newcommand{\bU}{\mathbb{U}}		% Notice this is \bU
\newcommand{\V}{\mathbb{V}}
\newcommand{\W}{\mathbb{W}}
\newcommand{\X}{\mathbb{X}}
\newcommand{\Y}{\mathbb{Y}}
\newcommand{\Z}{\mathbb{Z}}

 %%%% Sarif Fonts - i.e. ???? %%%%%
\newcommand{\sA}{\mathsf{A}}
\newcommand{\sB}{\mathsf{B}}
\newcommand{\sC}{\mathsf{C}}
\newcommand{\sD}{\mathsf{D}}
\newcommand{\sE}{\mathsf{E}}
\newcommand{\sF}{\mathsf{F}}
\newcommand{\sG}{\mathsf{G}}
\newcommand{\sH}{\mathsf{H}} 
\newcommand{\sI}{\mathsf{I}}
\newcommand{\sJ}{\mathsf{J}}
\newcommand{\sK}{\mathsf{K}}
\newcommand{\sL}{\mathsf{L}}
\newcommand{\sM}{\mathsf{M}}
\newcommand{\sN}{\mathsf{N}}
\newcommand{\sO}{\mathsf{O}}
\newcommand{\sP}{\mathsf{P}}
\newcommand{\sQ}{\mathsf{Q}}
\newcommand{\sR}{\mathsf{R}}
\newcommand{\sS}{\mathsf{S}}
\newcommand{\sT}{\mathsf{T}}
\newcommand{\sU}{\mathsf{U}} 
\newcommand{\sV}{\mathsf{V}}
\newcommand{\sW}{\mathsf{W}}
\newcommand{\sX}{\mathsf{X}}
\newcommand{\sY}{\mathsf{Y}}
\newcommand{\sZ}{\mathsf{Z}}
 
 %%%% Fraktur Fonts - i.e. maximal ideals, prime ideals, etc. %%%%%
\newcommand{\cl}{\mathfrak{cl}}
\newcommand{\g}{\mathfrak{g}}
\newcommand{\h}{\mathfrak{h}}
\newcommand{\m}{\mathfrak{m}}
\newcommand{\n}{\mathfrak{n}}
\newcommand{\p}{\mathfrak{p}}
\newcommand{\q}{\mathfrak{q}}
\renewcommand{\r}{\mathfrak{r}}



\newcommand{\juliette}[1]{{\color{red} \sf $\spadesuit\spadesuit\spadesuit$ Juliette: [#1]}}


\title{Juliette Bruce's Research Statement}

%\author{Juliette Bruce}
%\address{Department of Mathematics, University of Wisconsin, Madison, WI}
%\email{\href{mailto:juliette.bruce@math.wisc.edu}{juliette.bruce@math.wisc.edu}}
%\urladdr{\url{http://math.wisc.edu/~juliettebruce/}}

%\thanks{The author was partially supported by the NSF GRFP under Grant No. DGE-1256259 and NSF grant DMS-1502553.}

%\subjclass[2010]{13D02, 14M25}

\begin{document} 

%\maketitle
\begingroup  
  \centering
  \large\scshape\bfseries Juliette Bruce's Research Statement\\[1em]
\endgroup

%\tableofcontents

\setcounter{section}{0}

My research interests lie in algebraic geometry, commutative algebra, and arithmetic geometry. In particular, I am interested in using homological and combinatorial methods to study the geometry of zero loci of systems of polynomials (i.e. algebraic varieties). I am also interested in studying the arithmetic properties of varieties over finite fields. Further, I am passionate about promoting inclusivity, diversity, and justice in the mathematics community. Broadly speaking my current research follows these ideas in two directions. 

\begin{itemize}[leftmargin=*]
\item \textbf{Homological Algebra on Toric Varieties:} A classical story in algebraic geometry is that homological methods and tools like minimal free resolutions and Castelnuovo--Mumford  regularity capture the geometry of subvarieties of projective space in nuanced ways. My work has sought to generalize this story by developing ways homological algebra can be used to study the geometry of toric varieties (i.e., ``nice' compactifications of the torus $(\C^{\times})^{n}$). 

\item \textbf{Cohomology of Moduli Spaces and Arithmetic Groups:} Despite its importance in algebraic geometry and number theory much remains unknown about the topology of $\cA_{g}$, the moduli space of abelian varieties of dimension $g$. I have been working to study a canonical ``part'' of the cohomology of $\mathcal{A}_{g}$, called the top-weight cohomology. This turns out to be closely connected to the study of cohomology of various arithmetic groups like $\GL_{g}(\Z)$ and $\Sp_{2g}(\Z)$, as well as the study of automorphic forms. 

\end{itemize}

\section{Homological Algebra on Toric Varieties}

Given a graded module $M$ over a graded ring $R$, a helpful tool for understanding the structure of $M$ is its minimal graded free resolution. In essence, a minimal graded free resolution is a way of approximating $M$ by a sequence of free $R$-modules. More formally, a \textit{graded free resolution} of a module $M$ is an exact sequence 
\[
\cdots \xrightarrow{} F_{k} \xrightarrow{d_{k}} F_{k-1} \xrightarrow{d_{k-1}} \cdots \xrightarrow{d_{1}} F_{0}\xrightarrow{\epsilon}M\xrightarrow{} 0
\]
where each $F_{i}$ is a graded free $R$-module, and hence can be written as $\bigoplus_{j}R(-j)^{\beta_{i,j}}$. The module $R(-j)$ is the ring $R$ with a twisted grading, so that $R(-j)_{d}$ is equal to $R_{d-j}$ where $R_{d-j}$ is the graded piece of degree $d-j$. The $\beta_{i,j}$'s are the \textit{Betti numbers} of $M$, and they count the number of $i$-syzygies of $M$ of degree $j$. We will use syzygy and Betti number interchangeably throughout. 

Given a projective variety $X$ embedded in $\P^r$, we associate to $X$ the ring $S_X=S/I_X$, where $S=\C[x_0,\ldots,x_r]$ and $I_X$ is the ideal of homogenous polynomials vanishing on $X$. As $S_X$ is naturally a graded $S$-module we may consider its minimal graded free resolution, which is often closely related to both the extrinsic and intrinsic geometry of $X$.  An example of this phenomenon
 is Green's Conjecture, which relates the Clifford index of a curve with the vanishing of certain $\beta_{i,j}$ for its canonical embedding \cite{voisin02, voisin05, aproduFarkas19}. See also \cite{eisenbud05}*{Conjecture 9.6} and \cite{schreyer86, bayerEisenbud91,farkasPopa05, farkas06,aproduFarkas11,farkasKemeny16,farkasKemeny17}.
 
 Much of my work can be viewed as understanding how minimal graded free resolutions capture the geometry when the role of $\P^{r}$ is replaced by another variety $Y$. In particular, I have focused on the case when $Y$ is a toric variety, i.e., a compactification of the torus $(\C^{\times})^{r}$ where the action of the torus extends to the boundary. Examples of toric varieties include projective space, products of projective spaces, and Hirzebruch surfaces. Work of Cox shows there is a correspondence between (toric) subvarieties of a fixed toric variety and quotients of a polynomial ring similar to the story discussed above for $\P^{r}$ \cite{cox95}. As such recent years have seen substantial work looking to use homological algebra and to better understand the geometry of toric varieties \cite{almousaBruce19,berkeschErmanSmith17,brownErman22,brownErman23,BB21,cartwrightErmanVelscoViray09,EES15,GVT15,maclaganSmith04,maclaganSmith05}.

%\begin{theorem}[\cite{voisin02}, \cite{voisin05}]
%Let $C$ be a generic smooth projective curve of genus $g$ over a characteristic zero field embedded in $\P^{g-1}$ by the complete canonical series. Then the length of the first linear strand of the minimal free resolution of $I_X$ is $g-3-\text{Cliff}(C)$.
%\end{theorem}

\subsection{Asymptotic Syzygies}

 Broadly speaking, asymptotic syzygies is the study of the graded Betti numbers (i.e. the syzygies) of a projective variety as the positivity of the embedding grows. In many ways, this perspective dates back to classical work on the defining equations of curves of high degree and projective normality \cite{mumford66, mumford70}. However, the modern viewpoint arose from the pioneering work of Green \cite{green84-I, green84-II} and later Ein and Lazarsfeld \cite{einLazarsfeld12}. 

To give a flavor of the results of asymptotic syzygies we will focus on the question: In what degrees do non-zero syzygies occur? Going forward we will let $X\subset \P^{r_{d}}$ be a smooth projective variety embedded by a very ample line bundle $L_{d}$. Following \cite{ermanYang18} we set, 
\begin{align*}
\rho_q\left(X,L_{d}\right)\;\;\coloneqq&\ \;\; \frac{\#\left\{p\in\N |\; \big| \; \beta_{p,p+q}\left(X,L_{d}\right)\neq0\right\}}{r_{d}},
\end{align*}
which is the percentage of degrees in which non-zero syzygies appear \cite{eisenbud05}*{Theorem~1.1}. The asymptotic perspective asks how $\rho_{q}(X;L_{d})$ behaves along the sequence of line bundles $(L_{d})_{d\in \N}$. 
%\begin{align*}
%\rho_q\left(X;L_{d}\right)\;\;\coloneqq&\ \;\; \frac{\#\left\{p\in\N |\; \big| \; \beta_{p,p+q}\left(X,L_{d}\right)\neq0\right\}}{r_{d}}.
%\end{align*}
%which by the Hilbert Syzygy Theorem is the percentage of degrees in which non-zero syzygies appear \cite{eisenbud05}*{Theorem~1.1}. For any particular, $X$, $L_{d}$, and $q$ computing $\rho_{q}(X;L_{d})$ is often quite difficult. The asymptotic perspective thus, asks instead, to consider a sequence of line bundles $(L_{d})_{d\in \N}$ and ask how $\rho_{q}(X;L_{d})$ behaves along the sequence of $(L_{d})_{d\in \N}$. 

With this notation in hand, we may phrase Green's work on the vanishing of syzygies for curves of high degree as computing the asymptotic percentage of non-zero quadratic syzygies. 

\begin{theorem}\cite{green84-I}
Let $X\subset \P^r$ be a smooth projective curve. If $(L_{d})_{d\in\N}$ is a sequence of very ample line bundles on $X$ such that $\deg L_{d} = d$ then 
\[
\lim_{d\to \infty} \rho_{2}\left(X;L_{d}\right) = 0.
\]
\end{theorem}

Put differently, asymptotically the syzygies of curves are as simple as possible, occurring in the lowest possible degree. This inspired substantial work, with the intuition being that syzygies become simpler as the positivity of the embedding increases \cite{ottavianiPaoletti01, einLazarsfeld93, lazarsfeldPareschiPopa11, pareschi00, pareschiPopa03, pareschiPopa04}.  

In a groundbreaking paper, Ein and Lazarsfeld showed that for higher dimensional varieties this intuition is often misleading. Contrary to the case of curves, they show that for higher dimensional varieties, asymptotically syzygies appear in every possible degree. 
  
\begin{theorem}\cite{einLazarsfeld12}*{Theorem~C}
Let $X\subset \P^r$ be a smooth projective variety, $\dim X \geq2$, and fix an index $1\leq q \leq \dim X$. If $(L_{d})_{d\in\N}$ is a sequence of very ample line bundles such that $L_{d+1}-L_{d}$ is constant and ample then
\[
\lim_{d\to\infty} \rho_{q}\left(X; L_d\right) = 1.
\]
\end{theorem}

My work has focused on the behavior of asymptotic syzygies when the condition that $L_{d+1}-L_{d}$ is constant and ample is weakened to assuming $L_{d+1}-L_{d}$ is semi-ample. Recall a line bundle $L$ is \textit{semi-ample} if $|kL|$ is base point free for $k\gg0$. The prototypical example of a semi-ample line bundle is $\O(1,0)$ on $\P^{n}\times \P^{m}$. My exploration of asymptotic syzygies in the setting of semi-ample growth thus began by proving the following nonvanishing result for $\P^{n}\times\P^{m}$ embedded by $\O(d_{1},d_{2})$. 

\begin{theorem}\cite{bruce19-semiample}*{Corollary~B}\label{thm:bruce-semiample}
Let $X=\P^{n}\times\P^{m}$ and fix an index $1\leq q \leq n+m$. There exist constants $C_{i,j}$ and $D_{i,j}$ such that
\[
\rho_{q}\left(X; \O\left(d_1,d_2\right)\right)\geq1-\sum_{\substack{i+j=q \\  i \leq n, \; j \leq m}}\left(
\frac{C_{i,j}}{d_1^id_2^j}+\frac{D_{i,j}}{d_1^{n-i}d_2^{m-j}}\right)-O\left(\begin{matrix}\text{lower ord.}\\ \text{terms}\end{matrix}\right).
\]
\end{theorem}

Notice if both $d_{1}\to \infty$ and $d_{2}\to\infty$ then $\rho_{q}\left(\P^{n}\times\P^{m}; \O(d_1,d_2)\right)\to1$, recovering the results of Ein and Lazarsfeld for $\P^n\times\P^m$. However, if $d_{1}$ is fixed and $d_{2}\to \infty$ (i.e. semi-ample growth) my results bound the asymptotic percentage of non-zero syzygies away from zero. This together with work of Lemmens \cite{lemmens18} has led me to conjecture that, unlike in previously studied cases, in the semi-ample setting $\rho_{q}\left(\P^{n}\times\P^{m}; \O(d_1,d_2)\right)$ does not approach 1. Proving this would require a vanishing result for asymptotic syzygies, which is open even in the ample case  \cite[Conjectures~7.1,~7.5]{einLazarsfeld12}.


% In particular, the asymptotic behavior is dependent, in a nuanced way, on the relationship between $d_{1}$ and $d_{2}$. 

%For example, considering $\P^{1}\times\P^{5}$ and $q=2$ then Theorem~\ref{thm:bruce-semiample} shows that 
%\[
%\rho_{2}\left(\P^{1}\times\P^{5}; \O(d_1,d_2)\right)\geq1-\frac{20}{d_2^2}-\frac{60}{d_1d_2^3}-\frac{5}{d_1d_2}-\frac{120}{d_2^4}-O\left(\begin{matrix}\text{lower ord.}\\ \text{terms}\end{matrix}\right)\,.
%\]
%Moreover, if $d_2$ is fixed and $d_1\to\infty$, then the limit of $\rho_{2}\left(\P^{1}\times\P^{5}; \O(d_1,d_2)\right)$ is greater than or equal to $1-\frac{20}{d^2_2}-\frac{120}{d_2^4}$.

%Results of Lemmens in the case of $\P^1\times\P^1$ together with my work has led me to conjecture that unlike in previously study cases (i.e. curves and ample growth) in the case of semi-ample growth $\rho_{q}\left(\P^{n}\times\P^{m}; \O(d_1,d_2)\right)$ does not approach 1 as $d_{1}\to \infty$. Proving this would require a vanishing result for asymptotic syzygies, which is open even in the ample case. See \cite[Conjecture~7.1, Conjecture~7.5]{einLazarsfeld12}.

The proof of Theorem~\ref{thm:bruce-semiample} is based upon generalizing the monomial methods of Ein, Erman, and Lazarsfeld. Such a generalization is complicated by the difference between the Cox ring and homogenous coordinate ring of $\P^{n}\times\P^{m}$. A central theme in this work is to exploit the fact that a key regular sequence I use has a number of non-trivial symmetries. 
%These symmetries, when combined with a series of spectral sequence arguments, allow me to prove Theorem~\ref{thm:bruce-semiample}.

This work suggests that the theory of asymptotic syzygies in the setting of semi-ample growth is rich and substantially different from the other previously studied cases. Going forward I plan to use this work as a jumping-off point for the following question.  

\begin{question}\label{quest:semi-ample}
Let $X\subset \P^{r_d}$ be a smooth projective variety and fix an index $1\leq q \leq \dim X$. Let $(L_{d})_{d\in\N}$ be a sequence of very ample line bundles such that $L_{d+1}-L_{d}$ is constant and semi-ample, can one compute $\lim_{d\to\infty} \rho_{q}\left(X;L_{d}\right)$?
\end{question}

A natural next case in which to consider Question~\ref{quest:semi-ample} is that of Hirzebruch surfaces. I addressed a different, but related question for a narrow class of Hirzebruch surfaces in \cite{bruce19-hirzebruch}.

\subsection{Syzygies via Highly Distributed Computing}

It is quite difficult to compute examples of syzygies. For example, until recently the syzygies of the projective plane embedded by the $d$-uple Veronese embedding were only known for $d\leq 5$. My co-authors and I exploited recent advances in numerical linear algebra and high-throughput high-performance computing to generate a number of new examples of Veronese syzygies. A follow-up project used similar computational approaches to compute the syzygies of $\P^{1}\times\P^{1}$ in over 200 new examples. This data provided support for several existing conjectures and led to a number of new conjectures \cite{bruceErmanGoldsteinYang18,bruceErman19,BCEGLY22}.  %The resulting data is publicly available via the website SyzygyData and a package for Macaualy2 \cite{bruceErman19, M2}.


\subsection{Multigraded Castelnuovo–Mumford Regularity}\label{subsec:prior-mgreg}

Introduced by Mumford, the Castelnuovo–Mumford Regularity of a projective variety $X\subset \P^{r}$ is a measure of the complexity of $X$ given in terms of the vanishing of certain cohomology groups of $X$. Roughly speaking one should think about Castelnuovo--Mumford regularity as being a numerical measure of geometric complexity. Mumford was interested in such a measure as it plays a key role in constructing Hilbert and Quot schemes. In particular, being $d$-regular implies that $\cF(d)$ is globally generated. However, Eisenbud and Goto showed that regularity is also closely connected to interesting homological properties.


%Such a measure can be easily extended to modules over a standard graded polynomial ring $S=\C[x_{0},\ldots,x_{n}]$ by requiring the analogous vanishing conditions for local cohomology. 

%
%\begin{defn}
%A coherent sheaf $\cF$ on $\P^{n}$ is $d$-regular if and only if:
%\[
%H^{i}(\P^{n}, \cF(d-i))=0 \quad \quad \quad \text{for all $i>0$}.
%\] 
%The Castelnuovo--Mumford regularity of $\cF$ is then 
%\[
%\reg(\cF) \coloneqq \min\left\{ d\in \Z \;\; \big| \;\; \text{$\cF$ is $d$-regular}\right\}.
%\]
%\end{defn}

%Mumford was interested in such a measure as it plays a key role in constructing Hilbert and Quot schemes. In particular, being $d$-regular implies that $\cF(d)$ is globally generated. However, Eisenbud and Goto showed that regularity is also closely connected to interesting homological properties.

\begin{theorem}\cite{eisenbudGoto84}\label{thm:eisenbud-goto}
Let $\cF$ be a coherent sheaf on $\P^{r}$ and $M=\bigoplus_{e\in\Z} H^0(\P^{r},\cF(e))$ the corresponding section ring. The following are equivalent:
\begin{enumerate}
\item $M$ is $d$-regular;
\item $\beta_{p,q}(M)=0$ for all $p\geq0$ and $q>d+i$;
\item $M_{\geq d}$ has a linear resolution. 
\end{enumerate}
\end{theorem}

Maclagan and Smith introduced multigraded Castelnuovo--Mumford regularity, where $\P^{r}$ can be replaced by any toric variety. Similarly to the definition in the classical setting multigraded Castelnuovo--Mumford regularity is defined in terms of the vanishing of certain cohomology groups, however, the multigraded Castelnuovo--Mumford regularity of a subvariety or module is not a single number, but instead an infinite subset of $\Z^{r}$. 

As an example, let us consider the case of products of projective spaces. Fixing a dimension vector $\nn=(n_1,n_2,\ldots,n_{r})\in \N^{r}$ we let $\P^{\nn}\coloneqq \P^{n_1}\times \P^{n_2}\times \cdots \times \P^{n_r}$ and $S=\K[x_{i,j} \; |\; 1\leq i \leq r, 0\leq j \leq n_{i}]$ be the Cox ring of $\P^{\nn}$ with the $\Pic(X)\cong \Z^{r}$-grading given by $\deg x_{i,j} = \ee_{i} \in \Z^{r}$, where $\ee_{i}$ is the $i$-th standard basis vector in $\Z^{r}$. Fixing some notation given $\dd\in \Z^{r}$ and $i\in \Z_{\geq0}$ we let:
\[
L_{i}(\dd)\coloneqq \bigcup_{\substack{\vv \in \N, |\vv| = i}} (\dd-\vv)+\N^{r}.
\]
Note when $r=2$ the region $L_{i}(\dd)$ looks like a staircase with $(i+1)$-corners. Roughly speaking we define regularity by requiring the $i$-th cohomology of certain twists of $\cF$ to vanish on $L_{i}$. 

 


%to this setting in terms of certain cohomology vanishing. Fixing some notation given $\dd\in \Z^{r}$ and $i\in \Z_{\geq0}$ we let:
%\[
%L_{i}(\dd)\coloneqq \bigcup_{\substack{\vv \in \N \\ |\vv| = i}} (\dd-\vv)+\N^{r}.
%\]
%Note when $r=2$ the region $L_{i}(\dd)$ looks like a staircase with $(i+1)$-corners. Roughly speaking we define regularity by requiring the $i$-th cohomology of certain twists of $\cF$ to vanish on $L_{i}$. 
%
%My collaborators and I have worked to generalize this result to the multigraded setting, i.e. from coherent sheaves on a single projective space to sheaves on a product of projective spaces. In particular, fixing a dimension vector $\nn=(n_1,n_2,\ldots,n_{r})\in \N^{r}$ we let $\P^{\nn}\coloneqq \P^{n_1}\times \P^{n_2}\times \cdots \times \P^{n_r}$ and $S=\K[x_{i,j} \; |\; 1\leq i \leq r, 0\leq j \leq n_{i}]$ be the Cox ring of $\P^{\nn}$ with the $\Pic(X)\cong \Z^{r}$-grading given by $\deg x_{i,j} = \ee_{i} \in \Z^{r}$, where $\ee_{i}$ is the $i$-th standard basis vector in $\Z^{r}$. 



%%\begin{center}
%%\begin{figure}[H]
%%\newcommand{\makegrid}{
%%  \path[use as bounding box] (-3.45,-3.25) rectangle (5.45,5.25);
%%  \foreach \x in {-4,...,4}
%%  \foreach \y in {-4,...,4}
%%    { \fill[Gray,fill=gray] (\x,\y) circle (1.5pt); }
%%  \draw[-,  semithick] (-4,0)--(4,0);
%%  \draw[-,  semithick] (0,-4)--(0,4);
%%}
%%%%%%%%%%%%%%%%%%%%%%%%%%%%%%%%%%%%%%
%%%%%%%%%%%%%%%%%%%%%%%%%%%%%%%%%%%%%%
%%%%%%%%%%%%%%%%%%%%%%%%%%%%%%%%%%%%%%
%%\begin{tikzpicture}[scale=.3]
%%  \path[fill=Gray!45] (-1,4)--(-1,0)--(0,0)--(0,-1)--(4,-1)--(4,4)--(-1,4);
%%  \makegrid
%%  \draw[->, ultra thick] (-1,0)--(-1,4);
%%  \draw[-, cap=round,ultra thick] (-1,0)--(0,0)--(0,-1);
%%    \draw[->, ultra thick] (0,-1)--(4,-1);
%%  %\fill[Gray,fill=Gray] (-1,0) circle (6pt);
%% % \fill[Gray,fill=Gray] (0,-1) circle (6pt);
%%  \fill[Gray,fill=Black] (0,0) circle (6pt);
%%\end{tikzpicture}\quad\;
%%%%%%%%%%%%%%%%%%%%%%%%%%%%%%%%%%%%%%
%%%%%%%%%%%%%%%%%%%%%%%%%%%%%%%%%%%%%%
%%%%%%%%%%%%%%%%%%%%%%%%%%%%%%%%%%%%%%
%%\begin{tikzpicture}[scale=.3]
%%  \path[fill=Gray!45] (-2,4)--(-2,0)--(-1,0)--(-1,-1)--(0,-1)--(0,-2)--(4,-2)--(4,4)--(-2,4);
%%  \makegrid
%%  \draw[->, ultra thick] (-2,0)--(-2,4);
%%  \draw[-, cap=round,ultra thick] (-2,0)--(-1,0)--(-1,-1)--(0,-1)--(0,-2);
%%    \draw[->, ultra thick] (0,-2)--(4,-2);
%%%  \fill[Gray,fill=Gray] (-2,0) circle (6pt);
%% % \fill[Gray,fill=Gray] (-1,-1) circle (6pt);
%% % \fill[Gray,fill=Gray] (0,-2) circle (6pt);
%%  \fill[Gray,fill=Black] (0,0) circle (6pt);
%%\end{tikzpicture}\quad\;
%%\begin{tikzpicture}[scale=.3]
%%  \path[fill=Gray!45] (-3,4)--(-3,0)--(-2,0)--(-3,0)--(-2,0)--(-2,-1)--(-1,-1)--(-1,-2)--(0,-2)--(0,-3)--(4,-3)--(4,4)--(-2,4);
%%  \makegrid
%%  \draw[->, ultra thick] (-3,0)--(-3,4);
%%  \draw[-, cap=round,ultra thick] (-3,0)--(-2,0)--(-2,-1)--(-1,-1)--(-1,-2)--(0,-2)--(0,-3);
%%      \draw[->, ultra thick] (0,-3)--(4,-3);
%%%  \fill[Gray,fill=Gray] (-2,0) circle (6pt);
%% % \fill[Gray,fill=Gray] (-1,-1) circle (6pt);
%% % \fill[Gray,fill=Gray] (0,-2) circle (6pt);
%%  \fill[Gray,fill=Black] (0,0) circle (6pt);
%%\end{tikzpicture}
%%%
%%%\caption{Letting $r=2$ the regions $L_{1}(0,0)$, $L_{2}(0,0)$, and $L_{3}(0,0)$.}
%%\end{figure}
%%\end{center}

\begin{defn}\cite{maclaganSmith04}*{Definition 6.1}\label{def:mg-reg}
A coherent sheaf $\cF$ on $\P^{\nn}$ is $\dd$-regular if and only if
\[
H^i\left(\P^{\nn}, \cF(\ee)\right) =0 \quad \quad \quad \text{for all $\ee\in L_{i}(\dd)$}.
\]
The multigraded Castelnuovo--Mumford regularity of $\cF$ is then the set: 
\[
\reg(\cF) \coloneqq \left \{ \dd\in \Z^{r} \;\; \big| \;\; \text{$\cF$ is $\dd$-regular}\right\}\subset \Z^{r}.
\]
\end{defn}

%Even for relatively simple examples the multigraded Castelnuovo--Mumford regularity does not necessarily have a unique minimal element. That said $\reg(\cF)$ does have the structure of a module over the semi-group $\Nef(\P^{\nn})\cong\N^{r}$, i.e. if $\dd \in \reg(\cF)$ then $\dd+\ee\in \reg(\cF)$ for all $\ee\in \N^{r}$. 

%\begin{center}
%\begin{figure}\label{fig:example-of-reg}
%\newcommand{\makegrid}{
%  \path[use as bounding box] (-3.45,-3.25) rectangle (5.45,5.25);
%  \foreach \x in {-1,...,5}
%  \foreach \y in {-1,...,5}
%    { \fill[Gray,fill=gray] (\x,\y) circle (1.5pt); }
%  \draw[-,  semithick] (-1,0)--(5,0);
%  \draw[-,  semithick] (0,-1)--(0,5);
%}
%%%%%%%%%%%%%%%%%%%%%%%%%%%%%%%%%%%%%
%%%%%%%%%%%%%%%%%%%%%%%%%%%%%%%%%%%%%
%%%%%%%%%%%%%%%%%%%%%%%%%%%%%%%%%%%%%
%\begin{tikzpicture}[scale=.35]
%  \path[fill=Gray!45] (5,0)--(2,0)--(2,1)--(1,1)--(1,2)--(0,2)--(0,5)--(5,5);
%  \makegrid
%  \draw[->, ultra thick] (2,0)--(5,0);
%  \draw[-, cap=round,ultra thick] (2,0)--(2,1)--(1,1)--(1,2)--(0,2);
%    \draw[->, ultra thick] (0,2)--(0,5);
%  %\fill[Gray,fill=Gray] (-1,0) circle (6pt);
% % \fill[Gray,fill=Gray] (0,-1) circle (6pt);
%\end{tikzpicture}
%\caption{The multigraded Castelnuovo--Mumford regularity of $\O_{X}$ where $X\subset \P^{1}\times \P^{1}$ is the subscheme consisting of three distinct points $([1:1],[1:4])$, $([1:2],[1:5])$, and $([1:3],[1:6])$.}
%\end{figure}
%\end{center}

The obvious approaches to generalize Theorem~\ref{thm:eisenbud-goto} to a product of projective spaces turn out not to work. For example, the multigraded Betti numbers do not determine multigraded Castelnuovo--Mumford regularity \cite[Example 5.1]{bruceHellerSayrafi21} Despite this we show that part (3) of Theorem~\ref{thm:eisenbud-goto} can be generalized. To do so we introduce the following generalization of linear resolutions. 

\begin{defn}
A complex $F_{\bullet}$ of $\Z^{r}$-graded free $S$-modules is $\dd$-quasilinear if and only if $F_{0}$ is generated in degree $\dd$ and each twist of $F_{i}$ is contained in $L_{i-1}(\dd-\one)$.
%\begin{enumerate}
%\item We say that  $F_{\bullet}$ is $\dd$-linear if and only if $F_{0}$ is generated in degree $\dd$ and each twist of $F_{i}$ is contained in $L_{i}(\dd)$. 
%\item We say that $F_{\bullet}$ is $\dd$-quasilinear if and only if $F_{0}$ is generated in degree $\dd$ and each twist of $F_{i}$ is contained in $L_{i-1}(\dd-\one)$. 
%\end{enumerate}
\end{defn}

%In order to see the difference between linear and quasilinear resolutions we note that on a product of projective spaces the irrelevant ideal generally will have a quasilinear resolution, not a linear resolution.  For example, if we consider $\P^{1}\times \P^{2}$ so that $S=\K[x_{0},x_{1},y_{0},y_{1},y_{2}]$ and $B=\langle x_{0},x_{1}\rangle\cap\langle y_{0},y_{1},y_{2}\rangle$ then the minimal graded free resolution of $S/B$ is: 
%	\[\begin{tikzcd}[column sep=1.75em]
%	S & \lar S(-1,-1)^6 & \lar
%	  \begin{matrix}
%	    S(-1,-2)^6\\[-3pt]
%	    \oplus \\[-3pt]
%	    S(-2,-1)^3
%	  \end{matrix}
%	  &
%	  \lar
%	  \begin{matrix}
%	    S(-1,-3)^2\\[-3pt]
%	    \oplus \\[-3pt]
%	    S(-2,-2)^3
%	  \end{matrix}
%	  &\lar
%	  S(-2,-3)
%	  & \lar 0.
%	\end{tikzcd}\]
%In particular, we see that the minimal graded free resolution $S/B$ is not $(0,0)$-linear since $(-1,-1) \not\in L_1(0,0)$, however, it is $(0,0)$-quasilinear. 
%
%It is not the case that $M$ being $\dd$-regular implies $M_{\geq \dd}$ has a linear resolution \cite[Example 4.2]{bruce21}, however, we can characterize being $\dd$-regular in terms of $M_{\geq \dd}$ having a quasilinear resolution. 

\begin{theorem}\cite{bruceHellerSayrafi21}*{Theorem A}\label{thm:mgreg-main}
Let $M$ be a (saturated) finitely generated $\Z^{r}$-graded $S$-module:
\[
\text{$M$ is $\dd$-regular} \iff  \text{$M_{\geq\dd}$ has a $\dd$-quasilinear resolution}.
\]
\end{theorem}

The proof of Theorem~\ref{thm:mgreg-main} is based in part on a spectral sequence argument that relates the Betti numbers of $M_{\geq\dd}$ to the Fourier--Mukai transform of $\widetilde{M}$ with Beilinson's resolution of the diagonal as the kernel.  Recent breakthroughs \cite{HHL23, brownErman23-2} understanding resolutions of the diagonal on arbitrary toric varieties mean that there is hope one may be able to generalize the above argument to arbitrary toric varieties. With this in mind, I am interested in pursuing the following question

%Precisely, if $M$ is $\dd$-regular and $H_B^0(M)=0$ we prove the that
%\begin{align*}\label{eq:magic-equality}
%  \dim_{\C}\Tor^S_j(M_{\geq\dd}, \C)_\aa = h^{|\aa|-j}\big(\P^{\nn}, \widetilde{M}\otimes\O_{\P^{\nn}}^\aa(\aa)\big) \quad \text{for } |\aa|\geq j\geq 0,
%\end{align*}
%where the $\O_{\P^{\nn}}^\aa$ are cotangent sheaves on $\P^\nn$. The result then follows from showing that $M$ being $\dd$-regular is equivalent to certain vanishings of the right-hand side above. 

\begin{question}
How can Theorem~\ref{thm:mgreg-main} be generalized to arbitrary smooth projective toric varieties? in particular, what is the correct definition of quasilinear resolutions?
\end{question}

%We briefly sketching the proof of the above theorem:
%\begin{enumerate}
%\item Using a Fourier-Mukai argument we construct a complex $G_{\bullet}$ of free $\Z^{r}$-graded $S$-modules whose multigraded Betti numbers are given (in some range) as follows:
%\[
%\beta_{i,\aa}\left(G_{\bullet}\right) = \dim H^{|\aa|-i}\left(\P^{\nn}, \tilde{M}\otimes \Omega^{\aa}_{\P^{\nn}}(\aa)\right).
%\]
%\item Making use of a spectral sequence argument we show that even though $G_{\bullet}$ is not a priori a resolution of $M_{\geq\dd}$ we have that:
%\[
%\beta_{i,\aa}\left(M_{\geq\dd}\right) = \beta_{i,\aa}\left(G_{\bullet}\right).
%\]
%\item Finally, we characterize $M$ being $\dd$-regular in terms of the vanishing of the cohomology in (1) above.
%\end{enumerate}
%
%Note the complex $G_{\bullet}$ constructed in part (1) of the proof sketch above is a priori not a resolution of $M_{\geq\dd}$, but instead is a virtual resolution of $M$ \cite{berkesch17}. That said as noted above it does have the same Betti numbers as $M_{\geq\dd}$, and in all the examples we have done it turns out to be a resolution.
%
%
%


\subsubsection{Multigraded Regularity of Powers of Ideals}

Building on the work of many people \cite{bertramEinLazarsfeld91,chandler97}, Cutkosky, Herzog, Trung \cite{cutkoskyHerzogTrung99} and independently Kodiyalam \cite{kodiyalam00} showed the Castelnuovo--Mumford regularity for powers of ideals on a projective space $\P^r$ has surprisingly predictable asymptotic behavior. In particular, given an ideal $I\subset \K[x_0,\ldots,x_r]$, there exist constants $d,e\in\Z$ such that $\reg\!\left(I^t\right) = dt+e$ for $t\gg0$.

Building upon our work discussed above, my collaborators and I generalized this result to arbitrary toric varieties. In particular, Definition~\ref{def:mg-reg} can be extended to all toric varieties by letting $S$ be Cox ring of the toric variety $X$, replacing $\Z^r$ with the Picard group of $X$, and replacing $\N^{r}$ with the nef cone of $X$. My collaborators and I show that the multigraded regularity of powers of ideals is bounded and translates in a predictable way. In particular, the regularity of $I^{t}$ essentially translates within $\Nef X$ in fixed directions at a linear rate.

%The rough idea is that Definition~\ref{def:mg-reg} generalizes verbatim where $S$ is replaced with the Cox ring of the toric variety $X$, $\Z^{r}$ is replaced by the Picard group of $X$ an $\N^{r}$ is replaced by the nef cone of $X$> In this setting the regularity of a $\Pic(X)$-graded module is a subset of $\Pic X$ that is closed under the addition of nef divisors. \juliette{finish this pragraph}

 

\begin{theorem}\cite{bruceHellerSayrafi22}*{Theorem 4.1}
  There exists a degree $\aa\in\Pic X$, depending only on $I$, such that for each integer $t>0$ and each pair of degrees $\qq_1,\qq_2\in\Pic X$ satisfying $\qq_1\geq\deg f_i\geq\qq_2$ for all generators $f_i$ of $I$, we have
	\[ t\qq_1+\aa+\reg S \subseteq \reg\!\left(I^t\right) \subseteq t\qq_2+\Nef X. \]
\end{theorem}

A key aspect of the proof of this theorem is showing that the multigraded regularity of an ideal is finitely generated, in the sense that there exist vectors $\vv,\ww\in \Z^r$ such that $\vv+\Nef X \subset \reg(I) \subset \ww + \Nef X$. Perhaps somewhat surprisingly, my co-authors and I showed that this can fail for arbitrary modules \cite{bruceHellerSayrafi22}. This naturally raises the question of whether one can characterize when multigraded regularity is finitely generated. 


\begin{question}\label{quest:finitegen}
Let $X$ be a smooth projective toric variety. Can one characterize when $\reg(M)$ is finitely generated for a module $M$ over the Cox ring of $X$?
\end{question} 

An first case of this question that I think would make a lovely first research project for a student is to attempt to answer Question~\ref{quest:finitegen} when $M$ is the Cox ring of a torus fixed-point. In this special case, the question reduces to a delicate combinatorial question about vector partition functions.




\section{Cohomology of Moduli Spaces and Arithmetic Groups}

Some of the most classical objects in algebraic geometry are moduli spaces, i.e., spaces that parameterize a given collection of geometric objects. The quintessential example of a moduli space is $\cM_{g}$, the moduli space of (smooth) genus $g$ curves, also known as the moduli space of compact Riemann surfaces of genus $g$. Despite their classical nature, much remains unknown about the geometry of many moduli spaces. For example, the rational cohomology of $\cM_{g}$ is only known for $g\leq 4$. However, classical results suggest that $\cM_{g}$ should have a lot of cohomology because its Euler characteristic grows super exponentially. Recent groundbreaking work of Chan, Galatius, and Payne has shed the first direct light on this phenomenon by constructing new non-trivial cohomology classes, and showing that the dimension of certain cohomology groups of $\cM_{g}$ grow at least exponentially. 

\begin{theorem}\cite{CGP21}*{Theorem 1.1}\label{thm:Mg}
For $g\geq2$ the dimension of $H^{4g-6}(\cM_{g};\Q)$ grows at least exponentially. In particular $\dim H^{4g-6}\left(\cM_{g}; \Q\right) > \beta^{g}$ for any real number $\beta<\beta_{0}$ where $\beta_{0}\approx1.3247\ldots$ is the real solution of $t^3-t-1=0$. 
\end{theorem}

Much of my recent work has sought to build up the groundwork laid by Chan, Galatius, and Payne to study the rational cohomology of other moduli spaces. Of particular, interest to me has been the moduli space of abelian varieties and various generalizations. This work has deep connections to the cohomology of various arithmetic groups like $\Sp_{2g}(\Z)$ and  $\GL_{g}(\Z)$. 

%Their proof of this theorem is related upon 


%The situation is analogous to that of the moduli space of curves $\cM_g$, which is a rational classifying space for the mapping class group $\mathrm{Mod}_g$ via its action on Teichm\"uller space.  



\subsection{Cohomology of $\cA_{g}$}

The moduli space of (principally polarized) abelian varieties of dimension $g$, is a smooth variety $\cA_{g}$ (truthfully a smooth Delinge--Mumford stack) whose points are in one to one correspondence with isomorphism classes of principally polarized abelian varieties of dimension $g$. Concretely, we may view it as the quotient $[\mathbb{H}_g/\mathrm{Sp}_{2g}(\Z)]$ where $\H_{g}$ is the Siegel upper half-space. Notice this means that $\cA_{g}$  is a rational classifying space for the integral symplectic group $\mathrm{Sp}_{2g}(\Z)$. 

Similar to the moduli space of curves $\cA_{g}$ has long been studied, but much remains unknown about its geometry. For example, the (singular) cohomology is $\cA_{g}$ is only fully known for $g\leq 3$, with $g=0,1$ being relatively easy, $g=2$ which is a classical result of Igusa, and $g=3$ due to work of Hain. In fact the cohomology of $\cA_{g}$ is so mysterious until recently work by myself and co-authors it was unknown whether $H^{2i+1}(\cA_{g};\Q)\neq0$ for some $g$ and $i$. This was a question posed by Gruskestkky that my recent work answered. 

Building upon the work of Chan, Galatius, and Payne, my co-authors and I developed new methods for understanding a certain canonical quotient of the cohomology of $\cA_{g}$. In particular, our results construct non-trivial cohomology classes in $H^{k}(\cA_{g}; \Q)$ in a number of new cases. 

\begin{theorem}\cite{BBCMMW22}*{Theorem A}\label{thm:Ag}
The rational cohomology $H^{k}\left(\cA_{g};\Q\right)\neq0$ for:
\[
\text{$(g,k)=(5,15),(5,20),(6,30),(7,28),(7,33),(7,37)$, and $(7,42)$}.
\]
\end{theorem}

For broader context, since $\cA_g$ is a rational classifying space for $\mathrm{Sp}_{2g}(\Z)$ there is natural isomorphism $H^*(\cA_g;\Q) \cong H^*(\mathrm{Sp}_{2g}(\Z);\Q)$. In particular, the above results provide new non-vanishing results for $H^*(\mathrm{Sp}_{2g}(\Z);\Q)$. However, my work takes advantage of the fact that since $\cA_g$  is a smooth and separated Deligne Mumford stack with a coarse moduli space which is an algebraic variety, permitting Deligne's mixed Hodge theory to be applied to study the rational cohomology of these groups. In particular, the rational cohomology of a %smooth 
complex algebraic variety $X$ of dimension $d$ admits a weight filtration with graded pieces $\Gr_{j}^W\!H^k(X;\mathbb{Q})$. %with $j$ ranging from %$k$ 
%$0$ to ${\rm min}\{2k, 2d\}$. 
As $\Gr_{j}^W\!H^k(X;\mathbb{Q})$ vanishes whenever $j>2d$, $\Gr_{2d}^W\!H^k(X;\mathbb{Q})$ is referred to as the {\em top-weight} part of $H^k(X;\Q)$.)  In this way we deduce Theorem~\ref{thm:Ag} above as a corollary to computing the top-weight cohomology of $\cA_{g}$ for all $g\leq 7$.  




%The situation is analogous to that of the moduli space of curves $\cM_g$, which is a rational classifying space for the mapping class group $\mathrm{Mod}_g$ via its action on Teichm\"uller space.  Moreover, in both cases, we find ourselves in the advantageous situation that $\cM_g$ and $\cA_g$  are smooth and separated Deligne Mumford stacks with coarse moduli spaces which are algebraic varieties, permitting Deligne's mixed Hodge theory to be applied to study the rational cohomology of these groups. 



\subsection{Cohomology of $\cA_{g}(m)$}

The moduli space $\cA_{g}$ actually a special instance of the moduli space of (principally polarized) abelian varieties of dimension $g$ with level $m$-structure. Denoted by $\cA_{g}(m)$, we may view it as the quotient $[\mathbb{H}_g/\mathrm{Sp}_{2g}(\Z)](m)$  where $\Sp_{2g}(\Z)(m)$ is the principal congruence subgroup $\ker(\Sp_{2g}(\Z) \to \Sp_{2g}(\Z/m\Z)$. Note that when $m=1$, we have that $\cA_{g}(m)$ is isomorphic to $\cA_{g}$. From this perspective, one may hope to generalize Theorem~\ref{thm:Ag} and underlying methods my co-authors and I developed in \cite{BBCMMW22} to studying the rational cohomology of  $\cA_{g}(m)$ and $\Sp_{2g}(\Z)(m)$. In ongoing work, Melody Chan and I are developing such generalizations.

\begin{goalTheorem}\label{goalThm:agm}
Let $d=\binom{g+1}{2}$ be the dimension of $\cA_{g}(m)$. For any integers $m\geq1$ and $g\geq0$ there exists a cellular complex $LA_{g}(m)^{\trop}$ such that for all $i\geq0$ there is a natural isomorphism
\[
\tilde{H}_{i-1}\left(LA_{g}(m)^{\trop};\Q\right) \cong \Gr_{2d}^W\!H^{i}\left(\cA_{g}(m);\mathbb{Q}\right),
\]
\end{goalTheorem}

The methods behind  Goal Theorem~\ref{goalThm:agm} show new connections between the cohomology of $\cA_{g}(m)$ and the cohomology of $\GL_{g}(\Z)(m)$. The cohomology of $\Sp(2g,\Z)(m)$ -- and hence  $\cA_{g}(m)$ -- and $\GL_{g}(\Z)(m)$ are closely connected to automorphic forms. Thus it is natural to wonder whether our methods for computing the top-weight cohomology of $\cA_{g}(m)$ shed new light on automorphic forms. In particular, since the top-weight cohomology of $\cA_{g}(m)$ comes from understanding the boundary of a locally symmetric space, one may hope it is related to Siegel--Eisenstein series. In an ongoing conversation with, Melody Chan, and Peter Sarnak we hope to address this question.

\begin{question}
What is the relationship between the top-weight cohomology of $\cA_{g}(m)$ and Siegel Eisenstein series?
\end{question}

\subsection{Matroid Complexes and Cohomology of $\cA_{g}^{\text{mat}}$}

A key step in the proof of Theorem~\ref{thm:Ag} is constructing a chain complex $P^{(g)}_{\bullet}$ whose homology is precisely the top-weight cohomology of $\cA_{g}$. A major hurdle to pushing our results on the cohomology of $\cA_{g}$, further, is that this chain complex very quickly becomes extremely large and complicated. However,  with my co-authors, I identified a subcomplex $R^{(g)}_{\bullet} \subset P^{(g)}_{\bullet}$, called the regular matroid complex, which has rich combinatorics. In particular, $R^{(g)}_{k}$ is spanned by isomorphism classes of regular matroids on $k$ elements of rank $\leq g$.  I am working to study this complex from a number of perspectives. As an example, the following goal theorem is a result that I am working on with three graduate students.

\begin{goalTheorem}
Compute the homology of the matroid complex $R_{\bullet}^{(g)}$ for all $g\geq 14$.
\end{goalTheorem}

Currently by combining theoretical results and large-scale computations to compute the cohomology for all $g\leq 9$. Computing the homology of the regular matroid complex is interesting, not only because it provides a new approach for studying the combinatorics of matroids, but also because it is closely related to the cohomology of partial compactification of $\cA_{g}$ called the matroidal (partial) compactification $\cA_{g}^{\mat}$. In ongoing work with Madeline Brandy and Daniel Corey, I am looking to show that one can compute the top-weight cohomology of $\cA_{g}^{\mat}$ from the regular matroid complex. 

\begin{goalTheorem}
Compute the top-weigh cohomology of $\cA_{g}^{\mat}$ for all $g\leq 10$. 
\end{goalTheorem}

Work of Willwacher \cite{willwacher15} and Kpntsevich \cite{kontsevich93, kontsevich94} on graph complexes suggests that one may hope for $R_{\bullet}^{(g)}$ to have rich algebraic structure beyond just that of chain complex.

\begin{question}
Does the complex $R_{\bullet}^{(g)}$ carry a natural Lie bracket, endowing it with the structure of a differentially graded Lie algebra?
\end{question}

Constructing such a Lie bracket likely relies on developing a new understanding of the ways one can combine two matroids. Ongoing work with the graduate students mentioned above is studying this problem in the special cases of graphic and co-graphic matroids. The existence of similar Lie structure was crucial to achieving the exponential bounds in Theorem~\ref{thm:Mg}.


%
%\section{Varieties over Finite Fields}
%
%Over a finite field, a number of classical statements from algebraic geometry no longer hold. For example, if $X\subset\P^r$ is a smooth projective variety of dimension $n$ over $\C$, Bertini's theorem states that, if $H\subset \P^r$ is a generic hyperplane, then $X\cap H$ is smooth of dimension $n-1$. Famously, however, this fails if $\C$ is replaced by a finite field $\fF_{q}$. Using an ingenious probabilistic sieving argument, Poonen showed that if one is willing to replace the role of hyperplanes by hypersurfaces of arbitrarily large degree, then a version of Bertini's theorem is true \cite{poonen04}. More specifically Poonen showed that as, $d\to\infty$, the percentage of hypersurfaces $H\subset \P_{\fF_{q}}^{r}$ of degree $d$ such that $X\cap H$ is smooth is determined by the Hasse-Weil zeta function of $X$. Below we write $\fF_{q}[x_{0},\ldots,x_{r}]_{d}$ for the $\fF_{q}$-vector space of homogenous polynomials of degree $d$. 
%
%\begin{theorem}\cite{poonen04}*{Theorem~1.1}\label{thm:poonen}
%Let $X\subset \P^{r}_{\fF_{q}}$ be a smooth variety of dimension $n$. Then:
%\begin{equation}\label{eq:poonen}
%\lim_{d\to \infty} \Prob\left(\begin{matrix} f\in \fF_{q}[x_{0},x_{1},\ldots,x_{r}]_{d}\\ \text{$X\cap \V(f)$ is smooth of dimension $n-1$}\end{matrix}\right)=
%\zeta_X(n+1)^{-1} >0.
%\end{equation}
%\end{theorem}
%
%\subsection{A Probabilistic Study of Systems of Parameters} 
%
%Given an $n$ dimensional projective variety $X\subset \P^r$, a collection of homogenous polynomials $f_{0},f_{1},\ldots,f_{k}$ of degree $d$ is a (partial) system of parameters if $\dim X\cap \V(f_{0},f_{1},\ldots,f_{k}) = \dim X - (k+1)$. Systems of parameters are closely tied to Noether normalization, as the existence of a finite (i.e. surjective with finite fibers) map $X\rightarrow \P^n$ is equivalent to the existence of a system of parameters of length $n+1$.
%
%Inspired by work of Poonen \cite{poonen04} and Bucur and Kedlaya \cite{bucurKedlaya12}, Daniel Erman and I computed the asymptotic probability that randomly chosen homogenous polynomials $f_{0},f_{1},\ldots,f_{k}$ over $\fF_{q}$ form a system of parameters. By adapting Poonen's closed point sieve to sieve over higher dimensional varieties, we showed that, when $k<n$, the probability that randomly chosen $f_{0},f_{1},\ldots,f_{k}$ form a partial system of parameters is controlled by a zeta-function-like power series that enumerates higher dimensional varieties instead of closed points. In the following, $|Z|$ denotes the number of irreducible components of $Z$, and we write $\dim Z \equiv k$ if $Z$ is equidimensional of dimension $k$. 
%
%\begin{theorem}\cite{bruceErman-sop}*{Theorem~1.4}\label{thm:main finite field}
%Let $X\subseteq \P^r_{\fF_q}$ be a projective scheme of dimension $n$. Fix $e$ and let $k<n$. The probability that random polynomials $f_0,f_1,\dots,f_k$ of degree $d$ are parameters on $X$ is
%\[
%\Prob\left(\begin{matrix}f_0,f_1,\dots,f_{k} \text{ of degree $d$ } \\ \text{ are parameters on $X$}\end{matrix}\right) = 1 \ - 
%\sum_{\begin{smallmatrix}Z\subseteq X \text{reduced} \\ \dim Z \equiv n-k\\ \deg Z \leq e  \end{smallmatrix}}(-1)^{|Z|-1}q^{-(k+1)h^0(Z,\O_Z(d))}+ o\left(q^{-e(k+1)\binom{n-k+d}{n-k}}\right).
%\]
%\end{theorem}
%
%
%Notice that the power series on the right-hand side essentially enumerates subvarieties of dimension $n-k$. The main term determining the probability a randomly chosen set of homogenous polynomials forms a partial system of parameters is controlled by the number of $(n-k)$-planes contained in $X$. 
%
%\begin{cor}\cite{bruceErman-sop}\label{cor:error}
%Let $X\subseteq \P^r_{\fF_q}$ be a $n$-dimensional closed subscheme and let $k<n$.  Then
%\[
%\lim_{d\to \infty} \frac{\Prob\left(\begin{matrix}(f_0,\dots,f_{k}) \text{ of degree $d$} \\ \text{ are \underline{not} parameters on $X$}\end{matrix}\right)} {q^{-(k+1)\binom{n-k+d}{n-k}}} = \#\left\{\begin{matrix}\text{$(n-k)$-planes } L\subseteq \P^r_{\fF_q}\\\text{such that }  L\subseteq X\end{matrix}\right\}.
%\]
%\end{cor} 
%
%From this we proved the first explicit bound for Noether normalization over $\fF_{q}$ and gave a new proof of recent results on Noether normalizations of families over $\Z$ and $\fF_{q}[t]$ \cite{gabberLiuLorenzini15, cmbpt}.
%
%
%
%\subsection{Jacobians Covering Abelian Varieties}
%
%Over an infinite field, it is a classic result that every abelian variety is covered by a Jacobian variety of bounded dimension. Building upon work of Bucur and Kedlaya \cite{bucurKedlaya12}, Li and I proved an analogous result for abelian varieties over finite fields. We did so by first proving an effective version of Poonen's Bertini theorem over finite fields. 
%\begin{theorem}\cite{bruceLi19}*{Theorem~A}
%Fix $r,n\in \N$ with $n\geq2$, and let $\fF_{q}$ be a finite field of characteristic $p$. There exists an explicit constant $C_{r,q}$ such that if $A\subset \P^{r}_{\fF_q}$ is a non-degenerate abelian variety of dimension $n$, then for any $d\in \N$ satisfying 
%\[
%C_{r,q}\zeta_{A}\left(n+\tfrac{1}{2}\right) \deg(A) \leq  \frac{q^{\frac{d}{\max\{n+1,p\}}}d}{d^{n+1}+d^n+q^{d}},
%\]
%there exists a smooth curve over $\fF_{q}$ whose Jacobian $J$ maps surjectively onto $A$, where 
%\[
%\dim J\leq 
%\OO\left(\frac{ \deg(A)^2 d^{2(n-1)}}{r}\right).
%\OO\left( \deg(A)^2 d^{2(n-1)}r^{-1}\right).
%\]
%\end{theorem} 

%\subsection{Uniform Bertini}
%Notice that in the statement of Poonen's Bertini theorem, while the left-hand side of equation~\eqref{eq:poonen} is dependent of the embedding of $X$ into projective space (i.e. the choice of very ample line bundle), the overall limit is itself independent of the embedding of $X$. This suggests that there may be a more general and uniform statement of Poonen's Bertini theorem. One might hope that the analogous limit along any sequence $(L_{d})_{d\in\N}$ of line bundles growing in positivity equals $\zeta_{X}(n+1)^{-1}$. I am working with Isabel Vogt to formalize and prove such a theorem.
%
%Work of Erman and Wood on semi-ample Bertini theorems shows that a naive analogue of Theorem~\ref{thm:poonen} fails \cite{ermanWood15}. Vogt and I believe that this can be fixed by introducing an assumption on how the sequence of lines bundles grows in positivity. We say a sequence of line bundles  $\left(L_{d}\right)_{d\in\N}$ \textit{goes to infinity in all directions} if for every ample line bundle $A$ there exists $N\in \N$ such that $L_{i}-A$ is ample for all $i\geq N$. We are working to prove the following uniform version of Theorem~\ref{thm:poonen}.
%%%Notice that in the statement of Poonen's Bertini theorem, while the left-hand side of equation~\eqref{eq:poonen} is dependent of the embedding of $X$ into projective space (i.e. the choice of very ample line bundle), the overall limit is itself independent of the embedding of $X$. 
%%The probability that a random hypersurface of degree $d$ intersects $X\subset \P^r_{\fF_{q}}$ smoothly inherently depends on the embedding of $X$. However, Poonen's work shows that as $d\to\infty$ this is actually independent of the embedding. 
%%This suggests that there may be a more general and uniform statement of Poonen's Bertini theorem. That is one might hope that the analogous limit along any sequence $(L_{d})_{d\in\N}$ of line bundles growing in positivity equals $\zeta_{X}(n+1)^{-1}$.
%%
%%Work of Erman and Wood on semi-ample Bertini theorems shows that a naive analogue of Poonen's Bertini theorem fails \cite{ermanWood15}. Isabel Vogt and I believe that this can be fixed by introducing an assumption on how the sequence of lines bundles grows in positivity. We say a sequence of line bundles  $\left(L_{d}\right)_{d\in\N}$ \textit{goes to $\infty$ in all directions} if for every ample line bundle $A$ there exists $N\in \N$ such that $L_{i}-A$ is ample for all $i\geq N$. We are working to prove the following uniform Bertini theorem:
%% 
%\begin{conj}\label{gthm:effective-bertini}
%Let $X/\fF_{q}$ be a smooth projective variety of dimension $n$. If $\left(L_{d}\right)_{d\in\N}$ is a sequence of line bundles on $X$ going to infinity in all directions then 
%\begin{equation}
%\lim_{d\to \infty} \Prob\left(f \in H^0\left(X, L_{d}\right)  \;\;\; \bigg| \;\;\; 
%\begin{matrix}
% \text{$X\cap\V(f)$ is smooth}\\
% \text{of dimension $n-1$}
% \end{matrix}
%\right)=
%\zeta_X(n+1)^{-1}.
%\end{equation}
%\end{conj}
%
%We have verified this conjecture in a number of examples ($X=\P^1\times\P^1, \P^1\times\P^1\times\P^1$). We  are hopeful that similar methods will extend to whenever the nef cone of $X$ is a finitely generated.

%Inspired by Fujita's vanishing theorem and work of Erman and Wood, the following conjecture is another natural version of Poonen's Bertini theorem, which I also hope to work on with Vogt.  
%
%
% \begin{conj}\label{conj:bertini-fujita}
% Let $X/\fF_{q}$ be a smooth projective variety of dimension $n$ and $\cL$ be an ample line bundle on $X$. There exists an integer $m(\cL)$ such that for all $k\geq m(\cL)$ and all $\cD\in \Nef(X)$. 
%
%\begin{equation}
%\left\{
%f \in H^0\left(X, \cL(k)\otimes \cD\right) \quad \bigg| \quad 
%\begin{matrix}
% \text{$X\cap\V(f)$ is smooth}\\
% \text{of dimension $n-1$}
% \end{matrix}
%\right\}\neq\o.
%\end{equation}
% \end{conj}


\newpage 


%%%%%%%%%%%%%%%%%%%%%%%%%%%%%%%%%%%%%%%%%%%%%%%%%%%%%%%%%%%%%%%%%%%%%%%%%%%%%%%%%%%%%%%%%%%%%%%%%%%%%%%%%%

\begin{bibdiv}
\begin{biblist}[\normalsize]

\bib{almousaBruce19}{article}{
   author={Almousa, Ayah},
   author={Bruce, Juliette},
   author={Loper, Michael},
   author={Sayrafi, Mahrud},
   title={The virtual resolutions package for Macaulay2},
   journal={J. Softw. Algebra Geom.},
   volume={10},
   date={2020},
   number={1},
   pages={51--60},
}
   
\bib{aproduFarkas19}{article}{
   author={Aprodu, Marian},
   author={Farkas, Gavrill},
   author={Papadima, {\c{S}}tefan},
   author={Raicu, Claudiu},
   author={Weyman, Jerzy},
   title={Koszul modules and Green's conjecture},
   journal={Inventiones mathematicae},
   year={2019},
   month={Jun}
   day={15}
%   issn={1432-1297},
%   doi={10.1007/s00222-019-00894-1},
}

\bib{aproduFarkas11}{article}{
   author={Aprodu, Marian},
   author={Farkas, Gavril},
   title={Green's conjecture for curves on arbitrary $K3$ surfaces},
   journal={Compos. Math.},
   volume={147},
   date={2011},
   number={3},
   pages={839--851}
}


\bib{bayerEisenbud91}{article}{
   author={Bayer, Dave},
   author={Eisenbud, David},
   title={Graph curves},
   note={With an appendix by Sung Won Park},
   journal={Adv. Math.},
   volume={86},
   date={1991},
   number={1},
   pages={1--40},
%   issn={0001-8708},
%   review={\MR{1097026}},
%   doi={10.1016/0001-8708(91)90034-5},
}
%\bib{berkesch13}{article}{
%   author={Berkesch, Christine},
%   author={Erman, Daniel},
%   author={Kummini, Manoj},
%   author={Sam, Steven V.},
%   title={Tensor complexes: multilinear free resolutions constructed from
%   higher tensors},
%   journal={J. Eur. Math. Soc. (JEMS)},
%   volume={15},
%   date={2013},
%   number={6},
%   pages={2257--2295},
%%   issn={1435-9855},
%%   review={\MR{3120743}},
%%   doi={10.4171/JEMS/421},
%}
\bib{bertramEinLazarsfeld91}{article}{
  author={Bertram, Aaron},
  author={Ein, Lawrence},
  author={Lazarsfeld, Robert},
  title={Vanishing theorems, a theorem of {Severi}, and the equations defining projective varieties},
  journal={J. Amer. Math. Soc.},
  volume={4},
  date={1991},
  number={3},
  pages={587--602},
% issn={0894-0347},
% review={\MR{1092845}},
% doi={10.2307/2939270},
}

\bib{berkeschErmanSmith17}{article}{
   author={Berkesch, Christine},
   author={Erman, Daniel},
   author={Smith, Gregory G.},
   title={Virtual resolutions for a product of projective spaces},
   journal={Algebr. Geom.},
   volume={7},
   date={2020},
   number={4},
   pages={460--481},
   issn={2313-1691},
   review={\MR{4156411}},
   doi={10.14231/ag-2020-013},
}

\bib{BBCMMW22}{article}{
   author={Brandt, Madeline},
   author={Bruce, Juliette},
   author={Chan, Melody},
   author={Melo, Margarida},
   author={Moreland, Gwyneth},
   author={Wolfe, Corey},
   title={On the top-weight rational cohomology of $\mathcal{A}_g$},
   date={2022},
   journal={Geometry \& Topology},
   note={to appear},
}

\bib{brownErman22}{article}{
   author={Brown, Michael K.},
   author={Erman, Daniel},
   title={Tate resolutions on toric varietie},
   date={2022},
   note={Pre-print: \href{https://arxiv.org/abs/2108.03345}{arxiv:2108.03345}}
   }

\bib{brownErman23}{article}{
   author={Brown, Michael K.},
   author={Erman, Daniel},
   title={Linear syzygies of curves in weighted projective space},
   date={2023},
   note={Pre-print: \href{https://arxiv.org/abs/2301.09150}{arxiv:2301.0915}}
   }

\bib{brownErman23-2}{article}{
   author={Brown, Michael K.},
   author={Erman, Daniel},
   title={A short proof of the Hanlon-Hicks-Lazarev Theorem},
   date={2023},
   note={Pre-print: \href{https://arxiv.org/abs/2303.14319}{arxiv:2303.14319}}
   }
   
   
%\bib{bruceErman-sop}{article}{
%   author={Bruce, Juliette},
%   author={Erman, Daniel},
%   title={A probabilistic approach to systems of parameters and Noether
%   normalization},
%   journal={Algebra Number Theory},
%   volume={13},
%   date={2019},
%   number={9},
%   pages={2081--2102},
%}
%
%\bib{bruceLi19}{article}{
%   author={Bruce, Juliette},
%   author={Li, Wanlin},
%   title={Effective bounds on the dimensions of Jacobians covering abelian
%   varieties},
%   journal={Proc. Amer. Math. Soc.},
%   volume={148},
%   date={2020},
%   number={2},
%   pages={535--551}
%   }


\bib{bruceErmanGoldsteinYang18}{article}{
   author={Bruce, Juliette},
   author={Erman, Daniel},
   author={Goldstein, Steve},
   author={Yang, Jay},
   title={Conjectures and computations about Veronese syzygies},
   journal={Exp. Math.},
   volume={29},
   date={2020},
   number={4},
   pages={398--413},
}

\bib{bruceErman19}{article}{
   author={Bruce, Juliette},
   author={Erman, Daniel},
   author={Goldstein, Steve},
   author={Yang, Jay},
   title={The Schur-Veronese package in Macaulay2},
   journal={J. Softw. Algebra Geom.},
   volume={11},
   date={2021},
   number={1},
   pages={83--87},
}

\bib{bruce19-semiample}{article}{
   author={Bruce, Juliette},
   title={Asymptotic syzygies in the setting of semi-ample growth},
   date={2019},
   note={Pre-print: \href{https://arxiv.org/abs/1904.04944}{arxiv:1904.04944}}
   }


\bib{bruce19-hirzebruch}{article}{
   author={Bruce, Juliette},
   title={The quantitative behavior of asymptotic syzygies for Hirzebruch
   surfaces},
   journal={J. Commut. Algebra},
   volume={14},
   date={2022},
   number={1},
   pages={19--26},
}

%\bib{bruceNotices22}{article}{
%   author={Bruce, Juliette},
%   title={A word from... Juliette Bruce, Inaugural President of Spectra},
%   journal={Notices Amer. Math. Soc.},
%   volume={69},
%   date={2022},
%   number={6},
%   pages={898--899},
%}


\bib{BCEGLY22}{article}{
   author={Bruce, Juliette},
   author={Corey, Daniel},
   author={Erman, Daniel},
   author={Goldstein, Steve},
   author={Laudone, Robert P.},
   author={Yang, Jay},
   title={Syzygies of $\Bbb P^1\times\Bbb P^1$: data and conjectures},
   journal={J. Algebra},
   volume={593},
   date={2022},
   pages={589--621},
}
	

\bib{bruceHellerSayrafi21}{article}{
   author={Bruce, Juliette},
   author={Cranton Heller, Lauren},
   author={Sayrafi, Mahrud}
   title={Characterizing Multigraded Regularity on Products of Projective Spaces},
   date={2021},
   note={Pre-print: \href{https://arxiv.org/abs/2110.10705}{arxiv:2110.10705}}
   }
   
\bib{bruceHellerSayrafi22}{article}{
   author={Bruce, Juliette},
   author={Cranton Heller, Lauren},
   author={Sayrafi, Mahrud}
   title={Bounds on Multigraded Regularity},
   date={2022},
   note={Pre-print: \href{https://arxiv.org/abs/2208.11115}{arxiv:2208.11115}}
   }
   
   \bib{BB21}{article}{
   author={Buczy\'{n}ska, Weronika},
   author={Buczy\'{n}ski, Jaros\l aw},
   title={Apolarity, border rank, and multigraded Hilbert scheme},
   journal={Duke Math. J.},
   volume={170},
   date={2021},
   number={16},
   pages={3659--3702},
}

%\bib{bucurKedlaya12}{article}{
%   author={Bucur, Alina},
%   author={Kedlaya, Kiran S.},
%   title={The probability that a complete intersection is smooth},
%   language={English, with English and French summaries},
%   journal={J. Th\'eor. Nombres Bordeaux},
%   volume={24},
%   date={2012},
%   number={3},
%   pages={541--556},
%%   issn={1246-7405},
%%   review={\MR{3010628}},
%}


%\bib{charlesPoonen16}{article}{
%   author={Charles, Fran\c{c}ois},
%   author={Poonen, Bjorn},
%   title={Bertini irreducibility theorems over finite fields},
%   journal={J. Amer. Math. Soc.},
%   volume={29},
%   date={2016},
%   number={1},
%   pages={81--94},
%   issn={0894-0347},
%   review={\MR{3402695}},
%   doi={10.1090/S0894-0347-2014-00820-1},
%}

\bib{cartwrightErmanVelscoViray09}{article}{
   author={Cartwright, Dustin A.},
   author={Erman, Daniel},
   author={Velasco, Mauricio},
   author={Viray, Bianca},
   title={Hilbert schemes of 8 points},
   journal={Algebra Number Theory},
   volume={3},
   date={2009},
   number={7},
   pages={763--795},
}
	

\bib{CGP21}{article}{
   author={Chan, Melody},
   author={Galatius, S\o ren},
   author={Payne, Sam},
   title={Tropical curves, graph complexes, and top weight cohomology of
   $\mathcal{M}_g$},
   journal={J. Amer. Math. Soc.},
   volume={34},
   date={2021},
   number={2},
   pages={565--594}
}

\bib{chandler97}{article}{
  author={Chandler, Karen A.},
  title={Regularity of the powers of an ideal},
  journal={Comm. Algebra},
  volume={25},
  date={1997},
  number={12},
  pages={3773--3776},
% issn={0092-7872},
% review={\MR{1481564}},
% doi={10.1080/00927879708826084},
}

\bib{cox95}{article}{
   author={Cox, David A.},
   title={The homogeneous coordinate ring of a toric variety},
   journal={J. Algebraic Geom.},
   volume={4},
   date={1995},
   number={1},
   pages={17--50},
   issn={1056-3911},
   review={\MR{1299003}},
}


\bib{cutkoskyHerzogTrung99}{article}{
  author={Cutkosky, S. Dale},
  author={Herzog, J\"{u}rgen},
  author={Trung, Ng\^{o} Vi\^{e}t},
  title={Asymptotic behaviour of the Castelnuovo-Mumford regularity},
  journal={Compositio Mathematica},
  volume={118},
  date={1999},
  number={3},
  pages={243--261},
% issn={0010-437X},
% review={\MR{1711319}},
% doi={10.1023/A:1001559912258},
}


%\bib{cmbpt}{article}{
%   author={Chinburg, Ted},
%   author={Moret-Bailly, Laurent},
%   author={Pappas, Georgios},
%   author={Taylor, Martin J.},
%   title={Finite morphisms to projective space and capacity theory},
%   journal={J. Reine Angew. Math.},
%   volume={727},
%   date={2017},
%   pages={69--84},
%}

\bib{einLazarsfeld93}{article}{
   author={Ein, Lawrence},
   author={Lazarsfeld, Robert},
   title={Syzygies and Koszul cohomology of smooth projective varieties of
   arbitrary dimension},
   journal={Invent. Math.},
   volume={111},
   date={1993},
   number={1},
   pages={51--67},
%   issn={0020-9910},
%   review={\MR{1193597}},
%   doi={10.1007/BF01231279},
}
	
				
%\bib{einErmanLazarsfeld15}{article}{
%   author={Ein, Lawrence},
%   author={Erman, Daniel},
%   author={Lazarsfeld, Robert},
%   title={Asymptotics of random Betti tables},
%   journal={J. Reine Angew. Math.},
%   volume={702},
%   date={2015},
%   pages={55--75},
%   issn={0075-4102},
%   review={\MR{3341466}},
%   doi={10.1515/crelle-2013-0032},
%}

\bib{einLazarsfeld12}{article}{
   author={Ein, Lawrence},
   author={Lazarsfeld, Robert},
   title={Asymptotic syzygies of algebraic varieties},
   journal={Invent. Math.},
   volume={190},
   date={2012},
   number={3},
   pages={603--646},
%   issn={0020-9910},
%   review={\MR{2995182}},
%   doi={10.1007/s00222-012-0384-5},
}

\bib{EES15}{article}{
   author={Eisenbud, David},
   author={Erman, Daniel},
   author={Schreyer, Frank-Olaf},
   title={Tate resolutions for products of projective spaces},
   journal={Acta Math. Vietnam.},
   volume={40},
   date={2015},
   number={1},
   pages={5--36},
   issn={0251-4184},
   review={\MR{3331930}},
   doi={10.1007/s40306-015-0126-z},
}

 \bib{eisenbudGoto84}{article}{
   author={Eisenbud, David},
   author={Goto, Shiro},
   title={Linear free resolutions and minimal multiplicity},
   journal={J. Algebra},
   volume={88},
   date={1984},
   number={1},
   pages={89--133},
}

\bib{eisenbud05}{book}{
   author={Eisenbud, David},
   title={The geometry of syzygies},
   series={Graduate Texts in Mathematics},
   volume={229},
   note={A second course in commutative algebra and algebraic geometry},
   publisher={Springer-Verlag, New York},
   date={2005},
   pages={xvi+243},
%   isbn={0-387-22215-4},
%   review={\MR{2103875}},
}
%
%
%\bib{eisenbudSchreyer09}{article}{
%   author={Eisenbud, David},
%   author={Schreyer, Frank-Olaf},
%   title={Betti numbers of graded modules and cohomology of vector bundles},
%   journal={J. Amer. Math. Soc.},
%   volume={22},
%   date={2009},
%   number={3},
%   pages={859--888},
%%   issn={0894-0347},
%%   review={\MR{2505303}},
%%   doi={10.1090/S0894-0347-08-00620-6},
%}


\bib{ermanYang18}{article}{
   author={Erman, Daniel},
   author={Yang, Jay},
   title={Random flag complexes and asymptotic syzygies},
   journal={Algebra Number Theory},
   volume={12},
   date={2018},
   number={9},
   pages={2151--2166},
%   issn={1937-0652},
%   review={\MR{3894431}},
%   doi={10.2140/ant.2018.12.2151},
}

\bib{GVT15}{book}{
   author={Guardo, Elena},
   author={Van Tuyl, Adam},
   title={Arithmetically Cohen-Macaulay sets of points in $\Bbb P^{1} \times
   \Bbb P^{1}$},
   series={SpringerBriefs in Mathematics},
   publisher={Springer, Cham},
   date={2015},
   pages={viii+134},
   isbn={978-3-319-24164-7},
   isbn={978-3-319-24166-1},
   review={\MR{3443335}},
}

\bib{farkasPopa05}{article}{
   author={Farkas, Gavril},
   author={Popa, Mihnea},
   title={Effective divisors on $\overline{\mathcal{M}}_g$, curves on $K3$
   surfaces, and the slope conjecture},
   journal={J. Algebraic Geom.},
   volume={14},
   date={2005},
   number={2},
   pages={241--267},
%   issn={1056-3911},
%   review={\MR{2123229}},
%   doi={10.1090/S1056-3911-04-00392-3},
}

\bib{farkas06}{article}{
   author={Farkas, Gavril},
   title={Syzygies of curves and the effective cone of $\overline{
  \mathcal{M}}_g$},
   journal={Duke Math. J.},
   volume={135},
   date={2006},
   number={1},
   pages={53--98},
%   issn={0012-7094},
%   review={\MR{2259923}},
%   doi={10.1215/S0012-7094-06-13512-3},
}

\bib{farkasKemeny16}{article}{
   author={Farkas, Gavril},
   author={Kemeny, Michael},
   title={The generic Green-Lazarsfeld secant conjecture},
   journal={Invent. Math.},
   volume={203},
   date={2016},
   number={1},
   pages={265--301},
%   issn={0020-9910},
%   review={\MR{3437872}},
%   doi={10.1007/s00222-015-0595-7},
}

\bib{farkasKemeny17}{article}{
   author={Farkas, Gavril},
   author={Kemeny, Michael},
   title={The Prym-Green conjecture for torsion line bundles of high order},
   journal={Duke Math. J.},
   volume={166},
   date={2017},
   number={6},
   pages={1103--1124},
%   issn={0012-7094},
%   review={\MR{3635900}},
%   doi={10.1215/00127094-3792814},
}
	
\bib{gabberLiuLorenzini15}{article}{
   author={Gabber, Ofer},
   author={Liu, Qing},
   author={Lorenzini, Dino},
   title={Hypersurfaces in projective schemes and a moving lemma},
   journal={Duke Math. J.},
   volume={164},
   date={2015},
   number={7},
   pages={1187--1270},
%   issn={0012-7094},
%   review={\MR{3347315}},
%   doi={10.1215/00127094-2877293},
}

\bib{green84-I}{article}{
   author={Green, Mark L.},
   title={Koszul cohomology and the geometry of projective varieties},
   journal={J. Differential Geom.},
   volume={19},
   date={1984},
   number={1},
   pages={125--171},
%   issn={0022-040X},
%   review={\MR{739785}},
}

\bib{green84-II}{article}{
   author={Green, Mark L.},
   title={Koszul cohomology and the geometry of projective varieties. II},
   journal={J. Differential Geom.},
   volume={20},
   date={1984},
   number={1},
   pages={279--289},
%   issn={0022-040X},
%   review={\MR{772134}},
}


\bib{HHL23}{article}{
   author={Hanlon, Andrew},
   author={Hicks, Jeff},
   author={Lazarev, Oleg},
   title={Resolutions of toric subvarieties by line bundles and applications},
   date={2023},
   note={Pre-print: \href{https://arxiv.org/abs/2303.03763}{arxiv:2303.03763}}
   }
   
   
\bib{kodiyalam00}{article}{
  author={Kodiyalam, Vijay},
  title={Asymptotic behaviour of Castelnuovo-Mumford regularity},
  journal={Proc. Amer. Math. Soc.},
  volume={128},
  date={2000},
  number={2},
  pages={407--411},
% issn={0002-9939},
% review={\MR{1621961}},
% doi={10.1090/S0002-9939-99-05020-0},
}
\bib{kontsevich93}{article}{
   author={Kontsevich, Maxim},
   title={Formal (non)commutative symplectic geometry},
   conference={
      title={The Gelfand Mathematical Seminars, 1990--1992},
   },
   book={
      publisher={Birkh\"{a}user Boston, Boston, MA},
   },
   isbn={0-8176-3689-7},
   date={1993},
   pages={173--187}
}

\bib{kontsevich94}{article}{
   author={Kontsevich, Maxim},
   title={Feynman diagrams and low-dimensional topology},
   conference={
      title={First European Congress of Mathematics, Vol. II},
      address={Paris},
      date={1992},
   },
   book={
      series={Progr. Math.},
      volume={120},
      publisher={Birkh\"{a}user, Basel},
   },
   isbn={3-7643-2799-5},
   date={1994},
   pages={97--121}
   }

	
\bib{lazarsfeldPareschiPopa11}{article}{
   author={Lazarsfeld, Robert},
   author={Pareschi, Giuseppe},
   author={Popa, Mihnea},
   title={Local positivity, multiplier ideals, and syzygies of abelian
   varieties},
   journal={Algebra Number Theory},
   volume={5},
   date={2011},
   number={2},
   pages={185--196},
%   issn={1937-0652},
%   review={\MR{2833789}},
%   doi={10.2140/ant.2011.5.185},
}

\bib{lemmens18}{article}{
   author={Lemmens, Alexander},
   title={On the $n$-th row of the graded Betti table of an $n$-dimensional
   toric variety},
   journal={J. Algebraic Combin.},
   volume={47},
   date={2018},
   number={4},
   pages={561--584},
%   issn={0925-9899},
%   review={\MR{3813640}},
%   doi={10.1007/s10801-017-0786-y},
}


\bib{M2}{misc}{
    label={M2},
    author={Grayson, Daniel~R.},
    author={Stillman, Michael~E.},
    title = {Macaulay 2, a software system for research
	    in algebraic geometry},
    note = {Available at \url{http://www.math.uiuc.edu/Macaulay2/}},
}

\bib{maclaganSmith05}{article}{
   author={Maclagan, Diane},
   author={Smith, Gregory G.},
   title={Uniform bounds on multigraded regularity},
   journal={J. Algebraic Geom.},
   volume={14},
   date={2005},
   number={1},
   pages={137--164},
%   issn={1056-3911},
%   review={\MR{2092129}},
%   doi={10.1090/S1056-3911-04-00385-6},
}

\bib{maclaganSmith04}{article}{
   author={Maclagan, Diane},
   author={Smith, Gregory G.},
   title={Multigraded Castelnuovo-Mumford regularity},
   journal={J. Reine Angew. Math.},
   volume={571},
   date={2004},
   pages={179--212},
%   issn={0075-4102},
%   review={\MR{2070149}},
%   doi={10.1515/crll.2004.040},
}

\bib{mumford70}{article}{
   author={Mumford, David},
   title={Varieties defined by quadratic equations},
   conference={
      title={Questions on Algebraic Varieties},
      address={C.I.M.E., III Ciclo, Varenna},
      date={1969},
   },
   book={
      publisher={Edizioni Cremonese, Rome},
   },
   date={1970},
   pages={29--100},
%   review={\MR{0282975}},
}
	
\bib{mumford66}{article}{
   author={Mumford, D.},
   title={On the equations defining abelian varieties. I},
   journal={Invent. Math.},
   volume={1},
   date={1966},
   pages={287--354},
%   issn={0020-9910},
%   review={\MR{204427}},
%   doi={10.1007/BF01389737},
}
	
%	\bib{oeding17}{article}{
%   author={Oeding, Luke},
%   author={Raicu, Claudiu},
%   author={Sam, Steven V},
%   title={On the (non-)vanishing of syzygies of Segre embeddings},
%   journal={Algebraic Geometry},
%   volume={6},
%   number={5},
%   date={2019},
%   pages={571--591},
%}

\bib{ottavianiPaoletti01}{article}{
   author={Ottaviani, Giorgio},
   author={Paoletti, Raffaella},
   title={Syzygies of Veronese embeddings},
   journal={Compositio Math.},
   volume={125},
   date={2001},
   number={1},
   pages={31--37},
%   issn={0010-437X},
%   review={\MR{1818055}},
%   doi={10.1023/A:1002662809474},
}
	
\bib{pareschi00}{article}{
   author={Pareschi, Giuseppe},
   title={Syzygies of abelian varieties},
   journal={J. Amer. Math. Soc.},
   volume={13},
   date={2000},
   number={3},
   pages={651--664},
%   issn={0894-0347},
%   review={\MR{1758758}},
%   doi={10.1090/S0894-0347-00-00335-0},
}

\bib{pareschiPopa03}{article}{
   author={Pareschi, Giuseppe},
   author={Popa, Mihnea},
   title={Regularity on abelian varieties. I},
   journal={J. Amer. Math. Soc.},
   volume={16},
   date={2003},
   number={2},
   pages={285--302},
%   issn={0894-0347},
%   review={\MR{1949161}},
%   doi={10.1090/S0894-0347-02-00414-9},
}


\bib{pareschiPopa04}{article}{
   author={Pareschi, Giuseppe},
   author={Popa, Mihnea},
   title={Regularity on abelian varieties. II. Basic results on linear
   series and defining equations},
   journal={J. Algebraic Geom.},
   volume={13},
   date={2004},
   number={1},
   pages={167--193},
%   issn={1056-3911},
%   review={\MR{2008719}},
%   doi={10.1090/S1056-3911-03-00345-X},
}

\bib{schreyer86}{article}{
   author={Schreyer, Frank-Olaf},
   title={Syzygies of canonical curves and special linear series},
   journal={Math. Ann.},
   volume={275},
   date={1986},
   number={1},
   pages={105--137},
%   issn={0025-5831},
%   review={\MR{849058}},
%   doi={10.1007/BF01458587},
}	
	
\bib{voisin02}{article}{
   author={Voisin, Claire},
   title={Green's generic syzygy conjecture for curves of even genus lying
   on a $K3$ surface},
   journal={J. Eur. Math. Soc. (JEMS)},
   volume={4},
   date={2002},
   number={4},
   pages={363--404},
%   issn={1435-9855},
%   review={\MR{1941089}},
%   doi={10.1007/s100970200042},
}

\bib{voisin05}{article}{
   author={Voisin, Claire},
   title={Green's canonical syzygy conjecture for generic curves of odd
   genus},
   journal={Compos. Math.},
   volume={141},
   date={2005},
   number={5},
   pages={1163--1190},
%   issn={0010-437X},
%   review={\MR{2157134}},
%   doi={10.1112/S0010437X05001387},
}

\bib{willwacher15}{article}{
   author={Willwacher, Thomas},
   title={M. Kontsevich's graph complex and the Grothendieck-Teichm\"{u}ller
   Lie algebra},
   journal={Invent. Math.},
   volume={200},
   date={2015},
   number={3},
   pages={671--760},
   issn={0020-9910},
   review={\MR{3348138}},
   doi={10.1007/s00222-014-0528-x},
}
	
\end{biblist}
\end{bibdiv}
\end{document}