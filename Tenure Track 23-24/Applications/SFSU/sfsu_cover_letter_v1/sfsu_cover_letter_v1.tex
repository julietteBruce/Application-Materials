\documentclass[11pt]{article}
\usepackage{amsmath}
\usepackage{amsfonts}
\usepackage{graphicx}
\usepackage{color}
\usepackage[hidelinks]{hyperref}
\usepackage{fullpage}
\setlength{\topmargin}{-1.0in}
\setlength{\footskip}{0.5in}
\setlength{\textwidth}{6.5in}
\setlength{\textheight}{10.0in}
\setlength{\oddsidemargin}{-0.0in}
\setlength{\evensidemargin}{-0.0in}
\usepackage{parskip}
\usepackage{fancyhdr}
\pagestyle{fancy}
\addtolength{\headheight}{\baselineskip}
\addtolength{\headheight}{1in}
\addtolength{\textheight}{-\baselineskip}
\addtolength{\textheight}{-1in}
\lhead{}
\chead{\includegraphics[height=1in]{brownlogo.eps}}
\rhead{}
\lfoot{}
\cfoot{{\footnotesize
       Department of Mathematics \\
       Brown University \hspace{0.1in} 151 Thayer Street \hspace{0.1in} Providence, RI 02912 \\
       Phone: (401) 863-2708 \hspace{0.25in}
       Fax: (401) 863-9013 \hspace{0.25in}
       Web: \url{https://www.brown.edu/math/}
       }}
\rfoot{}
\renewcommand{\headrulewidth}{0pt}

\usepackage{enumitem}

\begin{document}

\section*{}

\noindent
\begin{minipage}{0.99\textwidth}
\begin{minipage}{0.69\textwidth}
\textcolor{white}{.}
\end{minipage}
\begin{minipage}{0.29\textwidth}
{
Juliette Bruce \\
Postdoctoral Researcher \\
Department of Mathematics \\
\href{mailto:juliette\_bruce1@brown.edu}{juliette\_bruce1@brown.edu}
%\url{juliettebruce.github.io}
%Phone: (810)--623--7610 
}

\vspace{12pt}
\today
\end{minipage}
\end{minipage}

%\vspace{12pt}
%\noindent
%John Doe \\
%Business, Inc. \\
%1234 Address Avenue \\
%City, State ZZZIP

\vspace{12pt}
\noindent
Dear Committee Members,

I am writing to apply for the tenure-track assistant professor position in the Department of Mathematics at San Francisco State University. Currently, I am a postdoctoral researcher in the Mathematics Department at Brown University and a lecturer in the Department of Mathematics at San Francisco State University. I received my Ph.D. in Mathematics from the University of Wisconsin-Madison under the guidance of my advisor Professor Daniel Erman in 2020. From 2020-2022 I was an NSF Postdoctoral Fellow in the Mathematics Department at the University of California, Berkeley. I was a postdoctoral fellow at the Mathematical Sciences Research Institute for 2020-2021.

I am especially interested in this position because during the Fall of 2023, while working remotely at Brown University, I had the opportunity to teach at San Francisco State University.  In this role, I saw the amazing positive impact San Francisco State University has on helping a wide range of students, especially those from underrepresented and underserved groups, achieve their dreams. This proved to be one of the most rewarding teaching experiences I have ever had. 

I feel like in this position I could achieve my career goals of continuing a thriving research program in algebra, promoting an inclusive and supportive community both at the departmental and university level, as well as in the wider mathematical community, and supporting students from broad and diverse social and cultural backgrounds. Towards this last point, I am particularly excited that this position is specifically focused on Black and Latinx/e student success. Further, I love how energetic the Department of Mathematics is and how enthusiastic and thoughtful all the students are. I would love the opportunity to contribute to such a department, build collaborations and connections with colleagues in the department and beyond, and work with and mentor students, especially on thesis projects. 


The Department of Mathematics's commitment to just, equitable, and inclusive education aligns extremely well with my passion for making mathematics a more inclusive community and supporting students from underrepresented communities. For example, I have worked to make mathematics more inclusive of people from underrepresented groups by founding events like \textit{Trans Math Day} and leading \textit{Spectra: the Association for LGBTQ+ Mathematicians}. Going forward, I am excited to continue working hard to promote these values through my research, teaching, and service. I am committed to continually trying to refine and advance my teaching practices, by seeking out new ways to better center the needs of students. For example, would love to organize initiatives and programs aimed at supporting LGBTQ+ students, students of color, and women in mathematics. In the long term, I would love to organize a summer research program, similar to MSRI-UP, which promotes and supports LGBTQ+ undergraduate and graduate students by providing them with a supportive introduction to mathematics research and helping them transition to graduate school. Given the San Francisco State University's commitment to diversity, social justice, and supporting students from diverse socio-economic and cultural backgrounds and the long LGBTQ+ history in San Francisco I feel like there would be few places better situated to host such a program. 

Two personal notes for my interest in the position: (i) currently my spouse (who works at a college in the Bay Area) and I are attempting to solve the two-body problem and would love to stay in San Francisco, (ii) I find San Francisco State University's long and storied commitment to activism -- for example the Third World Liberation Front strikes -- extremely moving. 

My research interests lie in the intersection of commutative algebra and algebraic geometry with connections to computational algebra, number theory, and recently topology. I am interested in using homological, combinatorial, and computational methods to study the geometry of algebraic varieties. Currently, my research program has two broad directions.
\begin{enumerate}[leftmargin=*,label=(\roman*)]
\item I have sought to deepen and expand our understanding of the ways homological algebra can be used to study the geometry of toric varieties. This seeks to generalize a very classical story using homological algebra to understand subvarieties of projective space.
\item I have been studying the geometry and topology of various moduli spaces, e.g., the moduli space of (principally polarized) abelian varieties of a fixed dimension, via combinatorially and homological methods. This has led to novel applications to arithmetic groups. 
\end{enumerate}
Further, I am passionate about promoting inclusivity, diversity, and justice in the mathematics community. This passion extends throughout my teaching where I am dedicated to creating an interactive and supportive classroom environment that helps students thrive. 

My research output includes 15 papers, with publications in journals such as \textit{Algebra \& Number Theory}, \textit{Geometry \& Topology}, and \textit{Experimental Mathematics}, as well as, multiple published software packages. Below are a few of the non-research highlights of my file.

\begin{itemize}[leftmargin=*]
\item I was awarded an \textit{NSF Postdoctoral Research Fellowship}, an \textit{NSF Graduate Research Fellowship}, and I have secured over \$100,000 in conference grants including 4 NSF conference grants. 
\item I have organized 12+ conferences, workshops, and special sessions, including multiple events aimed at supporting and promoting mathematicians from generally underrepresented groups, especially women and LGBTQ+ mathematicians. 
\item I was awarded the highest departmental and campus-wide teaching awards at the University of Wisconsin - Madison, the Capstone Teaching Award (2019) and the Teaching Assistant Award for Exceptional Service (2018), awarded to 1 and 3 students each year respectively. 
\item I served as the inaugural president of \textit{Spectra: The Association for LGBTQ+ Mathematicians}.
\item I organized two summer undergraduate research programs, and participated in one summer research program for students starting graduate school. One of the undergraduates I mentored was awarded an \textit{NSF Graduate Research Fellowship} to student mathematics.  
\end{itemize}

Discussing my qualifications and interest in this specific position I want to acknowledge that as a white trans woman, I have not faced the effects of historical and present-day structural racism, and do not know from first-hand experience the challenges faced by Black and Latinx/e students. The challenges and barriers I have faced have been uniquely my own, but they have made I am extremely passionate about supporting students from underrepresented groups and making mathematics more inclusive. In all these efforts I have attempted to both center the needs and desires of Black and Latinx/e people, as well as consider the way intersecting identities affect my teaching, research, and service. For example, I have done significant work supporting LGBTQ+ students, and when supporting LGBTQ+ students of color (including some who have been at San Francisco State University) it is important to recognize that these students cannot necessarily separate their LGBTQ+ and racial identities. I feel like I meet the following Black and Latinx/e Student Success Cohort Hire criteria:

1/2: As a graduate student out of a desire to better understand the needs and challenges faced by students from underrepresented groups, and how I can better support them, I took a semester-long course \textit{Inclusive Practices in the College Classroom} through the \textit{Delta Program for Integrating Research, Teaching and Learning}. This course took a wide view of ways to create an inclusive classroom from both understanding the historical and systematic barriers students, especially Black and Latinx/e students, may face, to understanding various pedagogical approaches to help minimize the impact of these systems in the classroom and support students. While at the University of California, Berkeley I actively participated in a reading/working group exploring antiracist and anti-oppressive pedagogy in the mathematics classroom. Further, I personally sought to engage with ways to humanize mathematics and support underrepresented students by exploring the works of Pamela E. Harris, Aris Winger, Rochelle Guti\'{e}rrez, Luis Leyva, and Francis Su. While teaching at San Francisco State University I participated in the Metro College Success Program and the Lecturer Faculty Teaching and Learning Community through the Center for Equity and Excellence in Teaching and Learning (CEETL). Both of these programs focused in part on understanding the ways structural racism and other oppressive systems affect students, and how we as instructors can better serve students and help them overcome challenges they may face. 

3: I have attempted to take the knowledge that I discussed in the point above and translate it into action in the classroom by implementing anti-racist pedagogical practices. In particular, beyond simply creating an inclusive learning environment I have also found it important to create a space where students feel comfortable bringing their whole selves, including all of their experiences, backgrounds, challenges, identities, struggles, and knowledge. For example, I recognize that all students, like all people, will have days when negative experiences outside the classroom affect their ability to engage in the classroom. This is even more true for students who face racism, sexism, homo/transphobia, and other systems of oppression. On such a day when students enter the classroom, I look to try to meet the students where they are. For example, sometimes this means I will walk the student to the campus mental health or cultural center, or sometimes it means I create new problems specifically to help keep the student's mind off of whatever is troubling them. I try to make sure my students know I am there to provide them with whatever resources they need to succeed both in the classroom and in their life beyond. However, this human-centered approach also leads to many beautiful moments. For example, by allowing students to bring all of themselves to class they experience mathematics in a humanistic way, seeing how mathematics and quantitative thinking are an integral aspect of their life. I have found this often increases students' motivation, as well as opens themselves up to making mistakes, growing, and learning. 

7: I have organized 10+ national/international conferences including \textit{M2@UW} (45 participants), \textit{Geometry and Arithmetic of Surfaces} (40 participants), \textit{Graduate Workshop in Commutative Algebra for Women \& Mathematicians of Minority Genders} (35 participants), \textit{CAZoom} (70 participants), \textit{Western Algebraic Geometry Symposium} (100 participants), \textit{GEMS of Combinatorics} (40 participants), $\text{Spec}(\overline{\mathbb{Q}})$ (50 participants), \textit{BATMOBILE} (30 participants),  \textit{GEMS of Combinatorics II} (30 participants), and \textit{GEMS of Commutative Algebra} (40 participants). Additionally, I organized three special sessions at AMS Sectional Meetings and the Joint Math Meetings. Many of these conferences were specifically aimed a supporting mathematicians from underrepresented groups. For example, the ``GEMS'' workshops sought to build a diverse community of mathematicians to address gender equity in the mathematical community from new perspectives. In particular, these workshops sought to recognize that in order to address gender equity we must consider the ways gender intersects with other identities like race and center the experiences of Black and Latinx/e people. Going forward I am interested in expanding these ``GEMS'' workshops to other areas of mathematics and creating cross-field discussions that broaden the standard notion of gender equity in mathematics. Further, when organizing these conferences and other events I have actively sought to promote the work and scholarship of Black and Latinx/e mathematicians. For example, when organizing Spectra's Lavender Lecture at the 2022 Joint Math Meetings I invited Luis Leyva to speak on his work studying the ways to disrupt racism and cisheteropatriarchy in the classroom. 

8: When doing service and outreach work I frequently attempt to find ways to help programs better serve underrepresented communities, especially Black and Latinx/e communities. As an example, the Madison Math Circle (MMC) is an outreach program sponsored by the UW-Madison Math Department. Its goal is to kindle excitement and appreciation of math in middle and high school students. In the Fall 2014, I began volunteering with the MMC. At the time, the circle's main programming was a weekly on-campus lecture given by a member of the math department. For a number of reasons, including Madison's long history of redlining and segregation, few Black and Latinx/e students attended these meetings. After a year of volunteering, I stepped into the role of organizer. During my three years as an organizer, I worked to build stronger connections between the MMC, local schools, and other outreach organizations with the goal of reaching students from underrepresented groups. For example, with other organizers, we made connections with Centro Hispano to better reach and serve Latinx/e students. These ties helped the weekly attendance more than double, and grow substantially more diverse. I also led the creation of a new outreach arm of the MMC, which visits high schools around the state of Wisconsin to better serve students from underrepresented groups. For example, I may make it a point to establish strong connections with the high schools in Madison with the largest Black and Latinx/e population and make multiple visits per semester. This program dramatically expanded the reach of the circle, and during my final year as an organizer, the MMC reached over 300 students including a larger number of Black and Latinx/e students.

With my application, I include a curriculum vitae, the AMS cover sheet, a research statement, a teaching statement, a diversity statement, a publication list, and a list of external funding. I will have five letters of recommendation. Four research letters: Christine Berkesch (\href{mailto:cberkesc@umn.edu}{cberkesc@umn.edu}), Melody Chan (\href{mailto:melody\_chan@brown.edu}{melody\_chan@brown.edu}), Daniel Erman (\href{mailto:erman@hawaii.edu}{erman@hawaii.edu}), and Gregory G. Smith (\href{mailto:ggsmith@mast.queensu.ca}{ggsmith@mast.queensu.ca}), and one teaching letter from Shirin Malekpour (\href{mailto:shirin.malekpour@wisc.edu}{shirin.malekpour@wisc.edu}).  

Please do not hesitate to contact me if there is anything else I can provide, or with any questions, and thank you in advance for your consideration. 

\vspace{24pt}
\noindent
\begin{minipage}{0.99\textwidth}
\begin{minipage}{0.69\textwidth}
\textcolor{white}{.}
\end{minipage}
\begin{minipage}{0.29\textwidth}
Sincerely, 

\vspace{36pt}
Juliette Bruce\\
Postdoctoral Research Associate\end{minipage}
\end{minipage}

% This command changes the page style to plain from this page onward.
% If your letter is 1 page long, then comment this out.
% If your letter is 2 pages long, then include this command.
% If your letter is longer than 2 pages, then you may need to place
% this command earlier in the document.
%\pagestyle{plain}

\end{document}