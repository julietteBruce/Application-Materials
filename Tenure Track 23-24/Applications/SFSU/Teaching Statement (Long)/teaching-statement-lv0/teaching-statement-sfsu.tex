\documentclass[11pt,reqno]{amsart}
\usepackage{amsfonts,amsmath,amssymb,amsbsy,amstext,amsthm,mathtools}
\usepackage{accents,color,enumerate,enumitem,float,fullpage,verbatim}

\usepackage{url}
\usepackage[colorlinks=true,hyperindex, linkcolor=magenta, pagebackref=false, citecolor=cyan]{hyperref}
\usepackage[alphabetic,lite]{amsrefs} 

%\usepackage{eucal,bm,kpfonts,mathbbol}

%\usepackage{parskip}
\usepackage{tikz,tikz-cd}	
\usetikzlibrary{positioning, matrix, shapes}         								    				
\usetikzlibrary{arrows,calc,matrix}

\usepackage{lscape}

\usepackage[letterpaper, margin=.62in]{geometry}


\usepackage{microtype}

%\usepackage{parskip}

\usepackage{titlesec}		
\setcounter{secnumdepth}{4}						     					% Allows one to use nice section titles
\titleformat{\section}[runin]{\scshape\bfseries}{\thesection.}{1em}{}		% Creates section titles
\titleformat{\subsection}[runin]{\scshape\bfseries}{\thesubsection}{1em}{}			% Creates subsection titles
\titleformat{\subsubsection}[runin]{\scshape\bfseries}{\thesubsubsection}{1em}{}			% Creates subsection titles

\usepackage[titles]{tocloft}								     					% Creates table of fancy contents
\setcounter{tocdepth}{4}
\renewcommand{\contentsname}{}	     					% Renames and centers title of ToC

\usepackage{multirow}
\usepackage{array}
\usepackage{booktabs}
\newcolumntype{M}[1]{>{\centering\arraybackslash}m{#1}}
\newcolumntype{N}{@{}m{0pt}@{}}
\usepackage{diagbox}
\usepackage{cancel}

\newtheorem{lemma}{Lemma}[section]
\newtheorem{theorem}[lemma]{Theorem}
\newtheorem{goalTheorem}[lemma]{Goal Theorem}
\newtheorem{prop}[lemma]{Proposition}
\newtheorem{cor}[lemma]{Corollary}
\newtheorem{conj}[lemma]{Conjecture}
\newtheorem{claim}[lemma]{Claim}
\newtheorem{defn}[lemma]{Definition} 
\newtheorem{notation}[lemma]{Notation} 
\newtheorem{exercise}[lemma]{Exercise}
\newtheorem{question}[lemma]{Question}
\newtheorem*{assumption}{Assumption}
\newtheorem{principle}[lemma]{Principle}
\newtheorem{heuristic}[lemma]{Heuristic}

\newtheorem{theoremalpha}{Theorem}
\newtheorem{corollaryalpha}[theoremalpha]{Corollary}
\renewcommand{\thetheoremalpha}{\Alph{theoremalpha}}

\theoremstyle{remark}
\newtheorem{remark}[lemma]{Remark}
\newtheorem{example}[lemma]{Example}
\newtheorem{cexample}[lemma]{Counterexample}

% Commands
\newcommand{\initial}{\operatorname{in}}
\newcommand{\NF}{\operatorname{NF}}
\newcommand{\HF}{\operatorname{HF}}
\newcommand{\Hilb}{\operatorname{Hilb}}
\newcommand{\depth}{\operatorname{depth}}
\newcommand{\reg}{\operatorname{reg}}
\newcommand{\Span}{\operatorname{span}}
\newcommand{\img}{\operatorname{img}}
\newcommand{\inn}{\operatorname{in}}

\newcommand{\length}{\operatorname{length}}
\newcommand{\coker}{\operatorname{coker}}
\newcommand{\adeg}{\operatorname{adeg}}
\newcommand{\pdim}{\operatorname{pdim}}
\newcommand{\Spec}{\operatorname{Spec}}
\newcommand{\Ext}{\operatorname{Ext}}
\newcommand{\Tor}{\operatorname{Tor}}
\newcommand{\LT}{\operatorname{LT}}
\newcommand{\im}{\operatorname{im}}
\newcommand{\NS}{\operatorname{NS}}
\newcommand{\Frac}{\operatorname{Frac}}
\newcommand{\Khar}{\operatorname{char}}
\newcommand{\Proj}{\operatorname{Proj}}
\newcommand{\id}{\operatorname{id}}
\newcommand{\Div}{\operatorname{Div}}
\newcommand{\Kl}{\operatorname{Cl}}
\newcommand{\tr}{\operatorname{tr}}
\newcommand{\Tr}{\operatorname{Tr}}
\newcommand{\Supp}{\operatorname{Supp}}
\newcommand{\ann}{\operatorname{ann}}
\newcommand{\Gal}{\operatorname{Gal}}
\newcommand{\Pic}{\operatorname{Pic}}
\newcommand{\QQbar}{{\overline{\mathbb Q}}}
\newcommand{\Br}{\operatorname{Br}}
\newcommand{\Bl}{\operatorname{Bl}}
\newcommand{\Kox}{\operatorname{Cox}}
\newcommand{\conv}{\operatorname{conv}}
\newcommand{\getsr}{\operatorname{Tor}}
\newcommand{\diam}{\operatorname{diam}}
\newcommand{\Hom}{\operatorname{Hom}} %done
\newcommand{\sheafHom}{\mathcal{H}om}
\newcommand{\Gr}{\operatorname{Gr}}
\newcommand{\rank}{\operatorname{rank}} 
\newcommand{\codim}{\operatorname{codim}}
\newcommand{\Sym}{\operatorname{Sym}} %done
\newcommand{\GL}{{GL}}
\newcommand{\Prob}{\operatorname{Prob}}
\newcommand{\Density}{\operatorname{Density}}
\newcommand{\Syz}{\operatorname{Syz}}
\newcommand{\pd}{\operatorname{pd}}
\newcommand{\supp}{\operatorname{supp}}
\newcommand{\cone}{\operatorname{\textbf{cone}}}
\newcommand{\Res}{\operatorname{Res}}
\newcommand{\HS}{\operatorname{HS}}
\newcommand{\Cl}{\operatorname{Cl}}
\newcommand{\oO}{\operatorname{O}}

\newcommand{\defi}[1]{\textsf{#1}} % for defined terms

\newcommand{\remd}{\operatorname{remd}}
\newcommand{\colim}{\operatorname{colim}}
\newcommand{\trideg}{\operatorname{tri.deg}}
\newcommand{\indeg}{\operatorname{index.deg}}
\newcommand{\moddeg}{\operatorname{mod.deg}}
\newcommand{\Desc}{\operatorname{Desc}}
\newcommand{\inter}{\operatorname{int}}
\newcommand{\Nef}{\operatorname{Nef}}
\newcommand{\Jac}{\operatorname{Jac}}
\newcommand{\Cox}{\operatorname{Cox}}

\newcommand{\doot}{\bullet}

\newcommand{\Alt}{\bigwedge\nolimits}
\newcommand{\Set}{\text{\bf Set}}										% Category of Sets
\newcommand{\Sch}{\text{\bf Sch}}										% Category of Abelian Groups
\newcommand{\Mod}[1]{\ (\mathrm{mod}\ #1)}




%%%%%%%%%%%%%%%%%%%%%%%%%%%%%% Letters  %%%%%%%%%%%%%%%%%%%%%%%%%%%%%%%%%%%%%%%%%%%%
%%%%%%%%%%%%%%%%%%%%%%%%%%%%%%%%%%%%%%%%%%%%%%%%%%%%%%%%%%%%%%%%%%%%%%%%%%%%%%
\newcommand{\ff}{\mathbf f}
\newcommand{\kk}{\mathbf k}
\renewcommand{\aa}{\mathbf a}
\newcommand{\bb}{\mathbf b}
\newcommand{\cc}{\mathbf c}
\newcommand{\dd}{\mathbf d}
\newcommand{\ee}{\mathbf e}
\newcommand{\vv}{\mathbf v}
\newcommand{\ww}{\mathbf w}
\newcommand{\xx}{\mathbf x}
\newcommand{\yy}{\mathbf y}
\newcommand{\rr}{\mathbf r}
\newcommand{\ii}{\mathbf i}
\newcommand{\nn}{\mathbf n}
\newcommand{\pp}{\mathbf p}
\newcommand{\mm}{\mathbf m}
\newcommand{\fF}{\mathbf F}
\newcommand{\gG}{\mathbf G}
\newcommand{\eE}{\mathbf E}
\newcommand{\qQ}{\mathbf Q}
\newcommand{\tT}{\mathbf T}
\renewcommand{\tt}{\mathbf t}
\newcommand{\one}{\mathbf 1}
\newcommand{\zero}{\mathbf 0}

\renewcommand{\H}{\operatorname{H}}
\newcommand{\OO}{\operatorname{O}}
\newcommand{\oo}{\operatorname{o}}


%%%% Caligraphic Fonts - i.e. ????. %%%%%
\newcommand{\cA}{\mathcal{A}}
\newcommand{\cB}{\mathcal{B}}
\newcommand{\cC}{\mathcal{C}}
\newcommand{\cD}{\mathcal{D}}
\newcommand{\cE}{\mathcal{E}}
\newcommand{\cF}{\mathcal{F}}
\newcommand{\cG}{\mathcal{G}}
\newcommand{\cH}{\mathcal{H}} 
\newcommand{\cI}{\mathcal{I}}
\newcommand{\cJ}{\mathcal{J}}
\newcommand{\cK}{\mathcal{K}}
\newcommand{\cL}{\mathcal{L}}
\newcommand{\cM}{\mathcal{M}}
\newcommand{\cN}{\mathcal{N}}
\renewcommand{\O}{\mathcal{O}}
\newcommand{\cP}{\mathcal{P}}
\newcommand{\cQ}{\mathcal{Q}}
\newcommand{\cR}{\mathcal{R}}
\newcommand{\cS}{\mathcal{S}}
\newcommand{\cT}{\mathcal{T}}
\newcommand{\U}{\mathcal{U}} 		% Notice this is different
\newcommand{\cV}{\mathcal{V}}
\newcommand{\cW}{\mathcal{W}}
\newcommand{\cX}{\mathcal{X}}
\newcommand{\cY}{\mathcal{Y}}
\newcommand{\cZ}{\mathcal{Z}}

%%%% Blackboard Fonts - i.e. Real Numbers, Integers, etc. %%%%%
\newcommand{\A}{\mathbb{A}}
\newcommand{\B}{\mathbb{B}}
\newcommand{\C}{\mathbb{C}}
\newcommand{\D}{\mathbb{D}}
\newcommand{\E}{\mathbb{E}}
\newcommand{\F}{\mathbb{F}}
\newcommand{\G}{\mathbb{G}}
\newcommand{\I}{\mathbb{I}}
\newcommand{\J}{\mathbb{J}}
\newcommand{\K}{\mathbb{K}}
\renewcommand{\L}{\mathbb{L}}
\newcommand{\M}{\mathbb{M}}
\newcommand{\N}{\mathbb{N}}
\newcommand{\bO}{\mathbb{O}}		% Notice this is \bO
\renewcommand{\P}{\mathbb{P}}
\newcommand{\Q}{\mathbb{Q}}
\newcommand{\R}{\mathbb{R}}
\newcommand{\T}{\mathbb{T}}
\newcommand{\bU}{\mathbb{U}}		% Notice this is \bU
\newcommand{\V}{\mathbb{V}}
\newcommand{\W}{\mathbb{W}}
\newcommand{\X}{\mathbb{X}}
\newcommand{\Y}{\mathbb{Y}}
\newcommand{\Z}{\mathbb{Z}}

 %%%% Sarif Fonts - i.e. ???? %%%%%
\newcommand{\sA}{\mathsf{A}}
\newcommand{\sB}{\mathsf{B}}
\newcommand{\sC}{\mathsf{C}}
\newcommand{\sD}{\mathsf{D}}
\newcommand{\sE}{\mathsf{E}}
\newcommand{\sF}{\mathsf{F}}
\newcommand{\sG}{\mathsf{G}}
\newcommand{\sH}{\mathsf{H}} 
\newcommand{\sI}{\mathsf{I}}
\newcommand{\sJ}{\mathsf{J}}
\newcommand{\sK}{\mathsf{K}}
\newcommand{\sL}{\mathsf{L}}
\newcommand{\sM}{\mathsf{M}}
\newcommand{\sN}{\mathsf{N}}
\newcommand{\sO}{\mathsf{O}}
\newcommand{\sP}{\mathsf{P}}
\newcommand{\sQ}{\mathsf{Q}}
\newcommand{\sR}{\mathsf{R}}
\newcommand{\sS}{\mathsf{S}}
\newcommand{\sT}{\mathsf{T}}
\newcommand{\sU}{\mathsf{U}} 
\newcommand{\sV}{\mathsf{V}}
\newcommand{\sW}{\mathsf{W}}
\newcommand{\sX}{\mathsf{X}}
\newcommand{\sY}{\mathsf{Y}}
\newcommand{\sZ}{\mathsf{Z}}
 
 %%%% Fraktur Fonts - i.e. maximal ideals, prime ideals, etc. %%%%%
\newcommand{\cl}{\mathfrak{cl}}
\newcommand{\g}{\mathfrak{g}}
\newcommand{\h}{\mathfrak{h}}
\newcommand{\m}{\mathfrak{m}}
\newcommand{\n}{\mathfrak{n}}
\newcommand{\p}{\mathfrak{p}}
\newcommand{\q}{\mathfrak{q}}
\renewcommand{\r}{\mathfrak{r}}



\newcommand{\juliette}[1]{{\color{red} \sf $\spadesuit\spadesuit\spadesuit$ Juliette: [#1]}}


\title{Juliette Bruce's Research Statement}

%\author{Juliette Bruce}
%\address{Department of Mathematics, University of Wisconsin, Madison, WI}
%\email{\href{mailto:juliette.bruce@math.wisc.edu}{juliette.bruce@math.wisc.edu}}
%\urladdr{\url{http://math.wisc.edu/~juliettebruce/}}

%\thanks{The author was partially supported by the NSF GRFP under Grant No. DGE-1256259 and NSF grant DMS-1502553.}

%\subjclass[2010]{13D02, 14M25}

\begin{document} 

%\maketitle
\begingroup  
  \centering
  \large\scshape\bfseries Juliette Bruce's Teaching Statement\\[1em]
\endgroup

%\tableofcontents

\setcounter{section}{0}

\noindent \textbf{I. Introduction.} My goal as an educator is to be an active guide for students, providing them with environments where they feel supported and encouraged to let their own mathematical and quantitative curiosities guide how they engage and learn. By taking this approach, I hope to engage with students as the complete people that they are, asking them to bring all of their experiences, backgrounds, identities, and knowledge into the learning environment. I want students to experience mathematics in a humanistic way, seeing how mathematics and quantitative thinking are integral aspects of their lives.  As one of my former students noted, ``Juliette obviously wants us to succeed not only in math but in life.'' Recognizing that learning mathematics is not necessarily confined to the classroom I have sought out new and non-traditional teaching opportunities. My teaching has been recognized through both awards and student evaluations:
\begin{itemize}[leftmargin=*]
\item In 2018, I was one of three graduate students recognized campus-wide with the Teaching Assistant Award for Exceptional Service.
\item I received two TA awards from the math department, the TA Service Award (2018) and the Capstone Teaching Award (2019), the latter of which is awarded to just one teaching assistant each year, for an exceptional record of teaching excellence and service.
\item My student evaluations are generally quite high; for instance, for one course 100\% of students agreed that I was an effective teacher.
\end{itemize}
I have sought to develop and refine my skills as an educator, both by viewing each teaching assignment as my own opportunity for growth and learning and by actively seeking out learning from other educators and education experts. In particular, I have implemented evidence-based techniques to support and engage students from diverse backgrounds, especially LGBTQ+ students, women, and Black and Latinx/e students.
\\
\\
\noindent \textbf{II. Teaching Experiences.} As a graduate student at the University of Wisconsin - Madison, I served as a teaching assistant and course coordinator for Calculus I for multiple semesters, the instructor of record for Math for Early Education Majors, and the instructor of record for a Calculus I course providing students from generally under-represented groups additional support during their first college math course. Additionally, for several semesters, I held a non-traditional teaching assistantship for my role as the organizer of the Madison Math Circle outreach program. My passion for promoting an interest in and excitement for math -- especially for people from generally underrepresented groups -- led me to take on teaching and outreach roles through the \textit{Girls Math Night Out} program and the \textit{Wisconsin Directed Reading Program}. 

My postdoctoral positions at Brown University and the University of California, Berkeley did not allow me to have formal teaching responsibilities, however, I have actively sought out non-traditional teaching opportunities and mentoring opportunities. For example, in 2020, in response to the COVID-19 pandemic, I helped Ravi Vakil and others organize \textit{Algebraic Geometry in the Time of COVID}, a massive open-access virtual algebraic geometry course, which drew over 1500 participants from around the world. Inspired by this experience, in 2021, I organized an online open access course, \textit{Virtual Directed Reading in Geometry \& Algebra}, aimed at undergraduates. During this time I continued to seek to grow as an educator. While at the University of California, Berkeley I actively participated in a reading/working group exploring antiracist and anti-oppressive pedagogy in the mathematics classroom. Further, I personally sought to engage with ways to humanize mathematics and support Black and Latinx/e students and students from other underrepresented identities by exploring the works of Pamela E. Harris, Aris Winger, Rochelle Guti\'{e}rrez, Luis Leyva, and Francis Su. %Since returning to the classroom I have found that this has substantially shaped my time in the classroom. 

In the Fall of 2023, with a desire to return to the classroom and while working remotely at Brown University, I had the privilege to teach at San Francisco State University. I served as the instructor of record for two sections of Business Calculus and one section of Pre-Calculus I. Two of these sections were part of San Francisco State University's  ``supportive pathways program'' which seeks to improve the retention and success of students from underserved and underrepresented communities by providing them extra support. Further,  I participated in multiple teaching communities with the hope of learning more about how I could better support students from underserved and underrepresented communities. This has led to some of the most enriching and motivating teaching experiences I have had. 
\\
\\
\noindent \textbf{III. Teaching Philosophy and Strategies for Classroom Success.} As an instructor, I view my role is to be an active guide. I encourage my students to explore, engage with, and question the course material for themselves. I try to structure much of the course around guided group work that gives students opportunities to develop and discuss their understanding and confusion with their fellow students. In addition to encouraging students to take an active role in learning, this format also helps students to learn to vocalize their thought processes and ideas.

Active learning presents challenges to me and my students, most notably, the challenge of managing student mistakes. In many ways, the most significant moments during the learning process are not necessarily the moments of success, but the moments of failure. It is at this moment that students can recognize errors and gaps in their understanding of a subject and begin trying to correct them. It is also the moment that as an instructor I can understand what my students are finding difficult and nudge the conversation in such a way as to overcome these hurdles.

Making mistakes is hard, and most students, like most people, would prefer not to make mistakes. With this in mind, I think it is crucial to promote an inclusive environment where all students feel comfortable and safe participating. This environment encourages students to be open about what confuses them and where they are making mistakes. Creating an inclusive classroom environment requires active attention and work to maintain. However, in my experience, this work is well worth it.

My approach to creating an inclusive classroom environment has been influenced by the semester-long course \textit{Inclusive Practices in the College Classroom}, which I took through the \textit{Delta Program for Integrating Research, Teaching and Learning}. For example, one activity I implemented successfully asked students to brainstorm attributes from classes they found productive and attributes from classes they found less productive. After collecting a list of such attributes, we use this as a jumping-off point for forming community standards that we wish to shape our classroom environment. Examples of such community standards that my classes have often adopted include: ``Respect everyone'' and ``Address the problem, not the person when discussing mistakes''. I have found this helps the students buy into the belief that the classroom is an inclusive space where it is safe to make mistakes.

However, beyond simply creating an inclusive learning environment I have also found it important to create a space where students feel comfortable bringing their whole selves, including all of their experiences, backgrounds, challenges, identities, struggles, and knowledge. For example, I recognize that all students, like all people, will have days when negative experiences outside the classroom affect their ability to engage in the classroom. This is even more true for students who face racism, sexism, homo/transphobia, and other systems of oppression. On such a day when students enter the classroom, I look to try to meet the students where they are. For example, sometimes this means I will walk the student to the campus mental health or cultural center, or sometimes it means I create new problems specifically to help keep the student's mind off of whatever is troubling them. I try to make sure my students know I am there to provide them with whatever resources they need to succeed both in the classroom and in their life beyond. However, this human-centered approach also leads to many beautiful moments. For example, by allowing students to bring all of themselves to class they experience mathematics in a humanistic way, seeing how mathematics and quantitative thinking are an integral aspect of their life. I have found this often increases students' motivation, as well as opens themselves up to making mistakes, growing, and learning. 
\\
\\
\noindent \textbf{IV. Sample Student Feedback.}The effectiveness of my teaching is highlighted in student comments:
\begin{itemize}[leftmargin=*]
\item ``I’ve always struggle with math and I’ve had a lot of teachers that didn’t believe in me so because of this I’ve always dreaded math courses. But Juliette always showed she cared, was constantly encouraging, believed in our class, and taught the material really clearly. From her constant availability to help and great instructing, her class became one of my favorites and I am more successful in a math course than I’ve ever been before.''
\item ``She went around and tried helping each student... She cared about each student’s success in the class and tried her best to make everyone understand the material.''
\item ``Juliette obviously wants us to succeed not only in math but in life. She is always making sure we know our resources especially when it comes to health. She also always wishes us a good day/weekend and that is awesome.''
\end{itemize}\ \\
\noindent \textbf{V. Conclusion.} As a graduate student and postdoctoral scholar, I have found teaching to be extremely rewarding. I developed a passion for supporting and engaging students from diverse backgrounds.  Going forward, I am excited for new opportunities to grow and learn as a teacher, continue to promote inclusivity, diversity, and justice in my teaching, and create human-centered learning environments for my students. \end{document}