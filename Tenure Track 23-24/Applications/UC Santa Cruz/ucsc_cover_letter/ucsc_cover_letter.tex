\documentclass[11pt]{article}

\usepackage{graphicx}
\usepackage{color}
\usepackage[hidelinks]{hyperref}
\usepackage{fullpage}
\setlength{\topmargin}{-1.0in}
\setlength{\footskip}{0.5in}
\setlength{\textwidth}{6.5in}
\setlength{\textheight}{10.0in}
\setlength{\oddsidemargin}{-0.0in}
\setlength{\evensidemargin}{-0.0in}
\usepackage{parskip}
\usepackage{fancyhdr}
\pagestyle{fancy}
\addtolength{\headheight}{\baselineskip}
\addtolength{\headheight}{1in}
\addtolength{\textheight}{-\baselineskip}
\addtolength{\textheight}{-1in}
\lhead{}
\chead{\includegraphics[height=1in]{brownlogo.eps}}
\rhead{}
\lfoot{}
\cfoot{{\footnotesize
       Department of Mathematics \\
       Brown University \hspace{0.1in} 151 Thayer Street \hspace{0.1in} Providence, RI 02912 \\
       Phone: (401) 863-2708 \hspace{0.25in}
       Fax: (401) 863-9013 \hspace{0.25in}
       Web: \url{https://www.brown.edu/math/}
       }}
\rfoot{}
\renewcommand{\headrulewidth}{0pt}

\usepackage{enumitem}

\begin{document}

\section*{}

\noindent
\begin{minipage}{0.99\textwidth}
\begin{minipage}{0.69\textwidth}
\textcolor{white}{.}
\end{minipage}
\begin{minipage}{0.29\textwidth}
{
Juliette Bruce \\
Postdoctoral Researcher \\
Department of Mathematics \\
\href{mailto:juliette\_bruce1@brown.edu}{juliette\_bruce1@brown.edu}
%\url{juliettebruce.github.io}
%Phone: (810)--623--7610 
}

\vspace{12pt}
\today
\end{minipage}
\end{minipage}

%\vspace{12pt}
%\noindent
%John Doe \\
%Business, Inc. \\
%1234 Address Avenue \\
%City, State ZZZIP

\vspace{12pt}
\noindent
Dear Committee Members,

I am writing to apply for the assistant professor position in the Mathematics Department at the University of California, Santa Cruz. Currently, I am a postdoctoral researcher in the Mathematics Department at Brown University, a position I have held since August 2022. I received my Ph.D. in Mathematics from the University of Wisconsin-Madison under the guidance of Professor Daniel Erman in 2020. From 2020-2022 I was an NSF Postdoctoral Fellow in the Mathematics Department at the University of California, Berkeley. Additionally, I was a postdoctoral fellow at the Mathematical Sciences Research Institute -- now the Simons Laufer Mathematical Sciences Institute -- for the 2020-2021 academic year.

I am extremely interested in this position as I feel my qualifications and goals are well matched to the position. For example, as I describe in more detail below my research interests are in algebra and broadly touch on algebraic geometry and commutative algebra with connections to combinatorics, number theory, and recently arithmetics groups. In this position, I feel like I could achieve my goals of continuing a thriving research program in algebra, promoting an inclusive and supportive community both at the departmental and university level, as well as in the wider mathematical community, and supporting students from broad and diverse social and cultural backgrounds. In the Spring of 2022, I had the privilege of visiting the University of California, Santa Cruz. to speak in the Algebra \& Number Theory Seminar, and during my visit, I was extremely moved by how energetic the Mathematics Department was and how enthusiastic and thoughtful all the students I met were. I would love the opportunity to contribute to such a department, build collaborations and connections with colleagues in the department and beyond, and work with and mentor students. 

For example, would love to organize initiatives and programs aimed at supporting LGBTQ+ students, students of color, and women in mathematics. In the long term, I would love to organize a summer research program, similar to MSRI-UP, which promotes and supports LGBTQ+ undergraduate and graduate students by providing them with a supportive introduction to mathematics research and helps them transition to graduate school school. Given the University of California, Santa Cruz's commitment to diversity, social justice, and the environment I feel like organizing such a program at the University of California, Santa Cruz would be ideal. 

Finally,  two personal notes for my interest in the position: (i) currently my spouse (who works at a college in the Bay Area) and I are attempting to solve the two-body problem and would love to stay in the greater Bay Area, (ii) I find the University of California, Santa Cruz's long and storied commitment to the environment -- and programs like \textit{Rachel Carson College} -- extremely moving. 

My research interests lie in the intersection of algebraic geometry and commutative algebra with connections to combinatorics, number theory, and recently arithmetics groups. I am interested in using homological, combinatorial, and computational methods to study the geometry of algebraic varieties. Currently, my research program has two broad directions.
\begin{enumerate}[leftmargin=*,label=(\roman*)]
\item I have sought to deepen and expand our understanding of the ways homological algebra can be used to study the geometry of toric varieties. This seeks to generalize a very classical story using homological algebra to understand subvarieties of projective space.
\item I have been studying the geometry and topology of various moduli spaces, e.g., the moduli space of (principally polarized) abelian varieties of a fixed dimension, via combinatorially and homological methods. This has led to novel applications to the cohomology of certain arithmetic groups. 
\end{enumerate}
Further, I am passionate about promoting inclusivity, diversity, and justice in the mathematics community. This passion extends throughout my teaching where I am dedicated to creating an interactive and supportive classroom environment that helps students thrive.


My research output includes 15 papers, with publications in journals such as \textit{Algebra \& Number Theory}, \textit{Geometry \& Topology}, and \textit{Experimental Mathematics}, as well as, multiple published software packages. Below are a few of the non-research highlights of my file.

\begin{itemize}[leftmargin=*]
\item I was awarded an \textit{NSF Postdoctoral Research Fellowship}, an \textit{NSF Graduate Research Fellowship}, and I have secured over \$100,000 in conference grants, including 4 NSF conference grants. 
\item I have organized 12+ conferences, workshops, and special sessions, including multiple events aimed at supporting and promoting mathematicians from generally underrepresented groups, especially women and LGBTQ+ mathematicians. 
\item I was awarded the highest departmental and campus-wide teaching awards at the University of Wisconsin-Madison, the Capstone Teaching Award (2019) and the Teaching Assistant Award for Exceptional Service (2018), awarded to 1 and 3 students each year respectively. 
\end{itemize}

%I am especially interested in the position given the Department of Mathematics's and Dartmouth's commitment to fostering and supporting a diverse, equitable, and inclusive campus community. In particular, I would be excited to help contribute to and support programs such as the Women in Science Program, and would also be committed to creating new initiatives for example programs aimed at supporting LGBTQ+ students and students of color. 


With my application, I include the AMS cover sheet, a curriculum vitae, a research statement, a teaching statement, and a statement on contributions to equity, diversity, and inclusion. I will have six letters of recommendation. Five research letters: Christine Berkesch (\href{mailto:cberkesc@umn.edu}{cberkesc@umn.edu}), Melody Chan (\href{mailto:melody\_chan@brown.edu}{melody\_chan@brown.edu}), David Eisenbud  (\href{mailto:de@berkeley.edu
}{de@berkeley.edu}), Daniel Erman (\href{mailto:erman@hawaii.edu}{erman@hawaii.edu}), and Gregory G. Smith (\href{mailto:ggsmith@mast.queensu.ca}{ggsmith@mast.queensu.ca}), and one teaching letter from Shirin Malekpour (\href{mailto:shirin.malekpour@wisc.edu}{shirin.malekpour@wisc.edu}).  

Please do not hesitate to contact me with any questions, or if there is anything else I can provide, and thank you in advance for your consideration. 

\vspace{24pt}
\noindent
\begin{minipage}{0.99\textwidth}
\begin{minipage}{0.69\textwidth}
\textcolor{white}{.}
\end{minipage}
\begin{minipage}{0.29\textwidth}
Sincerely, 

\vspace{36pt}
Juliette Bruce\\
Postdoctoral Research Associate\end{minipage}
\end{minipage}

% This command changes the page style to plain from this page onward.
% If your letter is 1 page long, then comment this out.
% If your letter is 2 pages long, then include this command.
% If your letter is longer than 2 pages, then you may need to place
% this command earlier in the document.
%\pagestyle{plain}

\end{document}