\documentclass[11pt]{article}

\usepackage{graphicx}
\usepackage{color}
\usepackage[hidelinks]{hyperref}
\usepackage{fullpage}
\setlength{\topmargin}{-1.0in}
\setlength{\footskip}{0.5in}
\setlength{\textwidth}{6.5in}
\setlength{\textheight}{10.0in}
\setlength{\oddsidemargin}{-0.0in}
\setlength{\evensidemargin}{-0.0in}
\usepackage{parskip}
\usepackage{fancyhdr}
\pagestyle{fancy}
\addtolength{\headheight}{\baselineskip}
\addtolength{\headheight}{1in}
\addtolength{\textheight}{-\baselineskip}
\addtolength{\textheight}{-1in}
\lhead{}
\chead{\includegraphics[height=1in]{brownlogo.eps}}
\rhead{}
\lfoot{}
\cfoot{{\footnotesize
       Department of Mathematics \\
       Brown University \hspace{0.1in} 151 Thayer Street \hspace{0.1in} Providence, RI 02912 \\
       Phone: (401) 863-2708 \hspace{0.25in}
       Fax: (401) 863-9013 \hspace{0.25in}
       Web: \url{https://www.brown.edu/math/}
       }}
\rfoot{}
\renewcommand{\headrulewidth}{0pt}

\usepackage{enumitem}

\begin{document}

\section*{}

\noindent
\begin{minipage}{0.99\textwidth}
\begin{minipage}{0.69\textwidth}
\textcolor{white}{.}
\end{minipage}
\begin{minipage}{0.29\textwidth}
{
Juliette Bruce \\
Postdoctoral Researcher \\
Department of Mathematics \\
\href{mailto:juliette\_bruce1@brown.edu}{juliette\_bruce1@brown.edu}
%\url{juliettebruce.github.io}
%Phone: (810)--623--7610 
}

\vspace{12pt}
\today
\end{minipage}
\end{minipage}

%\vspace{12pt}
%\noindent
%John Doe \\
%Business, Inc. \\
%1234 Address Avenue \\
%City, State ZZZIP

\vspace{12pt}
\noindent
Dear Committee Members,

I am writing to apply for the tenure-track assistant professorship in the Department of Mathematics and Statistics at California State University, Sacramento. Currently, I am a postdoctoral researcher in the Mathematics Department at Brown University, a position I have held since August 2022. I received my Ph.D. in Mathematics from the University of Wisconsin-Madison under the guidance of my advisor Professor Daniel Erman in 2020. From 2020-2022 I was an NSF Postdoctoral Fellow in the Mathematics Department at the University of California, Berkeley. I was a postdoctoral fellow at the Mathematical Sciences Research Institute for the 2020-2021 academic year.

I am especially interested in this position because during the Fall of 2023 while working remotely at Brown University, I had the opportunity to teach at San Francisco State University, and from this, I saw the amazing impact the California State University system has on helping a wide range of students, especially those from underrepresented and underserved groups, achieve their dreams. This proved to be one of the most rewarding teaching experiences I have ever had, and I am motivated to pursue similar experiences where I can support and work with students from a wide range of backgrounds.  In this way working at California State University, Sacramento, and contributing via my research, teaching, and service would be a dream. This is furthered by the fact that much of my family lives in California, and I would love to be near them. 

Further, the Department of Mathematics and Statistics's commitment to just, equitable, and inclusive education aligns extremely well with my passion for making mathematics a more inclusive community and supporting students from underrepresented communities. For example, I have worked to make mathematics more inclusive of people from underrepresented groups by founding events like Trans Math Day and leading Spectra: the Association for LGBTQ+ Mathematicians. To promote the success of mathematicians from underrepresented groups, I organized numerous national and international workshops and conferences. Further, in the classroom, I have sought to implement practices to make all students feel welcome, valued, and supported. 

Going forward, I am excited to continue working hard to promote these values through my research, teaching, and service. In particular, I would be excited to help develop a curriculum that centers on the lives, experiences, and needs of underrepresented students. I am committed to continually trying to refine and advance my teaching practices, by seeking out new ways to better center the needs of students. Further, I would like to organize an undergraduate research experience for LGBTQ+ students.

My research interests lie in the intersection of algebraic geometry and commutative algebra with strong connections to computational and applied algebra. I am interested in using homological, combinatorial, and computational methods to study the geometry of algebraic varieties. Further, I am passionate about promoting inclusivity, diversity, and justice in the mathematics community. This passion extends throughout my teaching, where I am dedicated to creating an interactive and supportive classroom environment that helps students thrive.

Much of my work carries a significant computational component. I have co-authored four published software packages, and multiple publications that revolve around using high-throughput high-performance computing to explore new mathematical phenomena. Computation is a driving component of of my research and teaching, and I would love the opportunity to share my views on computation, mathematics, and science with my colleagues and students. 

My research output includes 15 papers, with publications in journals such as \textit{Algebra \& Number Theory}, \textit{Geometry \& Topology}, and \textit{Experimental Mathematics}, as well as, multiple published software packages. Below are a few of the non-research highlights of my file.

\begin{itemize}[leftmargin=*]
\item I was awarded an \textit{NSF Postdoctoral Research Fellowship}, an \textit{NSF Graduate Research Fellowship}, and I have secured over \$100,000 in conference grants including 4 NSF conference grants. 
\item I served as the inaugural president for \textit{Spectra: The Associatoin for LGBTQ+ Mathematicians}, and continue to serve on the organizations board. 
\item I have organized 12+ conferences, workshops, and special sessions, including multiple events aimed at supporting and promoting mathematicians from generally underrepresented groups, especially women and LGBTQ+ mathematicians. 
\item I was awarded the highest departmental and campus-wide teaching awards at the University of Wisconsin - Madison, the Capstone Teaching Award (2019) and the Teaching Assistant Award for Exceptional Service (2018), awarded to 1 and 3 students each year respectively. 
\end{itemize}

With my application, I include a curriculum vitae, a research statement, a teaching statement, a diversity statement, two representative publications, and a list of three references. 

Please do not hesitate to contact me if there is anything else I can provide, or with any questions, and thank you in advance for your consideration. 

\vspace{24pt}
\noindent
\begin{minipage}{0.99\textwidth}
\begin{minipage}{0.69\textwidth}
\textcolor{white}{.}
\end{minipage}
\begin{minipage}{0.29\textwidth}
Sincerely, 

\vspace{36pt}
Juliette Bruce\\
Postdoctoral Research Associate\end{minipage}
\end{minipage}

% This command changes the page style to plain from this page onward.
% If your letter is 1 page long, then comment this out.
% If your letter is 2 pages long, then include this command.
% If your letter is longer than 2 pages, then you may need to place
% this command earlier in the document.
%\pagestyle{plain}

\end{document}