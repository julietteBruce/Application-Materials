\documentclass[11pt]{article}

\usepackage{graphicx}
\usepackage{color}
\usepackage[hidelinks]{hyperref}
\usepackage{fullpage}
\setlength{\topmargin}{-1.0in}
\setlength{\footskip}{0.5in}
\setlength{\textwidth}{6.5in}
\setlength{\textheight}{10.0in}
\setlength{\oddsidemargin}{-0.0in}
\setlength{\evensidemargin}{-0.0in}
\usepackage{parskip}
\usepackage{fancyhdr}
\pagestyle{fancy}
\addtolength{\headheight}{\baselineskip}
\addtolength{\headheight}{1in}
\addtolength{\textheight}{-\baselineskip}
\addtolength{\textheight}{-1in}
\lhead{}
\chead{\includegraphics[height=1in]{brownlogo.eps}}
\rhead{}
\lfoot{}
\cfoot{{\footnotesize
       Department of Mathematics \\
       Brown University \hspace{0.1in} 151 Thayer Street \hspace{0.1in} Providence, RI 02912 \\
       Phone: (401) 863-2708 \hspace{0.25in}
       Fax: (401) 863-9013 \hspace{0.25in}
       Web: \url{https://www.brown.edu/math/}
       }}
\rfoot{}
\renewcommand{\headrulewidth}{0pt}

\usepackage{enumitem}

\begin{document}

\section*{}

\noindent
\begin{minipage}{0.99\textwidth}
\begin{minipage}{0.69\textwidth}
\textcolor{white}{.}
\end{minipage}
\begin{minipage}{0.29\textwidth}
{
Juliette Bruce \\
Postdoctoral Researcher \\
Department of Mathematics \\
\href{mailto:juliette\_bruce1@brown.edu}{juliette\_bruce1@brown.edu}
%\url{juliettebruce.github.io}
%Phone: (810)--623--7610 
}

\vspace{12pt}
\today
\end{minipage}
\end{minipage}

%\vspace{12pt}
%\noindent
%John Doe \\
%Business, Inc. \\
%1234 Address Avenue \\
%City, State ZZZIP

\vspace{12pt}
\noindent
Dear Committee Members,

I am writing to apply for the tenure-track assistant professor position in the Mathematics Department at Los Medanos College. Currently, I am a lecturer faculty in Department of Mathematics at San Francisco State University and a postdoctoral researcher in the Mathematics Department at Brown University. I received my Ph.D. in Mathematics from the University of Wisconsin-Madison under the guidance of my advisor Professor Daniel Erman in 2020. From 2020-2022 I was an NSF Postdoctoral Fellow in the Mathematics Department at the University of California, Berkeley. I was a postdoctoral fellow at the Mathematical Sciences Research Institute for 2020-2021.

I am interested in this position as I feel like my background and career goals are well matched NEDEDED. 

My goal as an educator is to be an active guide for students, providing them with environments where they feel supported and encouraged to let their own mathematical and quantitative curiosities guide how they engage and learn. By taking this approach, I hope to engage with students as the complete people that they are, asking them to bring all of their experiences, backgrounds, identities, and knowledge into the learning environment. I want students to experience mathematics in a humanistic way, seeing how mathematics and quantitative thinking are integral aspects of their lives.  As one of my former students noted, ``Juliette obviously wants us to succeed not only in math but in life.'' Recognizing that learning mathematics is not necessarily confined to the classroom I have sought out new and non-traditional teaching opportunities. 

In the Fall of 2023, with a desire to return to the classroom and while working remotely at Brown University, I had the privilege to teach at San Francisco State University. I served as the instructor of record for two sections of Business Calculus and one section of Pre-Calculus I. Two of these sections were part of San Francisco State University's  ``supportive pathways program'' which seeks to improve the retention and success of students from underserved and underrepresented communities by providing them extra support. Further,  I participated in multiple teaching communities with the hope of learning more about how I could better support students from underserved and underrepresented communities. This has led to some of the most enriching and motivating teaching experiences I have had. 


 during the Fall of 2023 while working remotely at Brown University, I had the opportunity to teach at San Francisco State University. This proved to be one of the most rewarding teaching experiences I have ever had, and I am motivated to pursue similar experiences where I can support and work with students in the Bay Area from a wide range of backgrounds. In this way working at Santa Clara University, and contributing as a teacher-scholar via my research, teaching, and service would be a dream. On a personal note, my family lives in California, and I would love to be near them. 

The Department of Mathematics and Computer Science's commitment to the academic success of all our students, aligns extremely well with my passion for making mathematics a more inclusive community and supporting students from underrepresented communities. For example, I have worked to make mathematics more inclusive of people from underrepresented groups by founding events like \textit{Trans Math Day} and leading \textit{Spectra: the Association for LGBTQ+ Mathematicians}. To promote the success of mathematicians from underrepresented groups, I organized numerous national and international workshops and conferences. Further, in the classroom, I have sought to implement inclusive pedagogical practices to make all students feel welcome, valued, and supported. Going forward, I would be excited to help develop curriculum that centers the lives, experiences, and needs of underrepresented students. 

Further, I am passionate about promoting inclusivity, diversity, and justice in the mathematics community. This passion extends throughout my teaching where I am dedicated to creating an interactive and supportive classroom environment that helps students thrive. 

A few of the highlights of my file include:

\begin{itemize}[leftmargin=*]
\item I was awarded the highest departmental and campus-wide teaching awards at the University of Wisconsin - Madison, the Capstone Teaching Award (2019) and the Teaching Assistant Award for Exceptional Service (2018), awarded to 1 and 3 students each year respectively. 
\item I organized two summer undergraduate research programs, and participated in one summer research program for students starting graduate school. One of the undergraduates I mentored was awards an \textit{NSF Graduate Research Fellowship} to student mathematics.  
\item I have served as 
\item I have organized 12+ conferences, workshops, and special sessions, including multiple events aimed at supporting and promoting mathematicians from generally underrepresented groups, especially women and LGBTQ+ mathematicians. 

\end{itemize}

With my application, I include a curriculum vitae, a teaching statement, a research statement, sample student evaluations, sample course materials, unofficial graduate transcripts, representative publications, and the contact information for three references. Please do not hesitate to contact me if there is anything else I can provide, or with any questions, and thank you for your consideration. 

\vspace{24pt}
\noindent
\begin{minipage}{0.99\textwidth}
\begin{minipage}{0.69\textwidth}
\textcolor{white}{.}
\end{minipage}
\begin{minipage}{0.29\textwidth}
Sincerely, 

\vspace{20pt}
Juliette Bruce\\
Postdoctoral Associate\end{minipage}
\end{minipage}

% This command changes the page style to plain from this page onward.
% If your letter is 1 page long, then comment this out.
% If your letter is 2 pages long, then include this command.
% If your letter is longer than 2 pages, then you may need to place
% this command earlier in the document.
%\pagestyle{plain}

\end{document}