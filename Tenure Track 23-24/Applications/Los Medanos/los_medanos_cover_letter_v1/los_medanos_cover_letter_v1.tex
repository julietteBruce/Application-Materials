\documentclass[11pt]{article}

\usepackage{graphicx}
\usepackage{color}
\usepackage[hidelinks]{hyperref}
\usepackage{fullpage}
\setlength{\topmargin}{-1.0in}
\setlength{\footskip}{0.5in}
\setlength{\textwidth}{6.5in}
\setlength{\textheight}{10.0in}
\setlength{\oddsidemargin}{-0.0in}
\setlength{\evensidemargin}{-0.0in}
\usepackage{parskip}
\usepackage{fancyhdr}
\pagestyle{fancy}
\addtolength{\headheight}{\baselineskip}
\addtolength{\headheight}{1in}
\addtolength{\textheight}{-\baselineskip}
\addtolength{\textheight}{-1in}
\lhead{}
\chead{\includegraphics[height=1in]{brownlogo.eps}}
\rhead{}
\lfoot{}
\cfoot{{\footnotesize
       Department of Mathematics \\
       Brown University \hspace{0.1in} 151 Thayer Street \hspace{0.1in} Providence, RI 02912 \\
       Phone: (401) 863-2708 \hspace{0.25in}
       Fax: (401) 863-9013 \hspace{0.25in}
       Web: \url{https://www.brown.edu/math/}
       }}
\rfoot{}
\renewcommand{\headrulewidth}{0pt}

\usepackage{enumitem}

\begin{document}

\section*{}

\noindent
\begin{minipage}{0.99\textwidth}
\begin{minipage}{0.69\textwidth}
\textcolor{white}{.}
\end{minipage}
\begin{minipage}{0.29\textwidth}
{
Juliette Bruce \\
Postdoctoral Researcher \\
Department of Mathematics \\
\href{mailto:juliette\_bruce1@brown.edu}{juliette\_bruce1@brown.edu}
%\url{juliettebruce.github.io}
%Phone: (810)--623--7610 
}

\vspace{12pt}
\today
\end{minipage}
\end{minipage}

%\vspace{12pt}
%\noindent
%John Doe \\
%Business, Inc. \\
%1234 Address Avenue \\
%City, State ZZZIP

\vspace{12pt}
\noindent
Dear Committee Members,

I am writing to apply for the tenure-track assistant professor position in the Mathematics Department at Los Medanos College. Currently, I am a lecturer faculty in the Department of Mathematics at San Francisco State University and a postdoctoral researcher in the Mathematics Department at Brown University. I received my Ph.D. in Mathematics from the University of Wisconsin-Madison under the guidance of my advisor Professor Daniel Erman in 2020. From 2020-2022 I was an NSF Postdoctoral Fellow in the Mathematics Department at the University of California, Berkeley. I was a postdoctoral fellow at the Mathematical Sciences Research Institute for 2020-2021.

I am particularly interested in this position as I have witnessed firsthand the crucial and transformative role community colleges play in helping individuals pursue and achieve their dreams and more broadly promote social justice. As a high school student, I took courses at the local community college, these courses not only benefited me academically but also opened my eyes to a more diverse world and instilled in me a strong belief in the importance of diversity and justice. More recently, I have had the privilege of working with many students for whom community college played a crucial role in their educational journey. For example, one of the graduate students I have mentored at Brown University began studying math at a college at Diablo Valley College. From these interactions, I've seen the huge positive and transformative impact community colleges can make on students, especially students who may be underserved by other educational institutions. This has made me passionate about wanting to work with such students and help them achieve their dreams. I feel like working in the Mathematics Department at Los Medanos College would provide me with such an opportunity. On a personal note, my spouse (who works in the Mathematics Department at Diablo Valley College) and I live in the Bay Area so this job is also of interest as we want to remain in the area.

I feel like my background and career goals are well matched to the position. I have been working with college math students in various roles for over 10 years, and have taught a variety of college mathematics courses including algebra, statistics, probability, pre-calculus, calculus, business calculus, and a variety of support courses for underserved students. During this time I've developed a passion and love of working with and supporting students, especially students from diverse backgrounds. My goal as an educator is to be an active guide for students, providing them with environments where they feel supported and encouraged to let their own mathematical and quantitative curiosities guide how they engage and learn. By taking this approach, I hope to engage with students as the complete people that they are, asking them to bring all of their experiences, backgrounds, identities, and knowledge into the learning environment. I want students to experience mathematics in a humanistic way, seeing how mathematics and quantitative thinking are integral aspects of their lives.  As one of my former students noted, ``Juliette obviously wants us to succeed not only in math but in life.''  I love to continue to help students succeed both in math and in life as part of the Math Department at Los Medanos College. 

Los Medanos College’s commitment to the academic success of all students and social justice aligns extremely well with my passion for making mathematics a more inclusive community and supporting students from underrepresented communities. I have worked to make mathematics more inclusive of people from underrepresented groups by founding events like \textit{Trans Math Day} and leading \textit{Spectra: the Association for LGBTQ+ Mathematicians}.  In the classroom, I have sought to implement inclusive pedagogical practices to make all students feel welcome, valued, and supported. Going forward, I am excited to help develop a curriculum that centers the lives, experiences, and needs of underrepresented students. I would love to create initiatives to support LGBTQ+ students. 

A few of the highlights of my file include:

\begin{itemize}[leftmargin=*]
\item I was awarded the highest departmental and campus-wide teaching awards at the University of Wisconsin - Madison, the Capstone Teaching Award (2019) and the Teaching Assistant Award for Exceptional Service (2018), awarded to 1 and 3 students each year respectively. 
\item I organized two summer undergraduate research programs, and participated in one summer research program for students starting graduate school. One of the undergraduates I mentored was awards an \textit{NSF Graduate Research Fellowship} to student mathematics.  
\item I have served as board member, including the inaugural president, for \textit{Spectra: The Association for LGBTQ+ Mathematicians} which seeks to supprot and promote LGBTQ+ researchers, students, and teachers in mathematics. 
\item I have organized 12+ conferences, workshops, and special sessions, including multiple events aimed at supporting and promoting mathematicians from generally underrepresented groups, especially women and LGBTQ+ mathematicians. 
\end{itemize}

Finally the effectiveness of my teaching is highlighted in student comments such as the following:
\begin{itemize}[leftmargin=*]
\item ``I’ve always struggle with math and I’ve had a lot of teachers that didn’t believe in me so because of this I’ve always dreaded math courses. But Juliette always showed she cared, was constantly encouraging, believed in our class, and taught the material really clearly. From her constant availability to help and great instructing, her class became one of my favorites and I am more successful in a math course than I’ve ever been before.''
\item ``She went around and tried helping each student... She cared about each student’s success in the class and tried her best to make everyone understand the material.''
\item ``Juliette obviously wants us to succeed not only in math but in life. She is always making sure we know our resources especially when it comes to health. She also always wishes us a good day/weekend and that is awesome.''
\end{itemize}

With my application, I include a curriculum vitae, copies of my college transcripts, three professional references, and answers to the supplementary questions.  Please do not hesitate to contact me if there is anything else I can provide, or with any questions, and thank you for your consideration. 

\vspace{24pt}
\noindent
\begin{minipage}{0.99\textwidth}
\begin{minipage}{0.69\textwidth}
\textcolor{white}{.}
\end{minipage}
\begin{minipage}{0.29\textwidth}
Sincerely, 

\vspace{20pt}
Juliette Bruce\\
Postdoctoral Associate\end{minipage}
\end{minipage}

% This command changes the page style to plain from this page onward.
% If your letter is 1 page long, then comment this out.
% If your letter is 2 pages long, then include this command.
% If your letter is longer than 2 pages, then you may need to place
% this command earlier in the document.
%\pagestyle{plain}

\end{document}