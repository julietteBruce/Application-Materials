\documentclass[11pt]{article}

\usepackage{graphicx}
\usepackage{color}
\usepackage[hidelinks]{hyperref}
\usepackage{fullpage}
\setlength{\topmargin}{-1.0in}
\setlength{\footskip}{0.5in}
\setlength{\textwidth}{6.5in}
\setlength{\textheight}{10.2in}
\setlength{\oddsidemargin}{-0.0in}
\setlength{\evensidemargin}{-0.0in}
\usepackage{parskip}
\usepackage{fancyhdr}
\pagestyle{fancy}
\addtolength{\headheight}{\baselineskip}
\addtolength{\headheight}{1in}
\addtolength{\textheight}{-\baselineskip}
\addtolength{\textheight}{-1in}
\lhead{}
\chead{\includegraphics[height=1in]{brownlogo.eps}}
\rhead{}
\lfoot{}
\cfoot{{\footnotesize
       Department of Mathematics \\
       Brown University \hspace{0.1in} 151 Thayer Street \hspace{0.1in} Providence, RI 02912 \\
       %Phone: (401) 863-2708 \hspace{0.25in}
       %Fax: (401) 863-9013 \hspace{0.25in}
       %Web: \url{https://www.brown.edu/math/}
       }}
\rfoot{}
\renewcommand{\headrulewidth}{0pt}

\usepackage{enumitem}

\begin{document}

\section*{}

\noindent
\begin{minipage}{0.99\textwidth}
\begin{minipage}{0.69\textwidth}
\textcolor{white}{.}
\end{minipage}
\begin{minipage}{0.29\textwidth}
{
Juliette Newton \\
Postdoctoral Researcher \\
Department of Mathematics \\
\href{mailto:juliette\_bruce1@brown.edu}{juliette\_bruce1@brown.edu}
%\url{juliettebruce.github.io}
%Phone: (810)--623--7610 
}

\vspace{12pt}
\today
\end{minipage}
\end{minipage}

%\vspace{12pt}
%\noindent
%John Doe \\
%Business, Inc. \\
%1234 Address Avenue \\
%City, State ZZZIP

\vspace{12pt}
\noindent
Dear Committee Members,

I am writing to apply for the tenure-track mathematics instructor position at De Anza College. Currently, I am a lecturer faculty in the Department of Mathematics at San Francisco State University and a postdoctoral researcher in the Mathematics Department at Brown University. I received my Ph.D. in Mathematics from the University of Wisconsin-Madison under the guidance of my advisor Professor Daniel Erman in 2020 \textbf{(MQ2)}.\footnote{I use \textbf{MQ} and $\textbf{PQ}$ to indicate experiences and skills that I feel match the minimum and preferred qualifications listed in the job posting.}  From 2020-2022 I was an NSF Postdoctoral Fellow in the Mathematics Department at the University of California, Berkeley. I was a postdoctoral fellow at the Mathematical Sciences Research Institute for 2020-2021.I am particularly interested in this position as I have witnessed firsthand the crucial and transformative role community colleges play in helping individuals pursue and achieve their dreams and more broadly promote social justice. 

%As a high school student, I took courses at the local community college, these courses not only benefited me academically but also opened my eyes to a more diverse world and instilled in me a strong belief in the importance of diversity and justice. More recently, I have had the privilege of working with many students for whom community college played a crucial role in their educational journey. From these interactions, I've seen the huge positive and transformative impact community colleges can make on students, especially students who may be underserved by other educational institutions. This has made me passionate about wanting to work with such students and help them achieve their dreams. I feel like working in the Department of Mathematics Department at De Anza College would provide me with such an opportunity. On a personal note, my spouse (who works in the Mathematics Department at Diablo Valley College) and I live in the Bay Area so this job is also of interest as we want to remain in the area.

I feel like my background and career goals are well matched to the position. I have been working with and teaching college math students in various roles for over 10 years, and have taught a variety of college mathematics courses including algebra, statistics, probability, pre-calculus, calculus, business calculus, and a variety of support courses for underserved students \textbf{(PQ4)}. During this time I've developed a passion and love of working with and supporting students, especially students from diverse backgrounds. Further, I feel confident that my teaching, research, and educational background make me prepared to teach all mathematics and statistics courses offered by the department, and I would love to teach in a way that shares my excitement and curiosity for mathematics with students \textbf{(PQ4)}. 

\noindent \textbf{I. Teaching Experiences.} As a graduate student at the University of Wisconsin - Madison, I served as a teaching assistant and course coordinator for Calculus I for multiple semesters, the instructor of record for Math for Early Education Majors, and the instructor of record for a Calculus I course providing students from generally under-represented groups additional support during their first college math course \textbf{(PQ12, 13)}. Additionally, for several semesters, I held a non-traditional teaching assistantship for my role as the organizer of the Madison Math Circle outreach program. % My passion for promoting an interest in and excitement for math -- especially for people from generally underrepresented groups -- led me to take on teaching and outreach roles through the \textit{Girls Math Night Out} program and the \textit{Wisconsin Directed Reading Program}. 

My postdoctoral position at the University of California, Berkeley did not allow me to have formal teaching responsibilities, however, I have actively sought out non-traditional teaching opportunities and mentoring opportunities. In 2020, in response to the COVID-19 pandemic, I helped Ravi Vakil and others organize \textit{Algebraic Geometry in the Time of COVID}, a massive open-access virtual algebraic geometry course, which drew over 1500 participants from around the world  \textbf{(PQ9)}. During this time I continued to seek to grow as an educator. For example, while at the University of California, Berkeley I actively participated in a reading/working group exploring antiracist and anti-oppressive pedagogy in the mathematics classroom. Further, I personally sought to engage with ways to humanize mathematics and support underrepresented students by exploring the works of Pamela E. Harris, Aris Winger, Rochelle Guti\'{e}rrez, Luis Leyva, and Francis Su \textbf{(PQ1,2,7)}. %Since returning to the classroom I have found that this has substantially shaped my time in the classroom. 

In the Fall of 2023, with a desire to return to the classroom and while working remotely at Brown University, I had the privilege to teach at San Francisco State University. I served as the instructor of record for two sections of Business Calculus and one section of Pre-Calculus I \textbf{(MQ1)}. During this time I worked to further integrate Microsoft Excel and Desmos into the classroom to provide students with different ways of interacting with the material \textbf{(PQ9)}. Two of these sections were part of San Francisco State University's  ``supportive pathways program'' which seeks to improve the retention and success of students from underserved and underrepresented communities by providing them extra support \textbf{(PQ12,13)}. Further,  I participated in multiple teaching communities with the hope of learning more about how I could better support students from underserved and underrepresented communities \textbf{(PQ1,2,3,7)}. %This has led to some of the most enriching and motivating teaching experiences I have had. 
\\
\\
\noindent \textbf{II. Strategies for Classroom Success.} As an instructor, I view my role is to be an active guide. I encourage my students to explore, engage with, and question the course material for themselves. I try to structure much of the course around a mix of short lectures, guided group work, and other instructional methods that gives students opportunities to develop and discuss their understanding and confusion with their fellow students. In addition to encouraging students to take an active role in learning, these formats also helps students to learn to vocalize their thought processes and ideas \textbf{(PQ2,5,8)}.

Active learning presents challenges to me and my students, most notably, the challenge of managing student mistakes. In many ways, the most significant moments during the learning process are not necessarily the moments of success, but the moments of failure. It is at this moment that students can recognize errors and gaps in their understanding of a subject and begin trying to correct them. It is also the moment that as an instructor I can understand what my students are finding difficult and nudge the conversation in such a way as to overcome these hurdles. Making mistakes is hard, and most students, like most people, prefer not to make mistakes. With this in mind, I think it is crucial to promote an inclusive environment where all students feel comfortable and safe participating \textbf{(PQ11)}. This environment encourages students to be open about what confuses them and where they are making mistakes. %Creating an inclusive classroom environment requires active attention and work to maintain. However, this work is well worth it.

My approach to creating an inclusive classroom environment has been influenced by the semester-long course \textit{Inclusive Practices in the College Classroom}, which I took through the \textit{Delta Program for Integrating Research, Teaching and Learning} \textbf{(MQ1, PQ1,2,7)}. For example, one activity I implemented successfully asked students to brainstorm attributes from classes they found productive and attributes from classes they found less productive. After collecting a list of such attributes, we use this as a jumping-off point for forming community standards that we wish to shape our classroom environment. Examples of such community standards that my classes have often adopted include: ``Respect everyone'' and ``Address the problem, not the person when discussing mistakes''. I have found this helps the students buy into the belief that the classroom is an inclusive space where it is safe to make mistakes \textbf{(PQ1,5,11,12)}. Additionally, during this discussion the students and I worked to come to a community understanding of what forms of assessment we felt would be fair, equitable, and promote their learning \textbf{(PQ14)}.

However, beyond simply creating an inclusive learning environment I have also found it important to create a space where students feel comfortable bringing their whole selves, including all of their experiences, backgrounds, challenges, identities, struggles, and knowledge. For example, I recognize that all students, like all people, will have days when negative experiences outside the classroom affect their ability to engage in the classroom. This is even more true for students who face racism, sexism, homo/transphobia, and other systems of oppression. On such a day when students enter the classroom, I look to try to meet the students where they are. For example, sometimes this means I will walk the student to the campus mental health or cultural center, or sometimes it means I create new problems specifically to help keep the student's mind off of whatever is troubling them. I try to make sure my students know I am there to provide them with whatever resources they need to succeed both in the classroom and in their life beyond \textbf{(MQ1, PQ1,2,3)}. This human-centered approach also leads to many beautiful moments. For example, by allowing students to bring all of themselves to class they experience mathematics in a humanistic way, seeing how mathematics and quantitative thinking are an integral aspect of their life. %I have found this often increases students' motivation, as well as opens themselves up to making mistakes, growing, and learning. 


De Anza College's commitment to the academic success of all students and social justice aligns extremely well with my passion for making mathematics a more inclusive community and supporting students from underrepresented communities. I have worked to make mathematics more inclusive of people from underrepresented groups by founding events like \textit{Trans Math Day} and leading \textit{Spectra: the Association for LGBTQ+ Mathematicians} \textbf{(MQ1,PQ10)}.  In the classroom, I have sought to implement inclusive pedagogical practices to make all students feel welcome, valued, and supported. Going forward, I am excited to help develop a curriculum that centers the lives, experiences, and needs of underrepresented students \textbf{(PQ6)}. I would love to create initiatives to support students, especially LGBTQ+ students and women  \textbf{(PQ10)}. 

A few of the highlights of my file include:

\begin{itemize}[leftmargin=*]
\item I was awarded the highest departmental and campus-wide teaching awards at the University of Wisconsin - Madison, the Capstone Teaching Award (2019) and the Teaching Assistant Award for Exceptional Service (2018), awarded to 1 and 3 students each year respectively. 
%\item I organized two summer undergraduate research programs, and participated in one summer research program for students starting graduate school. One of the undergraduates I mentored was awards an \textit{NSF Graduate Research Fellowship} to student mathematics.  
\item I served as board member, including the inaugural president, for \textit{Spectra: The Association for LGBTQ+ Mathematicians} which supports LGBTQ+ people in mathematics. 
%\item I have organized 12+ conferences, workshops, and special sessions, including multiple events aimed at supporting and promoting mathematicians from generally underrepresented groups, especially women and LGBTQ+ mathematicians. 
\end{itemize}

Finally the effectiveness of my teaching is highlighted in student comments such as the following:
\begin{itemize}[leftmargin=*]
\item ``I’ve always struggle with math and I’ve had a lot of teachers that didn’t believe in me so because of this I’ve always dreaded math courses. But Juliette always showed she cared, was constantly encouraging, believed in our class, and taught the material really clearly. From her constant availability to help and great instructing, her class became one of my favorites and I am more successful in a math course than I’ve ever been before.''
\item ``She went around and tried helping each student... She cared about each student’s success in the class and tried her best to make everyone understand the material.''
\item ``Juliette obviously wants us to succeed not only in math but in life. She is always making sure we know our resources especially when it comes to health. ''
\end{itemize}

With my application, I include a curriculum vitae, copies of my college transcripts, a list of courses I have taught, and answers to the supplementary questions.  Please do not hesitate to contact me if there is anything else I can provide, or with any questions, and thank you for your consideration. 

\vspace{24pt}
\noindent
\begin{minipage}{0.99\textwidth}
\begin{minipage}{0.69\textwidth}
\textcolor{white}{.}
\end{minipage}
\begin{minipage}{0.29\textwidth}
Sincerely, 

\vspace{18pt}
Juliette Newton\\
Postdoctoral Associate\end{minipage}
\end{minipage}

% This command changes the page style to plain from this page onward.
% If your letter is 1 page long, then comment this out.
% If your letter is 2 pages long, then include this command.
% If your letter is longer than 2 pages, then you may need to place
% this command earlier in the document.
%\pagestyle{plain}

\end{document}