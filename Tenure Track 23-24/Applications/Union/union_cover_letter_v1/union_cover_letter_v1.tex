\documentclass[11pt]{article}

\usepackage{graphicx}
\usepackage{color}
\usepackage[hidelinks]{hyperref}
\usepackage{fullpage}
\setlength{\topmargin}{-1.0in}
\setlength{\footskip}{0.5in}
\setlength{\textwidth}{6.5in}
\setlength{\textheight}{10.0in}
\setlength{\oddsidemargin}{-0.0in}
\setlength{\evensidemargin}{-0.0in}
\usepackage{parskip}
\usepackage{fancyhdr}
\pagestyle{fancy}
\addtolength{\headheight}{\baselineskip}
\addtolength{\headheight}{1in}
\addtolength{\textheight}{-\baselineskip}
\addtolength{\textheight}{-1in}
\lhead{}
\chead{\includegraphics[height=1in]{brownlogo.eps}}
\rhead{}
\lfoot{}
\cfoot{{\footnotesize
       Department of Mathematics \\
       Brown University \hspace{0.1in} 151 Thayer Street \hspace{0.1in} Providence, RI 02912 \\
       Phone: (401) 863-2708 \hspace{0.25in}
       Fax: (401) 863-9013 \hspace{0.25in}
       Web: \url{https://www.brown.edu/math/}
       }}
\rfoot{}
\renewcommand{\headrulewidth}{0pt}

\usepackage{enumitem}

\begin{document}

\section*{}

\noindent
\begin{minipage}{0.99\textwidth}
\begin{minipage}{0.69\textwidth}
\textcolor{white}{.}
\end{minipage}
\begin{minipage}{0.29\textwidth}
{
Juliette Bruce \\
Postdoctoral Researcher \\
Department of Mathematics \\
\href{mailto:juliette\_bruce1@brown.edu}{juliette\_bruce1@brown.edu}
%\url{juliettebruce.github.io}
%Phone: (810)--623--7610 
}

\vspace{12pt}
\today
\end{minipage}
\end{minipage}

%\vspace{12pt}
%\noindent
%John Doe \\
%Business, Inc. \\
%1234 Address Avenue \\
%City, State ZZZIP

\vspace{12pt}
\noindent
Dear Committee Members,

I am writing to apply for the tenure-track position as an assistant professor in the Department of Mathematics at Union College. Currently, I am a postdoctoral researcher in the Mathematics Department at Brown University, a position I have held since August 2022. I received my Ph.D. in Mathematics from the University of Wisconsin-Madison under the guidance of my advisor Professor Daniel Erman in 2020. From 2020-2022 I was an NSF Postdoctoral Fellow in the Mathematics Department at the University of California, Berkeley. Additionally, I was a postdoctoral fellow at the Mathematical Sciences Research Institute for the 2020-2021 academic year.

I am especially interested in this position given Union College's commitment to providing an excellent liberal arts education to a diverse student body. In the Winter of 2023, I feel like this position is well matched to my goals of working to support students from a wide range of backgrounds, making the mathematics community more inclusive for LGBTQ+ people, and establishing a thriving research program that significantly incorporates undergraduate research. I would be extremely excited to work with such amazing students from diverse backgrounds both in and out of the classroom and on research experiences. For example, would love to organize initiatives and programs aimed at supporting LGBTQ+ students, students of color, and women in mathematics. In the long term, I would love to organize an REU program, similar to  MSRI-UP, which promotes and supports LGBTQ+ students exploring mathematics research and pursuing graduate school. 

During my time as a post-doc, I organized two summer undergraduate research programs and participated in one summer research program for students entering graduate school. I like to approach undergraduate mathematics research as a holistic opportunity for students to explore their curiosities and develop new skills both in and out of mathematics. For example, I feel like undergraduate research presents fantastic opportunities for students to develop their writing, communication, programming, and artistic skills. With this in mind, I view a crucial aspect of my role in supporting undergraduate research in mathematics as carefully tailoring projects to the interests and goals of the students. For example,  one of the students I did a research project with was very excited and interested in learning computer programming so we shifted the project goals so that the student learned to program in \textit{Macaulay2} and then used these skills to computationally explore the problem I originally proposed. These methods have led the students I have worked with to find joy and excitement in mathematics research and develop skills that will serve them well beyond mathematics research. For example, one of the undergraduates I mentored was awarded an \textit{NSF Graduate Research Fellowship} to student mathematics.  

My research interests lie in the intersection of algebraic geometry and commutative algebra with connections to combinatorics and number theory. I am interested in using homological, combinatorial, and computational methods to study the geometry of algebraic varieties. Currently, my research program has two broad directions.
\begin{enumerate}[leftmargin=*,label=(\roman*)]
\item I have sought to deepen and expand our understanding of the ways homological algebra can be used to study the geometry of toric varieties. This seeks to generalize a very classical story using homological algebra to understand subvarieties of projective space.
\item I have been studying the geometry and topology of various moduli spaces, e.g., the moduli space of (principally polarized) abelian varieties of a fixed dimension, via combinatorially and homological methods. This has led to novel applications to the cohomology of certain arithmetic groups. 
\end{enumerate}
Further, I am passionate about promoting inclusivity, diversity, and justice in the mathematics community. This passion extends throughout my teaching where I am dedicated to creating an interactive and supportive classroom environment that helps students thrive. 

My research output includes 15 papers, with publications in journals such as \textit{Algebra \& Number Theory}, \textit{Geometry \& Topology}, and \textit{Experimental Mathematics}, as well as, multiple published software packages. Below are a few of the non-research highlights of my file.
\begin{itemize}[leftmargin=*]
\item I was awarded an \textit{NSF Postdoctoral Research Fellowship}, an \textit{NSF Graduate Research Fellowship}, and I have secured over \$100,000 in conference grants, including 4 NSF conference grants. 
\item I have organized 12+ conferences, workshops, and special sessions, including multiple events aimed at supporting and promoting mathematicians from generally underrepresented groups, especially women and LGBTQ+ mathematicians. 
\item I was awarded the highest departmental and campus-wide teaching awards at the University of Wisconsin-Madison, the Capstone Teaching Award (2019), and the Teaching Assistant Award for Exceptional Service (2018), awarded to 1 and 3 students each year respectively. 
\item I have served on the board of \textit{Spectra, The Association for LGBTQ+ Mathematicians} since 2020, including serving as the inaugural president during 2022. 
\item I organized two summer undergraduate research programs, and participated in one summer research program for students starting graduate school. One of the undergraduates I mentored was awards an \textit{NSF Graduate Research Fellowship} to student mathematics.  
\end{itemize}

With my application, I include a curriculum vitae, a research statement, a teaching statement, and a diversity statement. I have three letters of recommendation, Melody Chan (\href{mailto:melody\_chan@brown.edu}{melody\_chan@brown.edu}), Daniel Erman (\href{mailto:erman@hawaii.edu}{erman@hawaii.edu}), and Shirin Malekpour (\href{mailto:shirin.malekpour@wisc.edu}{shirin.malekpour@wisc.edu}).  

Please do not hesitate to contact me if there is anything else I can provide, or with any questions, and thank you in advance for your consideration. 

\vspace{24pt}
\noindent
\begin{minipage}{0.99\textwidth}
\begin{minipage}{0.69\textwidth}
\textcolor{white}{.}
\end{minipage}
\begin{minipage}{0.29\textwidth}
Sincerely, 

\vspace{36pt}
Juliette Bruce\\
Postdoctoral Research Associate\end{minipage}
\end{minipage}

% This command changes the page style to plain from this page onward.
% If your letter is 1 page long, then comment this out.
% If your letter is 2 pages long, then include this command.
% If your letter is longer than 2 pages, then you may need to place
% this command earlier in the document.
%\pagestyle{plain}

\end{document}