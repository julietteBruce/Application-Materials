%
% brownletter_example.tex - an example latex file to illustrate brownletter.cls
%
% Copyright 2003, Nesime Tatbul (tatbul@cs.brown.edu)
%

\documentclass[11pt]{brownletter}

\usepackage{amsfonts,amsmath,amssymb,amsbsy,amstext,amsthm,mathtools}

\newcommand{\C}{\mathbb{C}}
\newcommand{\PP}{\mathbb{P}}


\name{Nesime Tatbul} % used as signature, if no signature is specified

\signature{Juliette Bruce}

\date{October 14, 2023} % if no date specified, today's date is used 

%\subject{Application for Lecturer Posititon} % optional subject line

\begin{document}

\begin{letter}{Juliette Bruce\\ 
               Brown University\\ 
               Providence, RI 02912}

\opening{Dear Committee Members,}

I am writing to apply for a position as Mathematics Instructional Assistant at Las Positas College. Currently, I am a postdoctoral researcher in the Mathematics Department at Brown University, and a lecturer faculty member at San Francisco State University. I received my Ph.D. in Mathematics from the University of Wisconsin - Madison in 2020 under the guidance of my advisor Professor Daniel Erman. From 2020-2022 I was an NSF Postdoctoral Fellow in the Mathematics Department at the University of California, Berkeley.

I am passionate about promoting inclusivity, diversity, and justice in the mathematics community. This passion extends throughout my teaching where I am dedicated to creating an interactive, engaging, and supportive classroom environment that helps students thrive. As a graduate student at the University of Wisconsin - Madison, I served as a teaching assistant and course coordinator for Calculus I for multiple semesters, the instructor of record for Math for Early Education Majors, and the instructor of record for a Calculus I course providing students from generally under-represented groups additional support during their first college math course. Additionally, for several semesters, I held a non-traditional teaching assistantship for my role as the organizer of the Madison Math Circle outreach program. My passion for promoting an interest in and excitement for math – especially for people from generally underrepresented groups – led me to take on teaching and outreach roles through the Girls Math Night Out program and the Wisconsin Directed Reading Program.

My postdoctoral positions at Brown University and the University of California, Berkeley did not allow me to have formal teaching responsibilities, however, I have spent this time exploring forms of informal forms of teaching and mentoring. For example, during the Summer and Fall of 2020, in response to the COVID-19 pandemic, I helped Ravi Vakil and others organize Algebraic Geometry in the Time of COVID, a massive open-access virtual algebraic geometry course, which drew over 1,200 participants from around the world. Inspired by this experience, in the Winter of 2021, I organized Virtual Directed Reading in Geometry \& Algebra a virtual open-access directed reading group for undergraduates interested in algebraic geometry and commutative algebra. During the summer of 2021, I led a group of 6 undergraduate students in a summer research project in combinatorial algebraic geometry, and during the summer of 2022, I led a summer research project with one undergraduate student.

As a lecturer at San Francisco State University, I taught multiple sections of Business Calculus and a section of Pre-Calculus. Additionally, given my background supporting students two of my sections are part of San Francisco State's ``Supportive Pathways Program'' which provides students from underserved and underrepresented groups with additional support, community, and guidance. 


As an instructor, I view my role as being an active guide. I encourage my students to explore, engage with, and question the course material for themselves. I try to structure much of the course around guided group work that gives students opportunities to develop and discuss their understanding and confusion with their fellow students. In addition to encouraging students to take an active role in learning, this format also helps students to learn to vocalize their thought processes and ideas.

Active learning presents challenges to me and my students, most notably, the challenge of managing student mistakes. In many ways, the most significant moments during the learning process are not necessarily the moments of success, but the moments of failure. It is at this moment that students can recognize errors and gaps in their understanding of a subject and can begin trying to correct them. It is also the moment that as an instructor I can understand what my students are finding difficult and nudge the conversation in such a way as to overcome these hurdles.
Making mistakes is hard, and most students, like most people, would prefer not to make mistakes. With this in mind, I think it is crucial to promote an inclusive environment where all students feel comfortable and safe participating. This environment encourages students to be open about what confuses them and where they are making mistakes.

With my application I have included my curriculum vitae. A few highlights in my file are:
\begin{itemize}

\item Capstone Teaching Award (2019) - Awarded by the Department of Mathematics at the University of Wisconsin to 1 student recognizing a record of exceptional teaching and service.
\item Teaching Assistant Award for Exceptional Service  (2018) - Awarded by the University of Wisconsin to up to 3 teaching assistants campus-wide recognizing their exceptional service.

\end{itemize}

Please contact me with any questions, and thank you in advance for your consideration.
\vspace{.1 in}
\\
Sincerely,
\\
Juliette Bruce

% \encl{brownletter.cls}

% \ps{Please see the enclosed file.}

% \cc{J. Kirschenbaum}

\end{letter}

\end{document}