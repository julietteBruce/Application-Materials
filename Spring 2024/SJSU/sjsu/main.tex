%
% brownletter_example.tex - an example latex file to illustrate brownletter.cls
%
% Copyright 2003, Nesime Tatbul (tatbul@cs.brown.edu)
%

\documentclass[11pt]{brownletter}

\usepackage{amsfonts,amsmath,amssymb,amsbsy,amstext,amsthm,mathtools}

\newcommand{\C}{\mathbb{C}}
\newcommand{\PP}{\mathbb{P}}


\name{Nesime Tatbul} % used as signature, if no signature is specified

\signature{Juliette Bruce}

\date{November 3, 2023} % if no date specified, today's date is used 

%\subject{Application for Lecturer Posititon} % optional subject line

\begin{document}

\begin{letter}{Juliette Bruce\\ 
               Brown University\\ 
               Providence, RI 02912}

\opening{Dear Committee Members,}

I am writing to apply for a position as a part-rime lecturer in mathematics and statistics at San Jose State University. I am interested in position because I am pasionate about teaching and working with students, especially students from diverse backgrounds. Currently, I am a lecturer faculty member at San Francisco State University, and a postdoctoral researcher in the Mathematics Department at Brown University. (Note that I live in San Francisco, CA, as I am working at Brown University remotely part-time.)  I received my Ph.D. in Mathematics from the University of Wisconsin - Madison in 2020 under the guidance of my advisor Professor Daniel Erman. From 2020-2022 I was an NSF Postdoctoral Fellow in the Mathematics Department at the University of California, Berkeley.

I care deeplu about promoting inclusivity, diversity, and justice in the mathematics community. This passion extends throughout my teaching where I am dedicated to creating an interactive, engaging, and supportive classroom environment that helps students thrive. My goal as an educator is to be an active guide for students, providing them with environments where they feel supported and encouraged to let their own mathematical and quantitative curiosities guide how they engage and learn. By taking this approach, I hope to engage with students as the complete people that they are, asking them to bring all of their experiences, backgrounds, identities, and knowledge into the learning environment. I want students to experience mathematics in a humanistic way, seeing how mathematics and quantitative thinking are integral aspects of their lives. As one of my former students noted, ``Juliette obviously wants us to succeed not only in math but in life.''


San Jose State University’s commitment to the academic success of all students and social justice aligns extremely well with my passion for making mathematics a more inclusive community and supporting students from underrepresented communities. I have worked to make mathematics more inclusive of people from underrepresented groups by founding events like \textit{Trans Math Day} and leading \textit{Spectra: the Association for LGBTQ+ Mathematicians}.  In the classroom, I have sought to implement inclusive pedagogical practices to make all students feel welcome, valued, and supported. Going forward, I am excited to help develop a curriculum that centers the lives, experiences, and needs of underrepresented students. 

A few of the highlights of my file include:

\begin{itemize}[leftmargin=*]
\item I was awarded the highest departmental and campus-wide teaching awards at the University of Wisconsin - Madison, the Capstone Teaching Award (2019) and the Teaching Assistant Award for Exceptional Service (2018), awarded to 1 and 3 students each year respectively. 
\item I organized two summer undergraduate research programs, and participated in one summer research program for students starting graduate school. One of the undergraduates I mentored was awards an \textit{NSF Graduate Research Fellowship} to student mathematics.  
\item I have served as board member, including the inaugural president, for \textit{Spectra: The Association for LGBTQ+ Mathematicians} which seeks to supprot and promote LGBTQ+ researchers, students, and teachers in mathematics. 
\item I have organized 12+ conferences, workshops, and special sessions, including multiple events aimed at supporting and promoting mathematicians from generally underrepresented groups, especially women and LGBTQ+ mathematicians. 
\end{itemize}

Finally the effectiveness of my teaching is highlighted in student comments such as the following:
\begin{itemize}[leftmargin=*]
\item ``I’ve always struggle with math and I’ve had a lot of teachers that didn’t believe in me so because of this I’ve always dreaded math courses. But Juliette always showed she cared, was constantly encouraging, believed in our class, and taught the material really clearly. From her constant availability to help and great instructing, her class became one of my favorites and I am more successful in a math course than I’ve ever been before.''
\item ``She went around and tried helping each student... She cared about each student’s success in the class and tried her best to make everyone understand the material.''
\item ``Juliette obviously wants us to succeed not only in math but in life. She is always making sure we know our resources especially when it comes to health. She also always wishes us a good day/weekend and that is awesome.''
\end{itemize}

With my application, I include a curriculum vitae, a teaching statement, a description of my professional expertise, and a list three professional references. Please do not hesitate to contact me if there is anything else I can provide, or with any questions, and thank you for your consideration. 

\vspace{.1 in}
\\
Sincerely,
\\ \vspace{.1 in}

Juliette Bruce

% \encl{brownletter.cls}

% \ps{Please see the enclosed file.}

% \cc{J. Kirschenbaum}

\end{letter}

\end{document}