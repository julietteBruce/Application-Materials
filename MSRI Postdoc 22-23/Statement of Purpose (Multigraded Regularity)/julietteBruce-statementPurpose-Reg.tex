\documentclass[11pt,reqno]{amsart}
\usepackage{amsfonts,amsmath,amssymb,amsbsy,amstext,amsthm,mathtools}
\usepackage{accents,color,enumerate,enumitem,float,fullpage,verbatim}

\usepackage{url}
\usepackage[colorlinks=true,hyperindex, linkcolor=magenta, pagebackref=false, citecolor=cyan]{hyperref}
\usepackage[alphabetic,lite]{amsrefs} 

%\usepackage{parskip}

%\usepackage{eucal,bm,kpfonts,mathbbol}
\usepackage[margin=1.02in,includeheadfoot]{geometry}

\usepackage{tikz,tikz-cd}	
\usetikzlibrary{positioning, matrix, shapes}         								    				
\usetikzlibrary{arrows,calc,matrix}

\usepackage{lscape}

\usepackage{microtype}


\usepackage{titlesec}		
\setcounter{secnumdepth}{4}						     					% Allows one to use nice section titles
\titleformat{\section}[block]{\scshape\bfseries\filcenter}{\thesection.}{1em}{}		% Creates section titles
\titleformat{\subsection}[runin]{\scshape\bfseries}{\thesubsection}{1em}{}			% Creates subsection titles
\titleformat{\subsubsection}[runin]{\scshape\bfseries}{\thesubsubsection}{1em}{}			% Creates subsection titles

\usepackage[titles]{tocloft}								     					% Creates table of fancy contents
\setcounter{tocdepth}{4}
\renewcommand{\contentsname}{}	     					% Renames and centers title of ToC

\usepackage{multirow}
\usepackage{array}
\usepackage{booktabs}
\newcolumntype{M}[1]{>{\centering\arraybackslash}m{#1}}
\newcolumntype{N}{@{}m{0pt}@{}}
\usepackage{diagbox}
\usepackage{cancel}

\newtheorem{lemma}{Lemma}[section]
\newtheorem{theorem}[lemma]{Theorem}
\newtheorem{goalTheorem}[lemma]{Goal Theorem}
\newtheorem{prop}[lemma]{Proposition}
\newtheorem{cor}[lemma]{Corollary}
\newtheorem{conj}[lemma]{Conjecture}
\newtheorem{claim}[lemma]{Claim}
\newtheorem{defn}[lemma]{Definition} 
\newtheorem{notation}[lemma]{Notation} 
\newtheorem{exercise}[lemma]{Exercise}
\newtheorem{question}[lemma]{Question}
\newtheorem*{assumption}{Assumption}
\newtheorem{principle}[lemma]{Principle}
\newtheorem{heuristic}[lemma]{Heuristic}

\newtheorem{theoremalpha}{Theorem}
\newtheorem{corollaryalpha}[theoremalpha]{Corollary}
\renewcommand{\thetheoremalpha}{\Alph{theoremalpha}}

\theoremstyle{remark}
\newtheorem{remark}[lemma]{Remark}
\newtheorem{example}[lemma]{Example}
\newtheorem{cexample}[lemma]{Counterexample}

% Commands
\newcommand{\initial}{\operatorname{in}}
\newcommand{\NF}{\operatorname{NF}}
\newcommand{\HF}{\operatorname{HF}}
\newcommand{\Hilb}{\operatorname{Hilb}}
\newcommand{\depth}{\operatorname{depth}}
\newcommand{\reg}{\operatorname{reg}}
\newcommand{\Span}{\operatorname{span}}
\newcommand{\img}{\operatorname{img}}
\newcommand{\inn}{\operatorname{in}}

\newcommand{\length}{\operatorname{length}}
\newcommand{\coker}{\operatorname{coker}}
\newcommand{\adeg}{\operatorname{adeg}}
\newcommand{\pdim}{\operatorname{pdim}}
\newcommand{\Spec}{\operatorname{Spec}}
\newcommand{\Ext}{\operatorname{Ext}}
\newcommand{\Tor}{\operatorname{Tor}}
\newcommand{\LT}{\operatorname{LT}}
\newcommand{\im}{\operatorname{im}}
\newcommand{\NS}{\operatorname{NS}}
\newcommand{\Frac}{\operatorname{Frac}}
\newcommand{\Khar}{\operatorname{char}}
\newcommand{\Proj}{\operatorname{Proj}}
\newcommand{\id}{\operatorname{id}}
\newcommand{\Div}{\operatorname{Div}}
\newcommand{\Kl}{\operatorname{Cl}}
\newcommand{\tr}{\operatorname{tr}}
\newcommand{\Tr}{\operatorname{Tr}}
\newcommand{\Supp}{\operatorname{Supp}}
\newcommand{\ann}{\operatorname{ann}}
\newcommand{\Gal}{\operatorname{Gal}}
\newcommand{\Pic}{\operatorname{Pic}}
\newcommand{\QQbar}{{\overline{\mathbb Q}}}
\newcommand{\Br}{\operatorname{Br}}
\newcommand{\Bl}{\operatorname{Bl}}
\newcommand{\Kox}{\operatorname{Cox}}
\newcommand{\conv}{\operatorname{conv}}
\newcommand{\getsr}{\operatorname{Tor}}
\newcommand{\diam}{\operatorname{diam}}
\newcommand{\Hom}{\operatorname{Hom}} %done
\newcommand{\sheafHom}{\mathcal{H}om}
\newcommand{\Gr}{\operatorname{Gr}}
\newcommand{\rank}{\operatorname{rank}} 
\newcommand{\codim}{\operatorname{codim}}
\newcommand{\Sym}{\operatorname{Sym}} %done
\newcommand{\GL}{{GL}}
\newcommand{\Prob}{\operatorname{Prob}}
\newcommand{\Density}{\operatorname{Density}}
\newcommand{\Syz}{\operatorname{Syz}}
\newcommand{\pd}{\operatorname{pd}}
\newcommand{\supp}{\operatorname{supp}}
\newcommand{\cone}{\operatorname{\textbf{cone}}}
\newcommand{\Res}{\operatorname{Res}}
\newcommand{\HS}{\operatorname{HS}}
\newcommand{\Cl}{\operatorname{Cl}}
\newcommand{\oO}{\operatorname{O}}

\newcommand{\defi}[1]{\textsf{#1}} % for defined terms

\newcommand{\remd}{\operatorname{remd}}
\newcommand{\colim}{\operatorname{colim}}
\newcommand{\trideg}{\operatorname{tri.deg}}
\newcommand{\indeg}{\operatorname{index.deg}}
\newcommand{\moddeg}{\operatorname{mod.deg}}
\newcommand{\Desc}{\operatorname{Desc}}
\newcommand{\inter}{\operatorname{int}}
\newcommand{\Nef}{\operatorname{Nef}}
\newcommand{\Jac}{\operatorname{Jac}}
\newcommand{\Cox}{\operatorname{Cox}}

\newcommand{\doot}{\bullet}

\newcommand{\Alt}{\bigwedge\nolimits}
\newcommand{\Set}{\text{\bf Set}}										% Category of Sets
\newcommand{\Sch}{\text{\bf Sch}}										% Category of Abelian Groups
\newcommand{\Mod}[1]{\ (\mathrm{mod}\ #1)}




%%%%%%%%%%%%%%%%%%%%%%%%%%%%%% Letters  %%%%%%%%%%%%%%%%%%%%%%%%%%%%%%%%%%%%%%%%%%%%
%%%%%%%%%%%%%%%%%%%%%%%%%%%%%%%%%%%%%%%%%%%%%%%%%%%%%%%%%%%%%%%%%%%%%%%%%%%%%%
\newcommand{\ff}{\mathbf f}
\newcommand{\qq}{\mathbf q}
\newcommand{\kk}{\mathbf k}
\renewcommand{\aa}{\mathbf a}
\newcommand{\bb}{\mathbf b}
\newcommand{\cc}{\mathbf c}
\newcommand{\dd}{\mathbf d}
\newcommand{\ee}{\mathbf e}
\newcommand{\vv}{\mathbf v}
\newcommand{\ww}{\mathbf w}
\newcommand{\xx}{\mathbf x}
\newcommand{\yy}{\mathbf y}
\newcommand{\rr}{\mathbf r}
\newcommand{\ii}{\mathbf i}
\newcommand{\nn}{\mathbf n}
\newcommand{\pp}{\mathbf p}
\newcommand{\mm}{\mathbf m}
\newcommand{\fF}{\mathbf F}
\newcommand{\gG}{\mathbf G}
\newcommand{\eE}{\mathbf E}
\newcommand{\qQ}{\mathbf Q}
\newcommand{\tT}{\mathbf T}
\renewcommand{\tt}{\mathbf t}
\newcommand{\one}{\mathbf 1}
\newcommand{\zero}{\mathbf 0}

\renewcommand{\H}{\operatorname{H}}
\newcommand{\OO}{\operatorname{O}}
\newcommand{\oo}{\operatorname{o}}


%%%% Caligraphic Fonts - i.e. ????. %%%%%
\newcommand{\cA}{\mathcal{A}}
\newcommand{\cB}{\mathcal{B}}
\newcommand{\cC}{\mathcal{C}}
\newcommand{\cD}{\mathcal{D}}
\newcommand{\cE}{\mathcal{E}}
\newcommand{\cF}{\mathcal{F}}
\newcommand{\cG}{\mathcal{G}}
\newcommand{\cH}{\mathcal{H}} 
\newcommand{\cI}{\mathcal{I}}
\newcommand{\cJ}{\mathcal{J}}
\newcommand{\cK}{\mathcal{K}}
\newcommand{\cL}{\mathcal{L}}
\newcommand{\cM}{\mathcal{M}}
\newcommand{\cN}{\mathcal{N}}
\renewcommand{\O}{\mathcal{O}}
\newcommand{\cP}{\mathcal{P}}
\newcommand{\cQ}{\mathcal{Q}}
\newcommand{\cR}{\mathcal{R}}
\newcommand{\cS}{\mathcal{S}}
\newcommand{\cT}{\mathcal{T}}
\newcommand{\U}{\mathcal{U}} 		% Notice this is different
\newcommand{\cV}{\mathcal{V}}
\newcommand{\cW}{\mathcal{W}}
\newcommand{\cX}{\mathcal{X}}
\newcommand{\cY}{\mathcal{Y}}
\newcommand{\cZ}{\mathcal{Z}}

%%%% Blackboard Fonts - i.e. Real Numbers, Integers, etc. %%%%%
\newcommand{\A}{\mathbb{A}}
\newcommand{\B}{\mathbb{B}}
\newcommand{\C}{\mathbb{C}}
\newcommand{\D}{\mathbb{D}}
\newcommand{\E}{\mathbb{E}}
\newcommand{\F}{\mathbb{F}}
\newcommand{\G}{\mathbb{G}}
\newcommand{\I}{\mathbb{I}}
\newcommand{\J}{\mathbb{J}}
\newcommand{\K}{\mathbb{K}}
\renewcommand{\L}{\mathbb{L}}
\newcommand{\M}{\mathbb{M}}
\newcommand{\N}{\mathbb{N}}
\newcommand{\bO}{\mathbb{O}}		% Notice this is \bO
\renewcommand{\P}{\mathbb{P}}
\newcommand{\Q}{\mathbb{Q}}
\newcommand{\R}{\mathbb{R}}
\newcommand{\T}{\mathbb{T}}
\newcommand{\bU}{\mathbb{U}}		% Notice this is \bU
\newcommand{\V}{\mathbb{V}}
\newcommand{\W}{\mathbb{W}}
\newcommand{\X}{\mathbb{X}}
\newcommand{\Y}{\mathbb{Y}}
\newcommand{\Z}{\mathbb{Z}}

 %%%% Sarif Fonts - i.e. ???? %%%%%
\newcommand{\sA}{\mathsf{A}}
\newcommand{\sB}{\mathsf{B}}
\newcommand{\sC}{\mathsf{C}}
\newcommand{\sD}{\mathsf{D}}
\newcommand{\sE}{\mathsf{E}}
\newcommand{\sF}{\mathsf{F}}
\newcommand{\sG}{\mathsf{G}}
\newcommand{\sH}{\mathsf{H}} 
\newcommand{\sI}{\mathsf{I}}
\newcommand{\sJ}{\mathsf{J}}
\newcommand{\sK}{\mathsf{K}}
\newcommand{\sL}{\mathsf{L}}
\newcommand{\sM}{\mathsf{M}}
\newcommand{\sN}{\mathsf{N}}
\newcommand{\sO}{\mathsf{O}}
\newcommand{\sP}{\mathsf{P}}
\newcommand{\sQ}{\mathsf{Q}}
\newcommand{\sR}{\mathsf{R}}
\newcommand{\sS}{\mathsf{S}}
\newcommand{\sT}{\mathsf{T}}
\newcommand{\sU}{\mathsf{U}} 
\newcommand{\sV}{\mathsf{V}}
\newcommand{\sW}{\mathsf{W}}
\newcommand{\sX}{\mathsf{X}}
\newcommand{\sY}{\mathsf{Y}}
\newcommand{\sZ}{\mathsf{Z}}
 
 %%%% Fraktur Fonts - i.e. maximal ideals, prime ideals, etc. %%%%%
\newcommand{\cl}{\mathfrak{cl}}
\newcommand{\g}{\mathfrak{g}}
\newcommand{\h}{\mathfrak{h}}
\newcommand{\m}{\mathfrak{m}}
\newcommand{\n}{\mathfrak{n}}
\newcommand{\p}{\mathfrak{p}}
\newcommand{\q}{\mathfrak{q}}
\renewcommand{\r}{\mathfrak{r}}


\renewcommand{\cite}[1]{{}}

\newcommand{\juliette}[1]{{\color{red} \sf $\spadesuit\spadesuit\spadesuit$ Juliette: [#1]}}


\title{Juliette Bruce's Research Statement}

%\author{Juliette Bruce}
%\address{Department of Mathematics, University of Wisconsin, Madison, WI}
%\email{\href{mailto:juliette.bruce@math.wisc.edu}{juliette.bruce@math.wisc.edu}}
%\urladdr{\url{http://math.wisc.edu/~juliettebruce/}}

%\thanks{The author was partially supported by the NSF GRFP under Grant No. DGE-1256259 and NSF grant DMS-1502553.}

%\subjclass[2010]{13D02, 14M25}

\begin{document} 

%\maketitle
%\begingroup  
%  \centering
%  \large\scshape\bfseries Juliette Bruce's Statement of Purpose\\[1em]
%\endgroup

%\tableofcontents

\setcounter{section}{0}

A projective variety $Z\subset \P^n$ is defined by a homogenous ideal $I_{Z}$ in a standard $\Z$-graded polynomial ring. The minimal graded free resolution of $I_{Z}$ often captures not only interesting algebraic information about the defining equations of $Z$, but also information about the geometry of $Z$. When $\P^{n}$ is replaced by another toric variety $X$, the ways minimal multigraded free resolutions over the Cox ring of $X$ capture algebraic and geometric information is often quite mysterious. Much of my recent work has focused on developing tools in multigraded homological algebra to better understand the geometry of toric varieties. As a taste of this, I will focus on my work studying the connection between minimal resolutions and multigraded Castelnuovo–Mumford regularity.

%Much of my recent research has focused on understanding how they ways minimal graded free resolutions capture information about subvarietes $\P^{n}$ is replaced by a different toric variety.  

%Given a projective variety $X$ embedded in $\P^n$, we associate to $X$  the ideal of homogeneous polynomials vanishing on $X$.  $S=\C[x_0,\ldots,x_n]$ and $I_X$ is the ideal of homogeneous polynomials vanishing on $X$. As $S_X$ is a graded $S$-module we may consider its minimal graded free resolution, which is often closely related to both the extrinsic and intrinsic geometry of $X$.


%Just as the connection between subvarieties of projective space and graded modules over the standard graded polynomial ring inspired substantial work in commutative algebra, recently there has be substantial work in developing homological tools to study multigraded modules over multigraded polynomial rings to better understand subvarieties of toric varieties. Much of my recent work fits into this trend. As a taste of some of this work I will focus on recent work studying the homological properties of multigraded Castelnuovo–Mumford regularity.


Castelnuovo–Mumford regularity is a measure of geometric complexity that has proven extremely useful in studying subvarieties of projective space. As originally introduced by Mumford, the regularity of a coherent sheaf $\cF$ on $\P^{n}$ is given in terms of certain cohomological vanishing conditions. Roughly, one can think of the regularity of $\cF$ as being an effective bound for Serre vanishing. Mumford was interested in such a measure as it plays a key role in constructing Hilbert schemes. However, as the following result of Eisenbud and Goto shows it is also closely connected to minimal graded free resolutions over the standard graded polynomial ring $S=\C[x_{0},\ldots,x_{n}]$. 

\begin{theoremalpha}[Eisenbud-Goto]\cite{eisenbudGoto84}\label{thm:eisenbud-goto}
Let $\cF$ be a coherent sheaf on $\P^{n}$ and $M=\bigoplus_{e\in\Z} H^0(\P^{n},\cF(e))$ the corresponding section ring. The following are equivalent: (1) $M$ is $d$-regular; (2) $\Tor_{p}^{S}(M,\C)_{q}=0$ for all $p\geq0$ and $q>d+i$; (3) $M_{\geq d}$ has a linear minimal graded free resolution. 
%\begin{enumerate}
%\item $M$ is $d$-regular;
%\item $\Tor_{p}^{S}(M,\C)_{q}=0$ for all $p\geq0$ and $q>d+i$;
%\item $M_{\geq d}$ has a linear resolution. 
%\end{enumerate}
\end{theoremalpha}

%the connection between coherent sheaves on projective space and graded modules over the standard $\Z$-graded polynomial ring $S=\C[x_{0},\ldots,x_{n}]$ has meant that Castelnuovo–Mumford regularity has proven a fruitful area of active research in commutative algebra. 

%In recent years an active area of research has been in developing the connections between toric geometry and commutative algebra over multigraded polynomial rings. This includes work by 

Maclagan and Smith introduced a generalization of Castelnuovo–Mumford regularity to coherent sheaves on other toric varieties. Following Mumford, they define the \emph{multigraded Castelnuovo–Mumford regularity} of a coherent sheaf $\cF$ on a toric variety $X$ in terms of certain cohomological vanishing. For example, if $X= \P^{n_1}\times \P^{n_2}\times \cdots \times \P^{n_r}$ and $\dd\in\Z^{r}$, then $\cF$ is $\dd$-regular if and only if
\[
H^i\left(\P^{\nn}, \cF(\ee)\right) =0 \quad \quad \quad \text{for all $\ee\in L_{i}(\dd)=\bigcup_{\substack{\vv \in \N, |\vv| = i}} (\dd-\vv)+\N^{r}$}.
\]



%In order to give a taste of some of my recent work in these directions. Let me focus on the following theorem of Eisenbud and Goto, which states that the Castelnuovo–Mumford regularity of a module over the standard graded polynomial ring $S$ can be characterized solely in terms of homological properties of minimal graded free resolutions. 


%As introduced by Mumford, the Castelnuovo–Mumford regularity of a coherent sheaf $\cF$ on $\P^{n}$ is a measure of the complexity of $\cF$ given in terms of the vanishing of certain cohomology groups. Such a measure can be easily extended to graded modules over the standard graded polynomial ring $S=\C[x_{0},\ldots,x_{n}]$ where $\deg(x_i)=1$ by requiring the analogous vanishing conditions for local cohomology. Mumford was interested  such a measure as it plays a key role in constructing Hilbert schemes. However, Castelnuovo–Mumford regularity has proven to be closely connected to 
%Eisenbud and Goto showed that regularity would be characterized solely in terms of homological properties of minimal graded free resolutions. 


%In particular, if $M$ is a graded $S$-module we say that $M$ is $d$-regular if and only if \juliette{ADD}. 

%Mumford was interested  such a measure as it plays a key role in constructing Hilbert and Quot schemes. However, Castelnuovo–Mumford regularity has proven to be closely connected to 
%Eisenbud and Goto showed that regularity is also closely connected to interesting homological properties.

%\begin{theoremalpha}[Eisenbud-Goto]\cite{eisenbudGoto84}\label{thm:eisenbud-goto}
%Let $\cF$ be a coherent sheaf on $\P^{n}$ and $M=\bigoplus_{e\in\Z} H^0(\P^{n},\cF(e))$ the corresponding section ring. The following are equivalent: (1) $M$ is $d$-regular; (2) $\Tor_{p}^{S}(M,\C)_{q}=0$ for all $p\geq0$ and $q>d+i$; (3) $M_{\geq d}$ has a linear minimal graded free resolution. 
%%\begin{enumerate}
%%\item $M$ is $d$-regular;
%%\item $\Tor_{p}^{S}(M,\C)_{q}=0$ for all $p\geq0$ and $q>d+i$;
%%\item $M_{\geq d}$ has a linear resolution. 
%%\end{enumerate}
%\end{theoremalpha}

%This close connection between Castelnuovo–Mumford regularity and minimal graded free resolutions has lead Castelnuovo–Mumford regularity to be an active area of research in commutative algebra. 

%For example, work of many people showed that the Castelnuovo–Mumford regularity of powers of ideals displays surprisingly predictable asymptotic behavoir. In particular, Cutkosky, Herzog, Trung and independently Kodiyalam proved that if $I\subset S$ is a homogenous ideal then there exist constants $d,e\in\Z$ such that $\reg\!\left(I^t\right) = dt+e$ for $t\gg0$.

My collaborators and I have worked to understand how multigraded Castelnuovo–Mumford regularity can be characterized in terms of minimal multigraded free resolutions. Continuing with $X= \P^{n_1}\times \cdots \times \P^{n_r}$, let $S=\C[x_{i,j} \; |\; 1\leq i \leq r, 0\leq j \leq n_{i}]$ be the Cox ring of $X$ with the $\Z^{r}$-grading given by $\deg x_{i,j} = \ee_{i} \in \Z^{r}$, where $\ee_{i}$ is the $i$-th standard basis vector. Our examples show that generalizing Theorem~\ref{thm:eisenbud-goto} to mulitgraded regularity is subtle. For example, the multigraded Betti numbers of a module do not determine its multigraded Castelnuovo--Mumford regularity, and $M$ being $\dd$-regular does not imply that the truncation $M_{\geq\dd}$ has a linear free resolution. (Here $M_{\geq \dd}$ is the submodule of $M$ generated by elements whose degree is coordinate-wise $\geq \dd$.) However, we prove that multigraded regularity can be characterized by the truncation $M_{\geq \dd}$ having a certain type of free resolution. We say that a complex $F_{\bullet}$ of $\Z^{r}$-graded free $S$-modules is $\dd$-quasilinear if and only if $F_{0}$ is generated in degree $\dd$, and each minimal generator $f$ of $F_{i}$ has $-\deg(f)\in L_{i-1}(\dd-\one)$.

%Maclagan and Smith generalized Castelnuovo--Mumford regularity to this setting in terms of certain cohomology vanishing. Fixing some notation given $\dd\in \Z^{r}$ and $i\in \Z_{\geq0}$ we let:
%\[
%L_{i}(\dd)\coloneqq \bigcup_{\substack{\vv \in \N \\ |\vv| = i}} (\dd-\vv)+\N^{r}.
%\]
%Note when $r=2$ the region $L_{i}(\dd)$ looks like a staircase with $(i+1)$-corners. Roughly speaking we define regularity by requiring the $i$-th cohomology of certain twists of $\cF$ to vanish on $L_{i}$. 

%\begin{center}
%\begin{figure}[H]
%\newcommand{\makegrid}{
%  \path[use as bounding box] (-3.45,-3.25) rectangle (5.45,5.25);
%  \foreach \x in {-4,...,4}
%  \foreach \y in {-4,...,4}
%    { \fill[Gray,fill=gray] (\x,\y) circle (1.5pt); }
%  \draw[-,  semithick] (-4,0)--(4,0);
%  \draw[-,  semithick] (0,-4)--(0,4);
%}
%%%%%%%%%%%%%%%%%%%%%%%%%%%%%%%%%%%%%
%%%%%%%%%%%%%%%%%%%%%%%%%%%%%%%%%%%%%
%%%%%%%%%%%%%%%%%%%%%%%%%%%%%%%%%%%%%
%\begin{tikzpicture}[scale=.3]
%  \path[fill=Gray!45] (-1,4)--(-1,0)--(0,0)--(0,-1)--(4,-1)--(4,4)--(-1,4);
%  \makegrid
%  \draw[->, ultra thick] (-1,0)--(-1,4);
%  \draw[-, cap=round,ultra thick] (-1,0)--(0,0)--(0,-1);
%    \draw[->, ultra thick] (0,-1)--(4,-1);
%  %\fill[Gray,fill=Gray] (-1,0) circle (6pt);
% % \fill[Gray,fill=Gray] (0,-1) circle (6pt);
%  \fill[Gray,fill=Black] (0,0) circle (6pt);
%\end{tikzpicture}\quad\;
%%%%%%%%%%%%%%%%%%%%%%%%%%%%%%%%%%%%%
%%%%%%%%%%%%%%%%%%%%%%%%%%%%%%%%%%%%%
%%%%%%%%%%%%%%%%%%%%%%%%%%%%%%%%%%%%%
%\begin{tikzpicture}[scale=.3]
%  \path[fill=Gray!45] (-2,4)--(-2,0)--(-1,0)--(-1,-1)--(0,-1)--(0,-2)--(4,-2)--(4,4)--(-2,4);
%  \makegrid
%  \draw[->, ultra thick] (-2,0)--(-2,4);
%  \draw[-, cap=round,ultra thick] (-2,0)--(-1,0)--(-1,-1)--(0,-1)--(0,-2);
%    \draw[->, ultra thick] (0,-2)--(4,-2);
%%  \fill[Gray,fill=Gray] (-2,0) circle (6pt);
% % \fill[Gray,fill=Gray] (-1,-1) circle (6pt);
% % \fill[Gray,fill=Gray] (0,-2) circle (6pt);
%  \fill[Gray,fill=Black] (0,0) circle (6pt);
%\end{tikzpicture}\quad\;
%\begin{tikzpicture}[scale=.3]
%  \path[fill=Gray!45] (-3,4)--(-3,0)--(-2,0)--(-3,0)--(-2,0)--(-2,-1)--(-1,-1)--(-1,-2)--(0,-2)--(0,-3)--(4,-3)--(4,4)--(-2,4);
%  \makegrid
%  \draw[->, ultra thick] (-3,0)--(-3,4);
%  \draw[-, cap=round,ultra thick] (-3,0)--(-2,0)--(-2,-1)--(-1,-1)--(-1,-2)--(0,-2)--(0,-3);
%      \draw[->, ultra thick] (0,-3)--(4,-3);
%%  \fill[Gray,fill=Gray] (-2,0) circle (6pt);
% % \fill[Gray,fill=Gray] (-1,-1) circle (6pt);
% % \fill[Gray,fill=Gray] (0,-2) circle (6pt);
%  \fill[Gray,fill=Black] (0,0) circle (6pt);
%\end{tikzpicture}
%%
%%\caption{Letting $r=2$ the regions $L_{1}(0,0)$, $L_{2}(0,0)$, and $L_{3}(0,0)$.}
%\end{figure}
%\end{center}

%\begin{defn}\label{def:mg-reg}
%A coherent sheaf $\cF$ on $\P^{\nn}$ is $\dd$-regular if and only if
%\[
%H^i\left(\P^{\nn}, \cF(\ee)\right) =0 \quad \quad \quad \text{for all $\ee\in L_{i}(\dd)=\bigcup_{\substack{\vv \in \N, |\vv| = i}} (\dd-\vv)+\N^{r}$}.
%\]
%The multigraded Castelnuovo--Mumford regularity of $\cF$ is then the set: 
%\[
%\reg(\cF) \coloneqq \left \{ \dd\in \Z^{r} \;\; | \;\; \text{$\cF$ is $\dd$-regular}\right\}\subset \Z^{r}.
%\]
%\end{defn}

%Even for relatively simple examples the multigraded Castelnuovo--Mumford regularity does not necessarily have a unique minimal element. That said $\reg(\cF)$ does have the structure of a module over the semi-group $\Nef(\P^{\nn})\cong\N^{r}$, i.e. if $\dd \in \reg(\cF)$ then $\dd+\ee\in \reg(\cF)$ for all $\ee\in \N^{r}$. 

%\begin{center}
%\begin{figure}\label{fig:example-of-reg}
%\newcommand{\makegrid}{
%  \path[use as bounding box] (-3.45,-3.25) rectangle (5.45,5.25);
%  \foreach \x in {-1,...,5}
%  \foreach \y in {-1,...,5}
%    { \fill[Gray,fill=gray] (\x,\y) circle (1.5pt); }
%  \draw[-,  semithick] (-1,0)--(5,0);
%  \draw[-,  semithick] (0,-1)--(0,5);
%}
%%%%%%%%%%%%%%%%%%%%%%%%%%%%%%%%%%%%%
%%%%%%%%%%%%%%%%%%%%%%%%%%%%%%%%%%%%%
%%%%%%%%%%%%%%%%%%%%%%%%%%%%%%%%%%%%%
%\begin{tikzpicture}[scale=.35]
%  \path[fill=Gray!45] (5,0)--(2,0)--(2,1)--(1,1)--(1,2)--(0,2)--(0,5)--(5,5);
%  \makegrid
%  \draw[->, ultra thick] (2,0)--(5,0);
%  \draw[-, cap=round,ultra thick] (2,0)--(2,1)--(1,1)--(1,2)--(0,2);
%    \draw[->, ultra thick] (0,2)--(0,5);
%  %\fill[Gray,fill=Gray] (-1,0) circle (6pt);
% % \fill[Gray,fill=Gray] (0,-1) circle (6pt);
%\end{tikzpicture}
%\caption{The multigraded Castelnuovo--Mumford regularity of $\O_{X}$ where $X\subset \P^{1}\times \P^{1}$ is the subscheme consisting of three distinct points $([1:1],[1:4])$, $([1:2],[1:5])$, and $([1:3],[1:6])$.}
%\end{figure}
%\end{center}



%\begin{defn}
%A complex $F_{\bullet}$ of $\Z^{r}$-graded free $S$-modules is $\dd$-quasilinear if and only if $F_{0}$ is generated in degree $\dd$ and each twist of $F_{i}$ is contained in $L_{i-1}(\dd-\one)$.
%%\begin{enumerate}
%%\item We say that  $F_{\bullet}$ is $\dd$-linear if and only if $F_{0}$ is generated in degree $\dd$ and each twist of $F_{i}$ is contained in $L_{i}(\dd)$. 
%%\item We say that $F_{\bullet}$ is $\dd$-quasilinear if and only if $F_{0}$ is generated in degree $\dd$ and each twist of $F_{i}$ is contained in $L_{i-1}(\dd-\one)$. 
%%\end{enumerate}
%\end{defn}

%In order to see the difference between linear and quasilinear resolutions we note that on a product of projective spaces the irrelevant ideal generally will have a quasilinear resolution, not a linear resolution.  For example, if we consider $\P^{1}\times \P^{2}$ so that $S=\K[x_{0},x_{1},y_{0},y_{1},y_{2}]$ and $B=\langle x_{0},x_{1}\rangle\cap\langle y_{0},y_{1},y_{2}\rangle$ then the minimal graded free resolution of $S/B$ is: 
%	\[\begin{tikzcd}[column sep=1.75em]
%	S & \lar S(-1,-1)^6 & \lar
%	  \begin{matrix}
%	    S(-1,-2)^6\\[-3pt]
%	    \oplus \\[-3pt]
%	    S(-2,-1)^3
%	  \end{matrix}
%	  &
%	  \lar
%	  \begin{matrix}
%	    S(-1,-3)^2\\[-3pt]
%	    \oplus \\[-3pt]
%	    S(-2,-2)^3
%	  \end{matrix}
%	  &\lar
%	  S(-2,-3)
%	  & \lar 0.
%	\end{tikzcd}\]
%In particular, we see that the minimal graded free resolution $S/B$ is not $(0,0)$-linear since $(-1,-1) \not\in L_1(0,0)$, however, it is $(0,0)$-quasilinear. 
%
%It is not the case that $M$ being $\dd$-regular implies $M_{\geq \dd}$ has a linear resolution \cite[Example 4.2]{bruce21}, however, we can characterize being $\dd$-regular in terms of $M_{\geq \dd}$ having a quasilinear resolution. 

\begin{theoremalpha}\label{thm:mgreg-main}
Let $M$ be a finitely generated $\Z^{r}$-graded $S$-module with $H^{0}_{B}(M)=0$:
\[
\text{$M$ is $\dd$-regular} \iff  \text{$M_{\geq\dd}$ has a $\dd$-quasilinear resolution}.
\]
\end{theoremalpha}

\noindent The proof of Theorem~\ref{thm:mgreg-main} is based in part on a \v{C}ech--Koszul spectral sequence argument that relates the multigraded Betti numbers of $M_{\geq\dd}$ to the Fourier--Mukai transform of $\widetilde{M}$. %with Beilinson's resolution of the diagonal as the kernel. 

  Theorem~\ref{thm:mgreg-main}, together with other work I have done on the multigraded regularity of powers of ideals -- as well as the work of many other people --  suggests that further developing tools in multigraded homological algebra will both highlight new algebraic phenomena as well as prove fruitful for understanding some of the rich geometry of toric varieties. Currently, my collaborators and I are exploring other ways minimal multigraded free resolutions capture geometric and algebraic properties of toric varieties including Bayer-Stillman type results for multigraded regularity, and generalizations of Green's conjecture for canonical stacky curves. 


%Berkesch, Brown, Chardin, Eisenbud, Erman, Schreyer, and Smith
%The proof of Theorem~\ref{thm:mgreg-main} is based in part on a \v{C}ech--Koszul spectral sequence that relates the Betti numbers of $M_{\geq\dd}$ to the Fourier--Mukai transform of $\widetilde{M}$ with Beilinson's resolution of the diagonal as the kernel.  Precisely, if $M$ is $\dd$-regular and $H_B^0(M)=0$ we prove the that
%\begin{align*}\label{eq:magic-equality}
%  \dim_{\C}\Tor^S_j(M_{\geq\dd}, \C)_\aa = h^{|\aa|-j}\big(\P^{\nn}, \widetilde{M}\otimes\O_{\P^{\nn}}^\aa(\aa)\big) \quad \text{for } |\aa|\geq j\geq 0,
%\end{align*}
%where the $\O_{\P^{\nn}}^\aa$ are cotangent sheaves on $\P^\nn$. The result then follows from showing that $M$ being $\dd$-regular is equivalent to certain vanishings of the right-hand side above. 

%\begin{theoremalpha}
%  There exists a degree $\aa\in\Z^{r}$, depending only on $I$, such that for each integer $t>0$ and each pair of degrees $\qq_1,\qq_2\in\Z^{r}$ satisfying $\qq_1\geq\deg f_i\geq\qq_2$ for all generators $f_i$ of $I$, we have
%	\[ t\qq_1+\aa+\N^r \subseteq \reg\!\left(I^t\right) \subseteq t\qq_2+\N^r. \]
%\end{theoremalpha}
%

\newpage 

\noindent \textbf{Relevance of Visit.} Being a postdoctoral fellow at MSRI would be extremely beneficial to my career goals as it would provide me with a stimulating mathematical environment, and the opportunity to engage and work with many leading researchers in commutative algebra. My research interests align very well with the topics of the program, and multiple projects I am working on directly touch upon many of the topics included in the program description. For example, one project I am working on concerns better understanding multigraded Castelnuovo--Mumford regularity, and in a separate project, I am exploring ways Green's conjecture can be generalized to canonical stacky curves. Being a postdoc at MSRI would likely prove valuable toward progressing both of these projects, as well as a fantastic opportunity to begin new collaborations with other participants. Further, given the ways my research interests align with the program, I feel I would be able to find numerous ways to contribute to the program, for example, I would love to help organize a seminar and social activities for myself and the other postdocs.% including by discussing my own work, building new collaborations, and NEDEDED. 

I am specifically applying for a postdoctoral position since my current position would make it extremely difficult for me to participate in the semester otherwise (i.e., as a research member or occasional visitor). Overall, being a postdoctoral fellow for this semester program in commutative algebra presents an amazing opportunity to connect and build relationships within the research area that has long felt like my research community and home. Such connections would significantly advance my goal of being a math professor at a research university.
\\

%Finally, a particular aspect I have enjoyed about working in commutative algebra has been the strong sense of community I have found within the field. Being a postdoctoral fellow for this semester program thus presents an amazing opportunity to connect and build relationships within the research area that has long felt like my research home.


%  Working with David Eisenbud would provide the opportunity to learn from one of the foremost experts in algebraic geometry and commutative algebra. Eisenbud has done foundational work concerning the syzygies of varieties, liaison theory, and the geometry of curves. Working with him would likely prove valuable towards the first two projects I plan on working on while visiting MSRI. In short, working with Eisenbud would significantly advance my goals of being a math professor at a research university.

\noindent \textbf{Building Mathematical Community.} I have worked hard to promote diversity, inclusivity, and justice in the mathematical community by mentoring students, supporting women and LGBTQ+ people in mathematics, and organizing conferences and outreach programs. %For brevity, below I will discuss a few of my more recent efforts in these directions to build community.

I have put significant effort into mentoring and working with students, especially those from underrepresented groups. During 2021 I organized a virtual summer research program for 6 undergraduates. In 2022 I advised an undergraduate student on a research project in commutative algebra. This student is now applying to graduate schools in math. I began research projects with multiple graduate students, in which I played a substantial mentoring role. In 2022, I led a reading course with a first-year graduate woman in commutative algebra.

%undergraduates
%I advised two summer research projects for undergraduate students. The first of these projects ran virtually during Summer 2021 when 6 undergraduates. In Summer 2022 I advised an undergraduate student on a research project related to my work on syzygies discussed on the previous page. This student is now applying to graduate schools in math. As a postdoc, I began research projects with multiple graduate students (a majority of whom identify with a generally underrepresented group), in which I played a substantial mentoring and guiding role. These projects have resulted in two pre-prints, with additional projects still ongoing. Throughout the Spring and Summer of 2022, I did a reading course with a first-year graduate woman on algebraic geometry.

Since 2020 I have organized an annual conference promoting the work of transgender and non-binary mathematicians, which regularly has over 50 participants. Highlighting the importance of such conferences one participant said, ``I've been really considering leaving mathematics.  [Trans Math Day 2020] reminded me why I'm here and why I want to stay. ... If a conference like this had been around for me five years ago, my life would have been a lot better.'' For the last two years, I have been a board member for \textit{Spectra: The Association for LGBTQ+ Mathematicians}, and currently, I am the inaugural president. As a board member, I led the creation and adoption of bylaws, the creation of an invited lecture at the Joint Mathematics Meetings, and a fundraising campaign that has raised over \$20,000 to support LGBTQ+ students and mathematicians. %I am currently the inaugural president of Spectra, and as of right now we have over 500 people on our mailing/membership lists.  


In response to the COVID-19 pandemic, I worked to find ways for these online activities to reach those often at the periphery. During 2020, I helped with Ravi Vakil's \textit{Algebraic Geometry in the Time of Covid} project; an online open-access course in algebraic geometry binging together $\sim2,000$ participants. In 2021, I organized a virtual undergraduate reading course in commutative algebra. 


I have organized over 15 conferences, special sessions, and workshops including: \textit{M2@UW} (45 participants), \textit{CAZoom} (70 participants), \textit{Western Algebraic Geometry Symposium} (100 participants), and $\Spec(\overline{\Q})$ (50 participants). When organizing these conferences I have given paid special attention to promoting women and other underrepresented groups in mathematics. For example, \textit{Graduate Workshop in Commutative Algebra for Women \& Mathematicians of Other Minority Genders} and \textit{GEMS in Combinatorics}  focused on forming communities of women and non-binary researchers in commutative algebra and combinatorics respectively. Further, $\Spec(\overline{\Q})$ highlighted the research of LGBTQ+ mathematicians in algebra, geometry, and number theory.









%At the 2018 Joint Mathematica Meeting, the PI served on a panel organized by \textit{Spectra}, the association of LGBTQ+ mathematicians, titled \textit{Professional Issues Facing LGBTQ Mathematicians}. At the 2020 Joint Meetings, the PI is organizing a \textit{Spectra} panel on transgender inclusion in mathematics. Going forward the PI plans to continue being involved in \textit{Spectra}, potentially stepping into more leadership and organizing roles. The PI has also founded and organized numerous groups for LGBTQ+ students at the University of Wisconsin. Since 2017 she has organized the campus social group for LGBTQ+ graduate students, which has over 350 members. While at Berkeley she plans to be involved 


%\noindent \textbf{Broader Impacts.} As an LGBTQ woman, the PI has worked hard to promote diversity, inclusivity, and justice in the mathematical community. This proposal will further the PI's work in this direction by her continued involvement  as a mentor to multiple undergraduate women via the Association for Women in Mathematics's mentor network. The PI also plans to mentor undergraduates via the Berkeley Directed Reading Program and the MSRI-Up program. 
%
%At the 2018 Joint Mathematica Meeting, the PI served on a panel organized by \textit{Spectra}, the association of LGBTQ+ mathematicians, titled \textit{Professional Issues Facing LGBTQ Mathematicians}. At the 2020 Joint Meetings, the PI is organizing a \textit{Spectra} panel on transgender inclusion in mathematics. Going forward the PI plans to continue being involved in \textit{Spectra}, potentially stepping into more leadership and organizing roles. The PI has also founded and organized numerous groups for LGBTQ+ students at the University of Wisconsin. Since 2017 she has organized the campus social group for LGBTQ+ graduate students, which has over 350 members. While at Berkeley she plans to be involved in similar organizing efforts. 
%
%The PI has organized a number of conferences including the \textit{Graduate Workshop in Commutative Algebra for Women and Mathematicians of Minority Genders} (2019), a workshop bringing together algebraic geometers and number theorists \textit{Geometry \& Arithmetic of Surfaces} (2019), and a five day conference dedicated to developing open-source computer software for algebraic geometry and commutative algebra \textit{M2@UW} (2018). The PI plans to organize a follow-up to \textit{Graduate Workshop in Commutative Algebra for Women and Mathematicians of Minority Genders} tentatively planned for Spring 2021, as well as a conference for LGBTQ+ mathematicians in algebraic geometry and commutative algebra. 



	

\end{document}