\documentclass[11pt,reqno]{amsart}
\usepackage{amsfonts,amsmath,amssymb,amsbsy,amstext,amsthm,mathtools}
\usepackage{accents,color,enumerate,enumitem,float,fullpage,verbatim}

\usepackage{url}
\usepackage[colorlinks=true,hyperindex, linkcolor=magenta, pagebackref=false, citecolor=cyan]{hyperref}
\usepackage[alphabetic,lite]{amsrefs} 

\usepackage{parskip}

%\usepackage{eucal,bm,kpfonts,mathbbol}
\usepackage[margin=1.02in,includeheadfoot]{geometry}

\usepackage{tikz,tikz-cd}	
\usetikzlibrary{positioning, matrix, shapes}         								    				
\usetikzlibrary{arrows,calc,matrix}

\usepackage{lscape}

\usepackage{microtype}


\usepackage{titlesec}		
\setcounter{secnumdepth}{4}						     					% Allows one to use nice section titles
\titleformat{\section}[block]{\scshape\bfseries\filcenter}{\thesection.}{1em}{}		% Creates section titles
\titleformat{\subsection}[runin]{\scshape\bfseries}{\thesubsection}{1em}{}			% Creates subsection titles
\titleformat{\subsubsection}[runin]{\scshape\bfseries}{\thesubsubsection}{1em}{}			% Creates subsection titles

\usepackage[titles]{tocloft}								     					% Creates table of fancy contents
\setcounter{tocdepth}{4}
\renewcommand{\contentsname}{}	     					% Renames and centers title of ToC

\usepackage{multirow}
\usepackage{array}
\usepackage{booktabs}
\newcolumntype{M}[1]{>{\centering\arraybackslash}m{#1}}
\newcolumntype{N}{@{}m{0pt}@{}}
\usepackage{diagbox}
\usepackage{cancel}

\newtheorem{lemma}{Lemma}[section]
\newtheorem{theorem}[lemma]{Theorem}
\newtheorem{goalTheorem}[lemma]{Goal Theorem}
\newtheorem{prop}[lemma]{Proposition}
\newtheorem{cor}[lemma]{Corollary}
\newtheorem{conj}[lemma]{Conjecture}
\newtheorem{claim}[lemma]{Claim}
\newtheorem{defn}[lemma]{Definition} 
\newtheorem{notation}[lemma]{Notation} 
\newtheorem{exercise}[lemma]{Exercise}
\newtheorem{question}[lemma]{Question}
\newtheorem*{assumption}{Assumption}
\newtheorem{principle}[lemma]{Principle}
\newtheorem{heuristic}[lemma]{Heuristic}

\newtheorem{theoremalpha}{Theorem}
\newtheorem{corollaryalpha}[theoremalpha]{Corollary}
\renewcommand{\thetheoremalpha}{\Alph{theoremalpha}}

\theoremstyle{remark}
\newtheorem{remark}[lemma]{Remark}
\newtheorem{example}[lemma]{Example}
\newtheorem{cexample}[lemma]{Counterexample}

% Commands
\newcommand{\initial}{\operatorname{in}}
\newcommand{\NF}{\operatorname{NF}}
\newcommand{\HF}{\operatorname{HF}}
\newcommand{\Hilb}{\operatorname{Hilb}}
\newcommand{\depth}{\operatorname{depth}}
\newcommand{\reg}{\operatorname{reg}}
\newcommand{\Span}{\operatorname{span}}
\newcommand{\img}{\operatorname{img}}
\newcommand{\inn}{\operatorname{in}}

\newcommand{\length}{\operatorname{length}}
\newcommand{\coker}{\operatorname{coker}}
\newcommand{\adeg}{\operatorname{adeg}}
\newcommand{\pdim}{\operatorname{pdim}}
\newcommand{\Spec}{\operatorname{Spec}}
\newcommand{\Ext}{\operatorname{Ext}}
\newcommand{\Tor}{\operatorname{Tor}}
\newcommand{\LT}{\operatorname{LT}}
\newcommand{\im}{\operatorname{im}}
\newcommand{\NS}{\operatorname{NS}}
\newcommand{\Frac}{\operatorname{Frac}}
\newcommand{\Khar}{\operatorname{char}}
\newcommand{\Proj}{\operatorname{Proj}}
\newcommand{\id}{\operatorname{id}}
\newcommand{\Div}{\operatorname{Div}}
\newcommand{\Kl}{\operatorname{Cl}}
\newcommand{\tr}{\operatorname{tr}}
\newcommand{\Tr}{\operatorname{Tr}}
\newcommand{\Supp}{\operatorname{Supp}}
\newcommand{\ann}{\operatorname{ann}}
\newcommand{\Gal}{\operatorname{Gal}}
\newcommand{\Pic}{\operatorname{Pic}}
\newcommand{\QQbar}{{\overline{\mathbb Q}}}
\newcommand{\Br}{\operatorname{Br}}
\newcommand{\Bl}{\operatorname{Bl}}
\newcommand{\Kox}{\operatorname{Cox}}
\newcommand{\conv}{\operatorname{conv}}
\newcommand{\getsr}{\operatorname{Tor}}
\newcommand{\diam}{\operatorname{diam}}
\newcommand{\Hom}{\operatorname{Hom}} %done
\newcommand{\sheafHom}{\mathcal{H}om}
\newcommand{\Gr}{\operatorname{Gr}}
\newcommand{\rank}{\operatorname{rank}} 
\newcommand{\codim}{\operatorname{codim}}
\newcommand{\Sym}{\operatorname{Sym}} %done
\newcommand{\GL}{{GL}}
\newcommand{\Prob}{\operatorname{Prob}}
\newcommand{\Density}{\operatorname{Density}}
\newcommand{\Syz}{\operatorname{Syz}}
\newcommand{\pd}{\operatorname{pd}}
\newcommand{\supp}{\operatorname{supp}}
\newcommand{\cone}{\operatorname{\textbf{cone}}}
\newcommand{\Res}{\operatorname{Res}}
\newcommand{\HS}{\operatorname{HS}}
\newcommand{\Cl}{\operatorname{Cl}}
\newcommand{\oO}{\operatorname{O}}

\newcommand{\defi}[1]{\textsf{#1}} % for defined terms

\newcommand{\remd}{\operatorname{remd}}
\newcommand{\colim}{\operatorname{colim}}
\newcommand{\trideg}{\operatorname{tri.deg}}
\newcommand{\indeg}{\operatorname{index.deg}}
\newcommand{\moddeg}{\operatorname{mod.deg}}
\newcommand{\Desc}{\operatorname{Desc}}
\newcommand{\inter}{\operatorname{int}}
\newcommand{\Nef}{\operatorname{Nef}}
\newcommand{\Jac}{\operatorname{Jac}}
\newcommand{\Cox}{\operatorname{Cox}}

\newcommand{\doot}{\bullet}

\newcommand{\Alt}{\bigwedge\nolimits}
\newcommand{\Set}{\text{\bf Set}}										% Category of Sets
\newcommand{\Sch}{\text{\bf Sch}}										% Category of Abelian Groups
\newcommand{\Mod}[1]{\ (\mathrm{mod}\ #1)}




%%%%%%%%%%%%%%%%%%%%%%%%%%%%%% Letters  %%%%%%%%%%%%%%%%%%%%%%%%%%%%%%%%%%%%%%%%%%%%
%%%%%%%%%%%%%%%%%%%%%%%%%%%%%%%%%%%%%%%%%%%%%%%%%%%%%%%%%%%%%%%%%%%%%%%%%%%%%%
\newcommand{\ff}{\mathbf f}
\newcommand{\kk}{\mathbf k}
\renewcommand{\aa}{\mathbf a}
\newcommand{\bb}{\mathbf b}
\newcommand{\cc}{\mathbf c}
\newcommand{\dd}{\mathbf d}
\newcommand{\ee}{\mathbf e}
\newcommand{\vv}{\mathbf v}
\newcommand{\ww}{\mathbf w}
\newcommand{\xx}{\mathbf x}
\newcommand{\yy}{\mathbf y}
\newcommand{\rr}{\mathbf r}
\newcommand{\ii}{\mathbf i}
\newcommand{\nn}{\mathbf n}
\newcommand{\pp}{\mathbf p}
\newcommand{\mm}{\mathbf m}
\newcommand{\fF}{\mathbf F}
\newcommand{\gG}{\mathbf G}
\newcommand{\eE}{\mathbf E}
\newcommand{\qQ}{\mathbf Q}
\newcommand{\tT}{\mathbf T}
\renewcommand{\tt}{\mathbf t}
\newcommand{\one}{\mathbf 1}
\newcommand{\zero}{\mathbf 0}

\renewcommand{\H}{\operatorname{H}}
\newcommand{\OO}{\operatorname{O}}
\newcommand{\oo}{\operatorname{o}}


%%%% Caligraphic Fonts - i.e. ????. %%%%%
\newcommand{\cA}{\mathcal{A}}
\newcommand{\cB}{\mathcal{B}}
\newcommand{\cC}{\mathcal{C}}
\newcommand{\cD}{\mathcal{D}}
\newcommand{\cE}{\mathcal{E}}
\newcommand{\cF}{\mathcal{F}}
\newcommand{\cG}{\mathcal{G}}
\newcommand{\cH}{\mathcal{H}} 
\newcommand{\cI}{\mathcal{I}}
\newcommand{\cJ}{\mathcal{J}}
\newcommand{\cK}{\mathcal{K}}
\newcommand{\cL}{\mathcal{L}}
\newcommand{\cM}{\mathcal{M}}
\newcommand{\cN}{\mathcal{N}}
\renewcommand{\O}{\mathcal{O}}
\newcommand{\cP}{\mathcal{P}}
\newcommand{\cQ}{\mathcal{Q}}
\newcommand{\cR}{\mathcal{R}}
\newcommand{\cS}{\mathcal{S}}
\newcommand{\cT}{\mathcal{T}}
\newcommand{\U}{\mathcal{U}} 		% Notice this is different
\newcommand{\cV}{\mathcal{V}}
\newcommand{\cW}{\mathcal{W}}
\newcommand{\cX}{\mathcal{X}}
\newcommand{\cY}{\mathcal{Y}}
\newcommand{\cZ}{\mathcal{Z}}

%%%% Blackboard Fonts - i.e. Real Numbers, Integers, etc. %%%%%
\newcommand{\A}{\mathbb{A}}
\newcommand{\B}{\mathbb{B}}
\newcommand{\C}{\mathbb{C}}
\newcommand{\D}{\mathbb{D}}
\newcommand{\E}{\mathbb{E}}
\newcommand{\F}{\mathbb{F}}
\newcommand{\G}{\mathbb{G}}
\newcommand{\I}{\mathbb{I}}
\newcommand{\J}{\mathbb{J}}
\newcommand{\K}{\mathbb{K}}
\renewcommand{\L}{\mathbb{L}}
\newcommand{\M}{\mathbb{M}}
\newcommand{\N}{\mathbb{N}}
\newcommand{\bO}{\mathbb{O}}		% Notice this is \bO
\renewcommand{\P}{\mathbb{P}}
\newcommand{\Q}{\mathbb{Q}}
\newcommand{\R}{\mathbb{R}}
\newcommand{\T}{\mathbb{T}}
\newcommand{\bU}{\mathbb{U}}		% Notice this is \bU
\newcommand{\V}{\mathbb{V}}
\newcommand{\W}{\mathbb{W}}
\newcommand{\X}{\mathbb{X}}
\newcommand{\Y}{\mathbb{Y}}
\newcommand{\Z}{\mathbb{Z}}

 %%%% Sarif Fonts - i.e. ???? %%%%%
\newcommand{\sA}{\mathsf{A}}
\newcommand{\sB}{\mathsf{B}}
\newcommand{\sC}{\mathsf{C}}
\newcommand{\sD}{\mathsf{D}}
\newcommand{\sE}{\mathsf{E}}
\newcommand{\sF}{\mathsf{F}}
\newcommand{\sG}{\mathsf{G}}
\newcommand{\sH}{\mathsf{H}} 
\newcommand{\sI}{\mathsf{I}}
\newcommand{\sJ}{\mathsf{J}}
\newcommand{\sK}{\mathsf{K}}
\newcommand{\sL}{\mathsf{L}}
\newcommand{\sM}{\mathsf{M}}
\newcommand{\sN}{\mathsf{N}}
\newcommand{\sO}{\mathsf{O}}
\newcommand{\sP}{\mathsf{P}}
\newcommand{\sQ}{\mathsf{Q}}
\newcommand{\sR}{\mathsf{R}}
\newcommand{\sS}{\mathsf{S}}
\newcommand{\sT}{\mathsf{T}}
\newcommand{\sU}{\mathsf{U}} 
\newcommand{\sV}{\mathsf{V}}
\newcommand{\sW}{\mathsf{W}}
\newcommand{\sX}{\mathsf{X}}
\newcommand{\sY}{\mathsf{Y}}
\newcommand{\sZ}{\mathsf{Z}}
 
 %%%% Fraktur Fonts - i.e. maximal ideals, prime ideals, etc. %%%%%
\newcommand{\cl}{\mathfrak{cl}}
\newcommand{\g}{\mathfrak{g}}
\newcommand{\h}{\mathfrak{h}}
\newcommand{\m}{\mathfrak{m}}
\newcommand{\n}{\mathfrak{n}}
\newcommand{\p}{\mathfrak{p}}
\newcommand{\q}{\mathfrak{q}}
\renewcommand{\r}{\mathfrak{r}}


\renewcommand{\cite}[1]{{}}

\newcommand{\juliette}[1]{{\color{red} \sf $\spadesuit\spadesuit\spadesuit$ Juliette: [#1]}}


\title{Juliette Bruce's Research Statement}

%\author{Juliette Bruce}
%\address{Department of Mathematics, University of Wisconsin, Madison, WI}
%\email{\href{mailto:juliette.bruce@math.wisc.edu}{juliette.bruce@math.wisc.edu}}
%\urladdr{\url{http://math.wisc.edu/~juliettebruce/}}

%\thanks{The author was partially supported by the NSF GRFP under Grant No. DGE-1256259 and NSF grant DMS-1502553.}

%\subjclass[2010]{13D02, 14M25}

\begin{document} 

%\maketitle
%\begingroup  
%  \centering
%  \large\scshape\bfseries Juliette Bruce's Statement of Purpose\\[1em]
%\endgroup

%\tableofcontents

\setcounter{section}{0}

Given a graded module $M$ over a graded ring $R$, in essence, a minimal graded free resolution is a way of approximating $M$ by a sequence of free $R$-modules. More formally, a \textit{graded free resolution} of a module $M$ is an exact sequence
%Given a graded module $M$ over a graded ring $R$, a helpful tool for understanding the structure of $M$ is its minimal graded free resolution. In essence, a minimal graded free resolution is a way of approximating $M$ by a sequence of free $R$-modules. More formally, a \textit{graded free resolution} of a module $M$ is an exact sequence 
\begin{center}
\begin{tikzcd}[column sep = 3em]
0 & \lar{} M & \arrow[l,"\epsilon" above]  F_{0} & \arrow[l,"d_{1}" above] \cdots &  & \cdots & \arrow[l,"d_{k-1}" above]  F_{k-1} & \arrow[l,"d_{k}" above] F_{k} & \lar \cdots
\end{tikzcd}
\end{center}
where each $F_{p}$ is a graded free $R$-module, and hence can be written as $\bigoplus_{q}R(-p)^{\beta_{p,q}}$. The module $R(-q)$ is the ring $R$ with a twisted grading, so that $R(-q)_{d}$ is equal to $R_{d-q}$ where $R_{d-q}$ is the graded piece of degree $d-q$. The $\beta_{p,q}$'s are the \textit{Betti numbers} of $M$, and they count the number of $p$-syzygies of $M$ of degree $q$. We will use syzygy and Betti number interchangeably throughout. 

Given a projective variety $X$ embedded in $\P^n$, we associate to $X$ the ring $S_X=S/I_X$, where $S=\C[x_0,\ldots,x_n]$ and $I_X$ is the ideal of homogeneous polynomials vanishing on $X$. As $S_X$ is a graded $S$-module we may consider its minimal graded free resolution, which is often closely related to both the extrinsic and intrinsic geometry of $X$. Much of my work has focused on studying the asymptotic properties of syzygies of projective varieties. Broadly, asymptotic syzygies is the study of the graded Betti numbers of a projective variety as the positivity of the embedding grows. 

%An example of this phenomenon is Green's Conjecture, which relates the Clifford index of a curve with the vanishing of certain $\beta_{p,q}$ for its canonical embedding \cite{voisin02, voisin05, aproduFarkas19}. See also \cite{eisenbud05}*{Conjecture 9.6} and \cite{schreyer86, bayerEisenbud91}.

%\begin{theorem}[\cite{voisin02}, \cite{voisin05}]
%Let $C$ be a generic smooth projective curve of genus $g$ over a characteristic zero field embedded in $\P^{g-1}$ by the complete canonical series. Then the length of the first linear strand of the minimal free resolution of $I_X$ is $g-3-\text{Cliff}(C)$.
%\end{theorem}

%Much of my work has focused on studying the asymptotic properties of syzygies of projective varieties. Broadly speaking, asymptotic syzygies is the study of the graded Betti numbers (i.e. the syzygies) of a projective variety as the positivity of the embedding grows. %In many ways, this perspective dates back to classical work on the defining equations of curves of high degree and projective normality \cite{mumford66, mumford70}. However, the modern viewpoint arose from the pioneering work of Green \cite{green84-I, green84-II} and later Ein and Lazarsfeld \cite{einLazarsfeld12}. 

To give a flavor of the results of asymptotic syzygies we will focus on the question: in what degrees do non-zero syzygies occur? Going forward we will let $X\subset \P^{n_{d}}$ be a smooth projective variety embedded by a very ample line bundle $L_{d}$, and we set, 
\begin{align*}
\rho_q\left(X,L_{d}\right)\;\;\coloneqq&\ \;\; \tfrac{1}{n_{d}}\cdot \#\left\{p\in\N |\; \big| \; \beta_{p,p+q}\left(X,L_{d}\right)\neq0\right\}
%\rho_q\left(X,L_{d}\right)\;\;\coloneqq&\ \;\; \frac{\#\left\{p\in\N |\; \big| \; \beta_{p,p+q}\left(X,L_{d}\right)\neq0\right\}}{n_{d}},
\end{align*}
which is the percentage of degrees in which non-zero syzygies appear. The asymptotic perspective asks how $\rho_{q}(X;L_{d})$ behaves along the sequence of line bundles $(L_{d})_{d\in \N}$. With this notation in hand, we may phrase Green's work on the vanishing of syzygies for curves of high degree as computing the asymptotic percentage of non-zero quadratic syzygies. 

%\begin{align*}
%\rho_q\left(X;L_{d}\right)\;\;\coloneqq&\ \;\; \frac{\#\left\{p\in\N |\; \big| \; \beta_{p,p+q}\left(X,L_{d}\right)\neq0\right\}}{r_{d}}.
%\end{align*}
%which by the Hilbert Syzygy Theorem is the percentage of degrees in which non-zero syzygies appear \cite{eisenbud05}*{Theorem~1.1}. For any particular, $X$, $L_{d}$, and $q$ computing $\rho_{q}(X;L_{d})$ is often quite difficult. The asymptotic perspective thus, asks instead, to consider a sequence of line bundles $(L_{d})_{d\in \N}$ and ask how $\rho_{q}(X;L_{d})$ behaves along the sequence of $(L_{d})_{d\in \N}$. 

%With this notation in hand, we may phrase Green's work on the vanishing of syzygies for curves of high degree as computing the asymptotic percentage of non-zero quadratic syzygies. 

\begin{theoremalpha}%\cite{green84-I}
Let $X$ be a smooth projective curve. If $(L_{d})_{d\in\N}$ is a sequence of very ample line bundles on $X$ such that $\deg L_{d} = d$ then $\rho_{2}\left(X;L_{d}\right) \to 0$ as $d\to \infty$.
%\[
%\lim_{d\to \infty} \rho_{2}\left(X;L_{d}\right) = 0.
%\]
\end{theoremalpha}

Put differently, asymptotically the syzygies of curves are as simple as possible, occurring in the lowest possible degree. This inspired substantial work, with the intuition being that syzygies become simpler as the positivity of the embedding increases. Ein and Lazarsfeld showed that for higher dimensional varieties this intuition is often misleading. Contrary to the case of curves, they show that for higher dimensional varieties, asymptotically syzygies appear in every possible degree.%\cite{ottavianiPaoletti01, einLazarsfeld93, lazarsfeldPareschiPopa11, pareschi00, pareschiPopa03, pareschiPopa04}.  
  
\begin{theoremalpha}%\cite{einLazarsfeld12}*{Theorem~C}
Let $X$ be a smooth projective variety, $\dim X \geq2$. Fix $1\leq q \leq \dim X$. If $(L_{d})$ is a sequence of very ample line bundles such that $L_{d+1}-L_{d}$ is constant and ample then $ \rho_{q}\left(X; L_d\right)\to 1$ as $d\to \infty$.
%\[
%\lim_{d\to\infty} \rho_{q}\left(X; L_d\right) = 1.
%\]
\end{theoremalpha}

My work has focused on the behavior of asymptotic syzygies when the condition that $L_{d+1}-L_{d}$ is constant and ample is weakened to assuming $L_{d+1}-L_{d}$ is semi-ample. A line bundle $L$ is \textit{semi-ample} if $|kL|$ is base point free for $k\gg0$. The prototypical example of a semi-ample line bundle is $\O(1,0)$ on $\P^{n}\times \P^{m}$. My exploration of asymptotic syzygies in the setting of semi-ample growth thus began by proving the following nonvanishing result for $\P^{n}\times\P^{m}$ embedded by $\O(d_{1},d_{2})$. 

\begin{theoremalpha}%\cite{bruce19-semiample}*{Corollary~B}\label{thm:bruce-semiample}
Let $X=\P^{n}\times\P^{m}$ and fix  $1\leq q \leq n+m$. There exist constants $C_{i,j}$, $D_{i,j}$ such that
\[
\rho_{q}\left(X; \O\left(d_1,d_2\right)\right)\geq1-\sum_{\substack{i+j=q \\  i \leq n, \; j \leq m}}\left(
\frac{C_{i,j}}{d_1^id_2^j}+\frac{D_{i,j}}{d_1^{n-i}d_2^{m-j}}\right)-O\left(\begin{matrix}\text{lower ord.}\\ \text{terms}\end{matrix}\right).
\]
\end{theoremalpha}

Notice if both $d_{1}\to \infty$ and $d_{2}\to\infty$ then $\rho_{q}\left(\P^{n}\times\P^{m}; \O(d_1,d_2)\right)\to1$, recovering the results of Ein and Lazarsfeld for $\P^n\times\P^m$. However, if $d_{1}$ is fixed and $d_{2}\to \infty$ (i.e. semi-ample growth) my results bound the asymptotic percentage of non-zero syzygies away from zero. This together with work of Lemmens has led me to conjecture that, unlike in previously studied cases, in the semi-ample setting $\rho_{q}\left(\P^{n}\times\P^{m}; \O(d_1,d_2)\right)$ does not approach 1. Proving this would require a vanishing result for asymptotic syzygies, which is open even in the ample case 

\noindent \textbf{Relevance of Visit.} Being a postdoctoral fellow at MSRI would be extremely beneficially towards my career goals as it would provide me with a stimulating mathematical environment, and the opportunity to engage and work with a number of leading researchers in commutative algebra. My research interest align extremely well with topics the of semester long program. For example, one ongoing project I am working on concerns better understanding the homological implications of certain generalizations of Castelnuovo--Mumford regularity to toric varieties, and in a separate project I am beginning to explore ways Green's conjecture may be generalized to describe the syzygies of canonical stacky curves. Both of these projects touch upon many of the topics included in the program description. As such begin a postdoc at MSRI would likely prove valuable toward progressing both of these projects, as well as a fantastic opprotunity to being new collaborations with other participants. Moreover, given the the ways my research interest align with the program I feel I would be able to find numerous ways to myself contribute to the program.% including by discussing my own work, building new collaborations, and NEDEDED. 

Overall, being a postdoctoral fellow for this semester program in commutative algebra presents an amazing opportunity to connect and build relationships within the research area that has long felt like my research community/home. And such connections would significantly advance my goals of being a math professor at a research university.

%Finally, a particular aspect I have enjoyed about working in commutative algebra has been the strong sense of community I have found within the field. Being a postdoctoral fellow for this semester program thus presents an amazing opportunity to connect and build relationships within the research area that has long felt like my research home.


%  Working with David Eisenbud would provide the opportunity to learn from one of the foremost experts in algebraic geometry and commutative algebra. Eisenbud has done foundational work concerning the syzygies of varieties, liaison theory, and the geometry of curves. Working with him would likely prove valuable towards the first two projects I plan on working on while visiting MSRI. In short, working with Eisenbud would significantly advance my goals of being a math professor at a research university.


\noindent \textbf{Building Mathematical Community.} As an LGBTQ+ woman, I have worked hard to promote diversity, inclusivity, and justice in the mathematical community by mentoring students, supporting women and LGBTQ+ people in mathematics, and organizing conferences and outreach programs. %For brevity, below I will discuss a few of my more recent efforts in these directions to build community.

As a postdoc, I have put significant effort into mentoring, advising, and working with students, especially those from underrepresented groups. During Summer 2021 in order to help fill gaps caused by the COVID-19 pandemic I organized a virtual summer undergraduate research program for 6 undergraduates from around the world. In Summer 2022 I advised an undergraduate student on a research project related to my work on asymptotic syzygies. This student is now applying to graduate schools in math. I began research projects with multiple graduate students, in which I played a substantial mentoring and guiding role. In the Spring and Summer of 2022, I did a reading course with a first-year graduate woman who is now working in commutative algebra.

%undergraduates
%I advised two summer research projects for undergraduate students. The first of these projects ran virtually during Summer 2021 when 6 undergraduates. In Summer 2022 I advised an undergraduate student on a research project related to my work on syzygies discussed on the previous page. This student is now applying to graduate schools in math. As a postdoc, I began research projects with multiple graduate students (a majority of whom identify with a generally underrepresented group), in which I played a substantial mentoring and guiding role. These projects have resulted in two pre-prints, with additional projects still ongoing. Throughout the Spring and Summer of 2022, I did a reading course with a first-year graduate woman on algebraic geometry.

Since Fall 2020 I have organized an annual virtual conference promoting the work of transgender and non-binary mathematicians. Highlighting the importance of such conferences one participant said, ``I've been really considering leaving mathematics.  [Trans Math Day 2020] reminded me why I'm here and why I want to stay. ... If a conference like this had been around for me five years ago, my life would have been a lot better.'' Trans Math Day regularly has ~50 participants. %The next Trans Math Day is in December 2022. 

For over the last two years I have served as a board member for \textit{Spectra: The Association for LGBTQ+ Mathematicians}. As a board member I have overseen the growth and formalization of the organization, including the creation and adoption of bylaws, the creation of an invited lecture at the Joint Mathematics Meetings, and a fundraising campaign that has raised over \$20,000 to support LGBTQ+ students and mathematicians. I am currently the inaugural president of Spectra, and as of right now we have over 500 people on our mailing/membership lists.  


In response to the COVID-19 pandemic and the shift of many mathematical activities to virtual formats, I worked to find ways for these online activities to reach those often at the periphery. During Summer and Fall 2020, I helped with Ravi Vakil's \textit{Algebraic Geometry in the Time of Covid} project. This massive online open-access course in algebraic geometry brought together $\sim2,000$ participants from around the world. In Spring 2021, I organized an 8-week virtual reading course for undergraduates in algebraic geometry and commutative algebra. 


I have organized over 15 conferences, special sessions, and workshops including: \textit{M2@UW} (45 participants), \textit{Graduate Workshop in Commutative Algebra for Women \& Mathematicians of Minority Genders} (35 participants), \textit{CAZoom} (70 participants), \textit{Western Algebraic Geometry Symposium} (100 participants)x, and $\Spec(\overline{\Q})$ (50 participants). When organizing these conferences I have given paid special attention to promoting women and other underrepresented groups in mathematics. For example, \textit{Graduate Workshop in Commutative Algebra for Women \& Mathematicians of Other Minority Genders} and \textit{GEMS in Combinatorics}  focused on forming communities of women and non-binary researchers in commutative algebra and combinatorics respectively. Further, $\Spec(\overline{\Q})$ highlighted the research of LGBTQ+ mathematicians in algebra, geometry, and number theory.









%At the 2018 Joint Mathematica Meeting, the PI served on a panel organized by \textit{Spectra}, the association of LGBTQ+ mathematicians, titled \textit{Professional Issues Facing LGBTQ Mathematicians}. At the 2020 Joint Meetings, the PI is organizing a \textit{Spectra} panel on transgender inclusion in mathematics. Going forward the PI plans to continue being involved in \textit{Spectra}, potentially stepping into more leadership and organizing roles. The PI has also founded and organized numerous groups for LGBTQ+ students at the University of Wisconsin. Since 2017 she has organized the campus social group for LGBTQ+ graduate students, which has over 350 members. While at Berkeley she plans to be involved 


%\noindent \textbf{Broader Impacts.} As an LGBTQ woman, the PI has worked hard to promote diversity, inclusivity, and justice in the mathematical community. This proposal will further the PI's work in this direction by her continued involvement  as a mentor to multiple undergraduate women via the Association for Women in Mathematics's mentor network. The PI also plans to mentor undergraduates via the Berkeley Directed Reading Program and the MSRI-Up program. 
%
%At the 2018 Joint Mathematica Meeting, the PI served on a panel organized by \textit{Spectra}, the association of LGBTQ+ mathematicians, titled \textit{Professional Issues Facing LGBTQ Mathematicians}. At the 2020 Joint Meetings, the PI is organizing a \textit{Spectra} panel on transgender inclusion in mathematics. Going forward the PI plans to continue being involved in \textit{Spectra}, potentially stepping into more leadership and organizing roles. The PI has also founded and organized numerous groups for LGBTQ+ students at the University of Wisconsin. Since 2017 she has organized the campus social group for LGBTQ+ graduate students, which has over 350 members. While at Berkeley she plans to be involved in similar organizing efforts. 
%
%The PI has organized a number of conferences including the \textit{Graduate Workshop in Commutative Algebra for Women and Mathematicians of Minority Genders} (2019), a workshop bringing together algebraic geometers and number theorists \textit{Geometry \& Arithmetic of Surfaces} (2019), and a five day conference dedicated to developing open-source computer software for algebraic geometry and commutative algebra \textit{M2@UW} (2018). The PI plans to organize a follow-up to \textit{Graduate Workshop in Commutative Algebra for Women and Mathematicians of Minority Genders} tentatively planned for Spring 2021, as well as a conference for LGBTQ+ mathematicians in algebraic geometry and commutative algebra. 



	

\end{document}