% (c) 2002 Matthew Boedicker <mboedick@mboedick.org> (original author) http://mboedick.org
% (c) 2003-2007 David J. Grant <davidgrant-at-gmail.com> http://www.davidgrant.ca
% (c) 2008-2014 Nathaniel Johnston <nathaniel@njohnston.ca> http://www.njohnston.ca
%
% Depending on your TeX distribution, you may need to download the revnum and longtable packages for this template to work!
%
%This work is licensed under the Creative Commons Attribution-Noncommercial-Share Alike 2.5 License. To view a copy of this license, visit http://creativecommons.org/licenses/by-nc-sa/2.5/ or send a letter to Creative Commons, 543 Howard Street, 5th Floor, San Francisco, California, 94105, USA.

\documentclass[letterpaper,11pt]{article}
\newlength{\outerbordwidth}
\usepackage{amstext,amsfonts,amssymb,amscd,amsbsy,amsmath}
\pagestyle{empty}
\raggedbottom
\raggedright
\usepackage{array}
\usepackage[svgnames]{xcolor}
\usepackage{enumerate}
\usepackage{framed}
\usepackage{longtable}
\usepackage{revnum}
\usepackage{textcomp}

\usepackage{parskip}
\usepackage[colorlinks=true,urlcolor=blue]{hyperref}
\usepackage{tocloft}
\usepackage{array}


%-----------------------------------------------------------
%Edit these values as you see fit

\setlength{\outerbordwidth}{3pt}  % Width of border outside of title bars
\definecolor{shadecolor}{gray}{0.75}  % Outer background color of title bars (0 = black, 1 = white)
\definecolor{shadecolorB}{gray}{0.93}  % Inner background color of title bars


%-----------------------------------------------------------
%Margin setup

\setlength{\evensidemargin}{-0.25in}
\setlength{\headheight}{0in}
\setlength{\headsep}{0in}
\setlength{\oddsidemargin}{-0.25in}
\setlength{\paperheight}{11in}
\setlength{\paperwidth}{8.5in}
\setlength{\tabcolsep}{0in}
\setlength{\textheight}{9.5in}
\setlength{\textwidth}{7in}
\setlength{\topmargin}{-0.3in}
\setlength{\topskip}{0in}
\setlength{\voffset}{0.1in}
\setlength\LTleft{0.2in} % needed to make longtable full-width
\setlength\LTright{0.2in}

%-----------------------------------------------------------
%Custom commands
\newcommand{\resitem}[1]{\item #1 \vspace{-2pt}}
\newcommand{\resheading}[1]{\vspace{8pt}
  \parbox{\textwidth}{\setlength{\FrameSep}{\fboxsep}
    \begin{shaded}
\setlength{\fboxsep}{0pt}\framebox[\textwidth][l]{\setlength{\fboxsep}{4pt}\fcolorbox{shadecolorB}{shadecolorB}{\textbf{\sffamily{\mbox{~}\makebox[6.762in][l]{\large #1} \vphantom{p\^{E}}}}}}
    \end{shaded}
  }\vspace{-5pt}
}

% the next four commands allow for the \ressubheading environment to be 1, 2, 3, or 4 subrows, depending on which command you use. This is admittedly hack-ish, and should probably be replaced by a single more flexible command (with optional arguments) in the future
\newcommand{\ressubheading}[4]{
\begin{tabular*}{6.5in}[t]{l@{\cftdotfill{\cftsecdotsep}\extracolsep{\fill}}r}
		\textbf{#1} & #2 \\
		\textit{#3} & \textit{#4} \\
\end{tabular*}\vspace{-6pt}}
\newcommand{\ressubheadingb}[6]{
\begin{tabular*}{6.5in}[t]{l@{\cftdotfill{\cftsecdotsep}\extracolsep{\fill}}r}
		\textbf{#1} & #2 \\
		\textit{#3} & \textit{#4} \\
		\textit{#5} & \textit{#6} \\
\end{tabular*}\vspace{-6pt}}
\newcommand{\ressubheadingc}[8]{
\begin{tabular*}{6.5in}[t]{l@{\cftdotfill{\cftsecdotsep}\extracolsep{\fill}}r}
		\textbf{#1} & #2 \\
		\textit{#3} & \textit{#4} \\
		\textit{#5} & \textit{#6} \\
		\textit{#7} & \textit{#8} \\
\end{tabular*}\vspace{-6pt}}
\newcommand\foo[9]{%
    \def\tempb{#2}%
    \def\tempc{#3}%
    \def\tempd{#4}%
    \def\tempe{#5}%
    \def\tempf{#6}%
    \def\tempg{#7}%
    \def\temph{#8}%
    \def\tempi{#9}%
    \foocontinued
}
\newcommand\foocontinued[7]{%
    % Do whatever you want with your 9+7 arguments here.
}

\newcommand{\ressubheadingd}[1]{
	\def\argten{#1}%
	\ressubheadingdb
}
\newcommand{\ressubheadingdb}[9]{
\begin{tabular*}{6.5in}[t]{l@{\cftdotfill{\cftsecdotsep}\extracolsep{\fill}}r}
		\textbf{\argten} & #1 \\
		\textit{#2} & \textit{#3} \\
		\textit{#4} & \textit{#5} \\
		\textit{#6} & \textit{#7} \\
		\textit{#8} & \textit{#9} \\
\end{tabular*}\vspace{-6pt}}
%-----------------------------------------------------------

\usepackage[lf]{Baskervaldx}

\begin{document}

%%%%%%%%%%%%%%%%%%%%%%%%%%%%%%
%%%%%%%%%%%%%%%%%%%%%%%%%%%%%%
\resheading{Summary of Student Evaluations}
%%%%%%%%%%%%%%%%%%%%%%%%%%%%%%
%%%%%%%%%%%%%%%%%%%%%%%%%%%%%%

During her time as a graduate student at the University of Wisconsin - Madison Juliette Bruce has held appointments as a teaching assistant for eleven semesters. In her first year Juliette was a teaching assistant for Math 221: Calculus I and the instructor for Math 132, a math course for education majors. In these roles her student evaluations were generally positive with particularly strong student comments. Since then she has held a non-standard teaching assistantship with the Madison Math Circle outreach program for six semesters. More recently Juliette has twice served as a teaching assistant and coordinator for Math 221: Calculus I, and once been the instructor for the accompanying Math 228: Wisconsin Emerging Scholars course. (Math 228 is a course taken in addition to Calculus I to provide students from generally underrepresented groups additional support and community.) In all of these courses her student evaluations were near perfect, and she received a number of glowing student comments.   An  overview of the student evaluations she has received is below:

\[
\begin{tabular}{| c | c | c | c | c | c |}
\hline
\textbf{Semester} & \textbf{Course \#} & \textbf{Course Title} & \textbf{Rating} & \;\;\textbf{Overall}\;\; & \;\;\textbf{Course Percentile}\;\;\\ \hline
Fall 2019& \;\; Math 221 \;\; & $\begin{matrix}
\text{Calculus \& }\\
\text{Analytic Geometry I}
\end{matrix}$ & \;\; Superior \;\;& 4.91 & 100\% \\ \hline

Fall 2018 & \;\; Math 228 \;\; & $\begin{matrix}
\text{Wisconsin Emerging Scholars}
\end{matrix}$ & \;\; Superior \;\;& 5.00 & 100\% \\ \hline

Fall 2018 & \;\; Math 221 \;\; & $\begin{matrix}
\text{Calculus \& }\\
\text{Analytic Geometry I}
\end{matrix}$ & \;\; Superior \;\;& 5.00 & 100\% \\ \hline

Fall 2018 & n/a & Madison Math Circle & n/a & n/a & n/a \\ \hline

Spring 2018 & n/a & Madison Math Circle & n/a & n/a & n/a \\ \hline

Fall 2017 & n/a & Madison Math Circle & n/a & n/a & n/a \\ \hline
%%
\;\; Spring 2017 \;\; & n/a & Madison Math Circle & n/a & n/a & n/a\\ \hline
%%
Fall 2016 & n/a & Madison Math Circle & n/a & n/a & n/a \\ \hline
%%
\;\;  Spring 2016 \;\;  & n/a & Madison Math Circle & n/a & n/a & n/a \\ \hline
%%
\;\;  Spring 2015 \;\; & \;\; Math 132 \;\; & $\quad\begin{matrix}
\text{Problem Solving in Algebra,}\\
\text{Statistics, \& Probability}\end{matrix}\quad$  & Satisfactory & 3.96 & n/a \\ \hline
%%
Fall 2014 & \;\; Math 221 \;\; & $\begin{matrix}
\text{Calculus \& }\\
\text{Analytic Geometry I}
\end{matrix}$ & \;\; Satisfactory Plus \;\;& 4.74 & n/a\\ \hline
\end{tabular}
\]

These ratings were determinate by the Mathematics Department's Committee on Teaching Assistant Performance and Pay, which consists of faculty, academic staff and teaching assistants. The committee bases its rating (Unsatisfactory, Needs Improvement, Satisfactory Minus, Satisfactory, Satisfactory Plus, Superior) on numerical scores and student comments. The overall score is the mean of 14 questions on a scale of 1-5. The course percentile is compared to instructors teaching the same course in recent years, and was only reported beginning in 2018.
\\
\
\\
Note the (relative) small number of students in a Math 132 class, and the small number of sections make the numerical evaluation scores extremely noisy and unstable. For example, when Juliette taught Math 132 only six out twenty of students completed the student evaluations. As such student comments are extremely important when evaluating someone teaching Math 132. In Juliette's case her comments are quite positive, and indicate she was effective in creating a classroom atmosphere in which students felt comfortable participating. This is generally in line with her evaluations for Math 221, which were exceptional.
\\
\
\\
Because of the non-standard nature of the Madison Math Circle teaching assistantship student evaluations are not done. Comments from parents and participants are collected, however, and some have been included in the excepted student comments section. 



\end{document}