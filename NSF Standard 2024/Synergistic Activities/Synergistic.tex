\documentclass[11pt]{amsart}

\usepackage{mathrsfs}
\usepackage[OT2, T1]{fontenc}
\usepackage{url}
\usepackage{amsmath}
\usepackage{array}
\usepackage{graphicx}
\usepackage{amsfonts}
\usepackage{amssymb}
\usepackage{amstext}
\usepackage{amsthm}
\usepackage{hyperref}
\usepackage{colonequals}
\usepackage{enumitem}
\usepackage[alphabetic,lite]{amsrefs}
\usepackage{cleveref}
\usepackage[all,cmtip]{xy}
\usepackage{fullpage}

\setcounter{totalnumber}{3}
\setcounter{topnumber}{1}
\setcounter{bottomnumber}{3}
\setcounter{secnumdepth}{3}

\numberwithin{equation}{subsection}

\newtheorem{theorem}{Theorem}
\numberwithin{theorem}{section}
\newtheorem{lemma}[theorem]{Lemma}
\newtheorem{proposition}[theorem]{Proposition}
\newtheorem{corollary}[theorem]{Corollary}
\newtheorem*{thm}{Theorem}
\newtheorem{conj}[theorem]{Conjecture}
\newtheorem{ques}[theorem]{Question}
\newtheorem*{conjecture}{Conjecture}
\newtheorem{project}[theorem]{Project}
\newtheorem{ttheorem}[theorem]{Example Target Theorem}

\theoremstyle{definition}
\newtheorem{defn}[theorem]{Definition}
\newtheorem{ass}[theorem]{Assumption}

\theoremstyle{remark}
\newtheorem{remark}[theorem]{Remark}
\newtheorem*{claim}{Claim}
\newtheorem{para}[theorem]{}
\newtheorem{example}[theorem]{Example}



\newcommand{\pdiv}{\mathscr{G}}
%-----------------------Blackboard letters
\newcommand{\bA}{\mathbb{A}}
\newcommand{\bB}{\mathbb{B}}
\newcommand{\bC}{\mathbb{C}}
\newcommand{\bD}{\mathbb{D}}
\newcommand{\bE}{\mathbb{E}}
\newcommand{\bF}{\mathbb{F}}
\newcommand{\bG}{\mathbb{G}}
\newcommand{\bH}{\mathbb{H}}
\newcommand{\bI}{\mathbb{I}}
\newcommand{\bJ}{\mathbb{J}}
\newcommand{\bK}{\mathbb{K}}
\newcommand{\bN}{\mathbb{N}}
\newcommand{\bP}{\mathbb{P}}
\newcommand{\bQ}{\mathbb{Q}}
\newcommand{\bR}{\mathbb{R}}
\newcommand{\bS}{\mathbb{S}}
\newcommand{\bT}{\mathbb{T}}
\newcommand{\bU}{\mathbb{U}}
\newcommand{\bV}{\mathbb{V}}
\newcommand{\bZ}{\mathbb{Z}}

%-----------------------Bold letters
\newcommand{\bbA}{\mathbf{A}}
\newcommand{\bbB}{\mathbf{B}}
\newcommand{\bbC}{\mathbf{C}}
\newcommand{\bbD}{\mathbf{D}}
\newcommand{\bbE}{\mathbf{E}}
\newcommand{\bbF}{\mathbf{F}}
\newcommand{\bbG}{\mathbf{G}}
\newcommand{\bbH}{\mathbf{H}}
\newcommand{\bbK}{\mathbf{K}}
\newcommand{\bbL}{\mathbf{L}}
\newcommand{\bbN}{\mathbf{N}}
\newcommand{\bbP}{\mathbf{P}}
\newcommand{\bbQ}{\mathbf{Q}}
\newcommand{\bbR}{\mathbf{R}}
\newcommand{\bbS}{\mathbf{S}}
\newcommand{\bbT}{\mathbf{T}}
\newcommand{\bbU}{\mathbf{U}}
\newcommand{\bbX}{\mathbf{X}}
\newcommand{\bbY}{\mathbf{Y}}
\newcommand{\bbZ}{\mathbf{Z}}

%-----------------------
\newcommand{\bbf}{\mathbf{f}}
\newcommand{\bbx}{\mathbf{x}}
\newcommand{\bby}{\mathbf{y}}
\newcommand{\bbz}{\mathbf{z}}
\newcommand{\bbn}{\mathbf{n}}



%-----------------------Calligraphic letters
\newcommand{\cA}{\mathcal{A}}
\newcommand{\cB}{\mathcal{B}}
\newcommand{\cC}{\mathcal{C}}
\newcommand{\cD}{\mathcal{D}}
\newcommand{\cE}{\mathcal{E}}
\newcommand{\cF}{\mathcal{F}}
\newcommand{\cG}{\mathcal{G}}
\newcommand{\cH}{\mathcal{H}}
\newcommand{\cI}{\mathcal{I}}
\newcommand{\cJ}{\mathcal{J}}
\newcommand{\cK}{\mathcal{K}}
\newcommand{\cL}{\mathcal{L}}
\newcommand{\cM}{\mathcal{M}}
\newcommand{\cN}{\mathcal{N}}
\newcommand{\cO}{\mathcal{O}}
\newcommand{\cP}{\mathcal{P}}
\newcommand{\cQ}{\mathcal{Q}}
\newcommand{\cR}{\mathcal{R}}
\newcommand{\cS}{\mathcal{S}}
\newcommand{\cT}{\mathcal{T}}
\newcommand{\cU}{\mathcal{U}}
\newcommand{\cV}{\mathcal{V}}
\newcommand{\cW}{\mathcal{W}}
\newcommand{\cX}{\mathcal{X}}
\newcommand{\cY}{\mathcal{Y}}
\newcommand{\cZ}{\mathcal{Z}}

\newcommand{\fa}{\mathfrak{a}}
\newcommand{\fd}{\mathfrak{d}}
\newcommand{\fp}{\mathfrak{p}}
\newcommand{\fo}{\mathfrak{o}}
\newcommand{\fq}{\mathfrak{q}}
\newcommand{\fl}{\mathfrak{l}}
\newcommand{\fG}{\mathfrak{G}}
\newcommand{\fsp}{\mathfrak{s}\mathfrak{p}}

\newcommand{\dA}{A^\vee}
\newcommand{\dB}{B^\vee}
\newcommand{\et}{{\text{\'et}}}
\newcommand{\gs}{{\gamma_\sigma}}
\newcommand{\rs}{{\rho_\sigma}}
\newcommand{\ob}{\overline{\beta}}
\newcommand{\ub}{\underline{\beta}}
\newcommand{\shbar}{\overline{\cS}}
\newcommand{\Vh}{\widehat{V}}
\newcommand{\bcH}{\overline{\cH}^{\rm{tor}}}
\newcommand{\bcS}{\overline{\cS}^{tor}}
\newcommand{\homg}{\overline{\omega}}
\newcommand{\sa}{\textrm{univ-sa}}


\newcommand{\dR}{_{\mathrm{dR}}}
\newcommand{\cris}{_{\mathrm{cris}}}
\newcommand{\MT}{_{\mathrm{MT}}}
\newcommand{\cm}{{\mathrm{CM}}}
\newcommand{\pet}{{\mathrm{Pet}}}
\newcommand{\Frob}{{\mathrm{Frob}}}

\newcommand{\ord}{{\mathrm{ord}}}
\newcommand{\can}{{\mathrm{can}}}
\newcommand{\an}{{\mathrm{an}}}
\newcommand{\So}{\Sigma^\ord}
\newcommand{\Sd}{\Sigma^{\mathrm {bad}}}
\newcommand{\inj}{\hookrightarrow}


\DeclareMathOperator{\GL}{GL}
\DeclareMathOperator{\SL}{SL}
\DeclareMathOperator{\GSp}{GSp}
\DeclareMathOperator{\Sp}{Sp}
\DeclareMathOperator{\SO}{SO}
\DeclareMathOperator{\GU}{GU}

\DeclareMathOperator{\Gal}{Gal}
\DeclareMathOperator{\End}{End}
\DeclareMathOperator{\Hom}{Hom}
\DeclareMathOperator{\Aut}{Aut}
\DeclareMathOperator{\Lie}{Lie}
\DeclareMathOperator{\Res}{Res}
\DeclareMathOperator{\Spec}{Spec}
\DeclareMathOperator{\ad}{ad}
\DeclareMathOperator{\Span}{Span}
\DeclareMathOperator{\der}{der}
\DeclareMathOperator{\Fil}{Fil}
\DeclareMathOperator{\disc}{disc}
\DeclareMathOperator{\diag}{diag}
\DeclareMathOperator{\Nm}{Nm}
\DeclareMathOperator{\Deg}{Deg}
\DeclareMathOperator{\Id}{Id}
\DeclareMathOperator{\Div}{Div}
\DeclareMathOperator{\Length}{Length}
\DeclareMathOperator{\Tr}{Tr}
\DeclareMathOperator{\coker}{coker}
\DeclareMathOperator{\Li}{Li}
\DeclareMathOperator{\Jac}{Jac}


\newcommand{\FEnd}{F{\text{-}}{\End}}
\newcommand{\FHom}{F{\text{-}}{\Hom}}

\usepackage[usenames,dvipsnames]{color}  %color comments
\newcommand{\yunqing}[1]{{\color{Blue} \sf  Yunqing: [#1]}}



%\usepackage[notcite]{showkeys}

\begin{document}
\title{Synergistic Activities}

\maketitle

\begin{itemize}
   \item \textbf{Research Conference Organizing:} The PI has organized many national research conferences and workshops. Most recently the PI was a co-organizer for the Fall 2024,  \textit{Algebraic Geometry Northeastern Series (AGNES)}, a weekend conference with ~100 participants held at Dartmouth College aimed at introducing early career researchers (e.g. graduate students and postdocs) to current research trends in algebraic geometry and related fields. 
   
    \vspace{.2 cm}
    
            \item \textbf{Developed Open-Source Software:} The PI has a long commitment to developing open-source software to promote and support research in algebraic geometry and commutative algebra. For example, the PI developed the \textit{VirtualResolutions} package for the computer algebra system \textit{Macaulay2}. This package, which includes core functionality for multigraded homological algebra on toric varieties, has been proven beneficial to many research projects and is frequently cited. 
    
        \vspace{.2 cm}
        
        \item \textbf{Invited Research Talks:} The PI has given many invited research talks including giving a plenary talk on homological algebra on toric varieties at $\text{Spec}(\overline{\mathbb{Q}}(2\pi i)$, a conference held at the Fields Institute in Summer 2024. 
      
        \vspace{.2 cm}

   \item \textbf{Promoting Women in Math:} The PI was a co-organizer/PI for the \textit{GEMS (Gender Equity in the Mathematical Study) of Commutative Algebra}, an NSF-funded workshop focusing on forming a community of women and non-binary researchers interested in commutative algebra by learning about specific topics in commutative algebra from a diverse group of prominent active researchers. This workshop was held at the University of Minnesota in November 2023 with ~40 participants.
       
    \vspace{.2 cm}
    
    \item \textbf{Leadership in Supporting Underrepresented Mathematicians:} The PI served as the inaugural president for \textit{Spectra: The Association for LGBTQ+ Mathematicians}, aimed at promoting community and supporting lesbian, gay, bisexual, transgender, and queer mathematicians and students. This included growing the organization's budget by \$20,000 and its membership to over 500 people, and launching the organization's annual invited lecture series at the Joint Mathematics Meeting.
    \vspace{.2 cm}
    \end{itemize}

%\begin{itemize}
%    \item The PI was the lecturer for mini-course on ``Elliptic curves with complex multiplication'' at Preliminary Arizona Winter School 2023: Elliptic Curves and Abelian Varieties held online in Sep-Nov 2023.
%
%    \vspace{.2 cm}
%
%    \item The PI was a co-organizer for ICERM Workshop The Ceresa Cycle in Arithmetic and Geometry at the Institute for Computational and Experimental Research in Mathematics in May 2024.
%
%    \vspace{.2 cm}
%
%    \item The PI was a co-organizer for conference ``Western Algebraic Geometry Symposium'' at Washington University in St. Louis in Nov 2023.
%
%    \vspace{.2cm}
%
%    \item The PI has been a co-organizer of the Algebraic and Arithmetic geometry seminar at Washington University in St. Louis starting from Sep 2022.
%
%    \vspace{.2 cm}
%
%    \item The PI was a program committee member for the Sixteenth Algorithmic Number Theory Symposium (ANTS
%XVI) in Jul 2024.
%\end{itemize}


\end{document}