\documentclass[10pt,reqno]{amsart}
\usepackage{amsfonts,amsmath,amssymb,amsbsy,amstext,amsthm}
\usepackage{accents,color,enumerate,enumitem,float,fullpage,parskip,verbatim}

\usepackage{url}
\usepackage[colorlinks=true,hyperindex, linkcolor=magenta, pagebackref=false, citecolor=cyan]{hyperref}
\usepackage[alphabetic,lite,backrefs]{amsrefs} 

\usepackage{eucal,bm,kpfonts,mathbbol}

\usepackage{tikz,tikz-cd}	
\usetikzlibrary{positioning, matrix, shapes}         								    				
\usetikzlibrary{arrows,calc,matrix}

\usepackage{lscape}

\usepackage{titlesec}		
\setcounter{secnumdepth}{4}						     					% Allows one to use nice section titles
\titleformat{\section}[block]{\large\scshape\bfseries\filcenter}{\thesection.}{1em}{}		% Creates section titles
\titleformat{\subsection}[hang]{\large\scshape\bfseries}{\thesubsection}{1em}{}			% Creates subsection titles
\titleformat{\subsubsection}[hang]{\large\scshape\bfseries}{\thesubsubsection}{1em}{}			% Creates subsection titles

\usepackage[titles]{tocloft}								     					% Creates table of fancy contents
\setcounter{tocdepth}{4}
\renewcommand{\contentsname}{}	     					% Renames and centers title of ToC

\usepackage{multirow}
\usepackage{array}
\usepackage{booktabs}
\newcolumntype{M}[1]{>{\centering\arraybackslash}m{#1}}
\newcolumntype{N}{@{}m{0pt}@{}}
\usepackage{diagbox}
\usepackage{cancel}

\newtheorem{lemma}{Lemma}[section]
\newtheorem{theorem}[lemma]{Theorem}
\newtheorem{prop}[lemma]{Proposition}
\newtheorem{cor}[lemma]{Corollary}
\newtheorem{conj}[lemma]{Conjecture}
\newtheorem{claim}[lemma]{Claim}
\newtheorem{defn}[lemma]{Definition} 
\newtheorem{notation}[lemma]{Notation} 
\newtheorem{exercise}[lemma]{Exercise}
\newtheorem{question}[lemma]{Question}
\newtheorem*{assumption}{Assumption}
\newtheorem{principle}[lemma]{Principle}
\newtheorem{heuristic}[lemma]{Heuristic}

\newtheorem{theoremalpha}{Theorem}
\newtheorem{corollaryalpha}[theoremalpha]{Corollary}
\renewcommand{\thetheoremalpha}{\Alph{theoremalpha}}

\theoremstyle{remark}
\newtheorem{remark}[lemma]{Remark}
\newtheorem{example}[lemma]{Example}
\newtheorem{cexample}[lemma]{Counterexample}

% Commands
\newcommand{\initial}{\operatorname{in}}
\newcommand{\NF}{\operatorname{NF}}
\newcommand{\HF}{\operatorname{HF}}
\newcommand{\Hilb}{\operatorname{Hilb}}
\newcommand{\depth}{\operatorname{depth}}
\newcommand{\reg}{\operatorname{reg}}
\newcommand{\Span}{\operatorname{span}}
\newcommand{\img}{\operatorname{img}}
\newcommand{\inn}{\operatorname{in}}

\newcommand{\length}{\operatorname{length}}
\newcommand{\coker}{\operatorname{coker}}
\newcommand{\adeg}{\operatorname{adeg}}
\newcommand{\pdim}{\operatorname{pdim}}
\newcommand{\Spec}{\operatorname{Spec}}
\newcommand{\Ext}{\operatorname{Ext}}
\newcommand{\Tor}{\operatorname{Tor}}
\newcommand{\LT}{\operatorname{LT}}
\newcommand{\im}{\operatorname{im}}
\newcommand{\NS}{\operatorname{NS}}
\newcommand{\Frac}{\operatorname{Frac}}
\newcommand{\Khar}{\operatorname{char}}
\newcommand{\Proj}{\operatorname{Proj}}
\newcommand{\id}{\operatorname{id}}
\newcommand{\Div}{\operatorname{Div}}
\newcommand{\Kl}{\operatorname{Cl}}
\newcommand{\tr}{\operatorname{tr}}
\newcommand{\Tr}{\operatorname{Tr}}
\newcommand{\Supp}{\operatorname{Supp}}
\newcommand{\ann}{\operatorname{ann}}
\newcommand{\Gal}{\operatorname{Gal}}
\newcommand{\Pic}{\operatorname{Pic}}
\newcommand{\QQbar}{{\overline{\mathbb Q}}}
\newcommand{\Br}{\operatorname{Br}}
\newcommand{\Bl}{\operatorname{Bl}}
\newcommand{\Kox}{\operatorname{Cox}}
\newcommand{\conv}{\operatorname{conv}}
\newcommand{\getsr}{\operatorname{Tor}}
\newcommand{\diam}{\operatorname{diam}}
\newcommand{\Hom}{\operatorname{Hom}} %done
\newcommand{\sheafHom}{\mathcal{H}om}
\newcommand{\Gr}{\operatorname{Gr}}
\newcommand{\rank}{\operatorname{rank}}
\newcommand{\codim}{\operatorname{codim}}
\newcommand{\Sym}{\operatorname{Sym}} %done
\newcommand{\GL}{{GL}}
\newcommand{\Prob}{\operatorname{Prob}}
\newcommand{\Density}{\operatorname{Density}}
\newcommand{\Syz}{\operatorname{Syz}}
\newcommand{\pd}{\operatorname{pd}}
\newcommand{\supp}{\operatorname{supp}}
\newcommand{\cone}{\operatorname{\textbf{cone}}}
\newcommand{\Res}{\operatorname{Res}}
\newcommand{\HS}{\operatorname{HS}}
\newcommand{\Cl}{\operatorname{Cl}}
\newcommand{\oO}{\operatorname{O}}

\newcommand{\defi}[1]{\textsf{#1}} % for defined terms

\newcommand{\remd}{\operatorname{remd}}
\newcommand{\colim}{\operatorname{colim}}
\newcommand{\trideg}{\operatorname{tri.deg}}
\newcommand{\indeg}{\operatorname{index.deg}}
\newcommand{\moddeg}{\operatorname{mod.deg}}
\newcommand{\Desc}{\operatorname{Desc}}
\newcommand{\inter}{\operatorname{int}}
\newcommand{\Nef}{\operatorname{Nef}}

\newcommand{\doot}{\bullet}

\newcommand{\Alt}{\bigwedge\nolimits}
\newcommand{\Set}{\text{\bf Set}}										% Category of Sets
\newcommand{\Sch}{\text{\bf Sch}}										% Category of Abelian Groups
\newcommand{\Mod}[1]{\ (\mathrm{mod}\ #1)}


\newcommand{\floor}[1]{\lfloor #1 \rfloor}
\newcommand{\ideal}[1]{\langle #1 \rangle}

%%%%%%%%%%%%%%%%%%%%%%%%%%%%%% Letters  %%%%%%%%%%%%%%%%%%%%%%%%%%%%%%%%%%%%%%%%%%%%
%%%%%%%%%%%%%%%%%%%%%%%%%%%%%%%%%%%%%%%%%%%%%%%%%%%%%%%%%%%%%%%%%%%%%%%%%%%%%%
\newcommand{\ff}{\mathbf f}
\newcommand{\kk}{\mathbf k}
\renewcommand{\aa}{\mathbf a}
\newcommand{\bb}{\mathbf b}
\newcommand{\cc}{\mathbf c}
\newcommand{\dd}{\mathbf d}
\newcommand{\ee}{\mathbf e}
\newcommand{\vv}{\mathbf v}
\newcommand{\ww}{\mathbf w}
\newcommand{\xx}{\mathbf x}
\newcommand{\yy}{\mathbf y}
\newcommand{\rr}{\mathbf r}
\newcommand{\ii}{\mathbf i}
\newcommand{\nn}{\mathbf n}
\newcommand{\pp}{\mathbf p}
\newcommand{\mm}{\mathbf m}
\newcommand{\fF}{\mathbf F}
\newcommand{\gG}{\mathbf G}
\newcommand{\eE}{\mathbf E}
\newcommand{\qQ}{\mathbf Q}
\newcommand{\tT}{\mathbf T}
\renewcommand{\tt}{\mathbf t}
\newcommand{\one}{\mathbf 1}
\newcommand{\zero}{\mathbf 0}

\renewcommand{\H}{\operatorname{H}}
\newcommand{\OO}{\operatorname{O}}
\newcommand{\oo}{\operatorname{o}}


%%%% Caligraphic Fonts - i.e. ????. %%%%%
\newcommand{\cA}{\mathcal{A}}
\newcommand{\cB}{\mathcal{B}}
\newcommand{\cC}{\mathcal{C}}
\newcommand{\cD}{\mathcal{D}}
\newcommand{\cE}{\mathcal{E}}
\newcommand{\cF}{\mathcal{F}}
\newcommand{\cG}{\mathcal{G}}
\newcommand{\cH}{\mathcal{H}} 
\newcommand{\cI}{\mathcal{I}}
\newcommand{\cJ}{\mathcal{J}}
\newcommand{\cK}{\mathcal{K}}
\newcommand{\cL}{\mathcal{L}}
\newcommand{\cM}{\mathcal{M}}
\newcommand{\cN}{\mathcal{N}}
\renewcommand{\O}{\mathcal{O}}
\newcommand{\cP}{\mathcal{P}}
\newcommand{\cQ}{\mathcal{Q}}
\newcommand{\cR}{\mathcal{R}}
\newcommand{\cS}{\mathcal{S}}
\newcommand{\cT}{\mathcal{T}}
\newcommand{\U}{\mathcal{U}} 		% Notice this is different
\newcommand{\cV}{\mathcal{V}}
\newcommand{\cW}{\mathcal{W}}
\newcommand{\cX}{\mathcal{X}}
\newcommand{\cY}{\mathcal{Y}}
\newcommand{\cZ}{\mathcal{Z}}

%%%% Blackboard Fonts - i.e. Real Numbers, Integers, etc. %%%%%
\newcommand{\A}{\mathbb{A}}
\newcommand{\B}{\mathbb{B}}
\newcommand{\C}{\mathbb{C}}
\renewcommand{\D}{\mathbb{D}}
\newcommand{\E}{\mathbb{E}}
\newcommand{\F}{\mathbb{F}}
\newcommand{\G}{\mathbb{G}}
\newcommand{\I}{\mathbb{I}}
\newcommand{\J}{\mathbb{J}}
\newcommand{\K}{\mathbb{K}}
\renewcommand{\L}{\mathbb{L}}
\newcommand{\M}{\mathbb{M}}
\newcommand{\N}{\mathbb{N}}
\newcommand{\bO}{\mathbb{O}}		% Notice this is \bO
\renewcommand{\P}{\mathbb{P}}
\newcommand{\Q}{\mathbb{Q}}
\newcommand{\R}{\mathbb{R}}
\newcommand{\T}{\mathbb{T}}
\newcommand{\bU}{\mathbb{U}}		% Notice this is \bU
\newcommand{\V}{\mathbb{V}}
\newcommand{\W}{\mathbb{W}}
\newcommand{\X}{\mathbb{X}}
\newcommand{\Y}{\mathbb{Y}}
\newcommand{\Z}{\mathbb{Z}}

 %%%% Sarif Fonts - i.e. ???? %%%%%
\newcommand{\sA}{\mathsf{A}}
\newcommand{\sB}{\mathsf{B}}
\newcommand{\sC}{\mathsf{C}}
\newcommand{\sD}{\mathsf{D}}
\newcommand{\sE}{\mathsf{E}}
\newcommand{\sF}{\mathsf{F}}
\newcommand{\sG}{\mathsf{G}}
\newcommand{\sH}{\mathsf{H}} 
\newcommand{\sI}{\mathsf{I}}
\newcommand{\sJ}{\mathsf{J}}
\newcommand{\sK}{\mathsf{K}}
\newcommand{\sL}{\mathsf{L}}
\newcommand{\sM}{\mathsf{M}}
\newcommand{\sN}{\mathsf{N}}
\newcommand{\sO}{\mathsf{O}}
\newcommand{\sP}{\mathsf{P}}
\newcommand{\sQ}{\mathsf{Q}}
\newcommand{\sR}{\mathsf{R}}
\newcommand{\sS}{\mathsf{S}}
\newcommand{\sT}{\mathsf{T}}
\newcommand{\sU}{\mathsf{U}} 
\newcommand{\sV}{\mathsf{V}}
\newcommand{\sW}{\mathsf{W}}
\newcommand{\sX}{\mathsf{X}}
\newcommand{\sY}{\mathsf{Y}}
\newcommand{\sZ}{\mathsf{Z}}
 
 %%%% Fraktur Fonts - i.e. maximal ideals, prime ideals, etc. %%%%%
\newcommand{\cl}{\mathfrak{cl}}
\newcommand{\g}{\mathfrak{g}}
\newcommand{\h}{\mathfrak{h}}
\newcommand{\m}{\mathfrak{m}}
\newcommand{\n}{\mathfrak{n}}
\newcommand{\p}{\mathfrak{p}}
\newcommand{\q}{\mathfrak{q}}
\renewcommand{\r}{\mathfrak{r}}



\newcommand{\juliette}[1]{{\color{red} \sf $\spadesuit\spadesuit\spadesuit$ Juliette: [#1]}}


\title{The Quantitative Behavior of Asymptotic Syzygies for Hirzebruch Surfaces}

\author{Juliette Bruce}
%\address{Department of Mathematics, University of Wisconsin, Madison, WI}
%\email{\href{mailto:juliette.bruce@math.wisc.edu}{juliette.bruce@math.wisc.edu}}
%\urladdr{\url{http://math.wisc.edu/~juliettebruce/}}

%\thanks{The author was partially supported by the NSF GRFP under Grant No. DGE-1256259 and NSF grant DMS-1502553.}

%\subjclass[2010]{13D02, 14M25}

\begin{document} 

%\begin{abstract}
%The goal of this note is to quantitatively study the behavior of asymptotic syzygies for certain toric surfaces, including Hirzebruch surfaces. In particular, we show that the asymptotic linear syzygies of Hirzebruch surfaces embedded by $\O(d,2)$ conform to Ein, Erman, and Lazarsfeld's normality heuristic. We also show that the higher degree asymptotic syzygies are not asymptotically normally distributed. 
%\end{abstract}

%\tableofcontents

\setcounter{section}{1}

Throughout this note for any real number $x$ we let $\floor{x}$ the integer part of $x$ and let $\ideal{x}$ denote the decimal part of $x$. Further we write $\N$ for $\Z_{\geq0}$. The goal of this note is to prove and explore the following.

\begin{claim}\label{claim:main} 
Given a real number $a_{0}>0$, define a sequence recursively by letting $a_{n+1}=\floor{a_{n}}\ideal{a_{n}}+1$ for all $n\geq 0$. For any initial value $a_{0}>0$ the sequence $\{a_{n}\}_{n\in \N}$ eventually stabilizes in the sense that there exists $N\in \N$ such that $a_{n}=a_{n+1}$ for all $n\geq N$. 
\end{claim}

Before moving on we note two key, but straightforward observations:
\begin{enumerate}
\item If $a_{k}=a_{k+1}$ for some $k\in \N$ then $a_{n}=a_{n+1}$ for all $n\geq k$, that is to say sequences constructed in the above way stabilize as soon as NEDEDD.
\item If $a_{k}=1$ for some $k\in \N$ then the sequence $\{a_{n}\}_{n\in \N}$ stabilizes and $a_{n}=1$ for all $n\geq k$. 
\end{enumerate}

With this in mind we now show that Claim~\ref{claim:main} is true in a number of special cases, in particular when $a_{0}$ is an integer or $0<a_{0}<2$. Im both of these case the sequence $\{a_{n}\}_{n\in\N}$ stabilizes to 1 because either eventually either $\floor{a_k}=0$ or $\ideal{a_{k}}=0$.

\begin{lemma}
Let $0<a_{0}$ be a real number.
\begin{enumerate}
\item If $a_{0}$ is an integer then the sequence $\{a_{n}\}$ stabilizes and $a_{n}=1$ for all $n\geq 1$.
\item If $a_{0}\leq 1$ then the sequence $\{a_{n}\}_{n\in\N}$ stabilizes and $a_{n}=1$ for all $n\geq1$. 
\item If $1<a_{0}<2$ then the sequence stabilizes and $a_{n}=a_{0}$ for all $n\geq1$.
\end{enumerate}
\end{lemma}

\begin{proof}
We begin by proving (1). If $a_{0}\in \Z_{>0}$ then the decimal part of $a_{0}$ is equal to zero (i.e., $\ideal{a_{0}}=0$). Using the definition the next term in the sequence is $a_{1}=\floor{a_{0}}\ideal{a_{0}}+1=a_{0}\cdot 0 + 1 = 1$. A similar computation shows $a_{2}=1$ and so by point (1) above our sequence stabilizes as claimed. 

Turning our attention to case (2), if $a_{0}=1$ then the claim follows from Lemma~\ref{lem:integers}, and so without lose of generality we may assume that $a_{0}<1$. Since $a_{0}<1$ the integer part of $a_{0}$ is equal to zero, and so $a_{1}=0\ideal{a_{0}}+1=1$. A similar computation shows $a_{2}=1$ and so by point (1) above our sequence stabilizes as claimed. 

Finally for case (3) note that if $1<a_{0}<2$ then $\floor{a_{0}}=1$. The next term in the sequence is then $a_{1}=1\ideal{a_{0}}+1=1+\ideal{a_{0}}$. Since $\floor{a_{0}}=1$ we can re-write this as $a_{1}=\floor{a_{0}}+\ideal{a_{0}}$, however, since $a_{0}=\floor{a_{0}}+\ideal{a_{0}}$ we get that $a_{1}=a_{0}$. Thus, by (1) above the sequence stabilizes to $a_{0}$ as claimed. 
\end{proof}


\begin{lemma}\label{lem:floor}
If $a_{0}\geq2$ is not an integer there exists a number $k\in \N$, depending on $a_{0}$, such that $a_{k} < \floor{a_{0}}$
\end{lemma}

\begin{proof}
Towards a contradiction assume that $\floor{a_{n}}=\floor{a_{0}}$ for all $n\in \N$. Consider the function $f(x)=\floor{a_{0}}\left(x-\floor{a_{0}}\right)+1$. In order to proceed we want to establish the following two properties about $f(x)$
\begin{enumerate}
\item $f(x)$ is an increasing function; and
\item $f^{(n)}(a_{0})=a_{n}$ for all $n\in \N$. 
\end{enumerate}
Here $f^{(n)}$ denotes the $n$-fold composition of $f$ with itself. Claim (i) follows from the fact that $f'(x)=\floor{a_{0}}\geq2$, with the last inequality coming from the fact that $a_{0}\geq2$. For (ii) note that since $\floor{a_{k}}=\floor{a_{0}}$ we also have that $a_{n}-\floor{a_{0}}=\ideal{a_{n}}$ for all $n\in \N$. The claim follows via induction from these facts. 

Now let us consider the sequence $\{b_{n}\}_{n\in \N}$ defined by $b_{n}=\floor{a_{0}}+1+\floor{a_{0}}^{-n}$. The key property of this sequence, which follows from direct computation, is that $f(b_{n+1})=b_{n}$ for any $n\in \N$. Now notice that $b_{0}=\floor{a_{0}}<a_{0}$ and $a_{0}< \lim_{n\to \infty}b_{n}= \floor{a_{0}}+1$, and so there exists $k\in \N$ such that 
\[
b_{k-1} \leq a_{0} < b_{k}.
\]
\end{proof}

\begin{proof}[Proof of Claim~\ref{claim:main}]
By Lemmas~\ref{NEDEDED} we know the claim is true if $a_{0}$ is an integer or $0<a_{0}<2$ respectively. Thus, without lose of generality we may assume that $a_{0}\geq 2$ and $a_{0}$ is not an integer. By Lemma~\ref{NEDEDED}  it is enough for us to show NEDEDED
\end{proof}

Turning to using Claim~\ref{claim:main} as a jumping off point for an REU-style research project; I like to approach student research as a holistic opportunity for students to explore their curiosities and develop new skills both in and out of mathematics. I feel like student research presents fantastic opportunities for students to develop skills in a number of different areas, e.g., writing, programming, art/visualization. With this in mind, I view a crucial aspect of my role in supporting student research in mathematics as carefully tailoring projects to the interests and goals of the students.

That said a natural first questions that I think may be interesting for a student to explore is studying how quickly the sequence $\{a_{n}\}_{n\in\N}$ stabilizes. In particular, since Lemma~\ref{lem:floor} is non-constructive our proof of Claim~\ref{claim:main} does not provide much insight into when the sequence converges. 

\begin{question}
Given a real number $a_{0}>0$ how quickly does the sequence $\{a_{n}\}_{n\in\N}$ stabilize?
\end{question}

There are many different ways one might answer this. For example, one could try to provide a precise formula, an upper bound,  or an asymptotic description on the smallest value of $n\in \N$ for which the sequence stabilizes. Moreover, one could attempt to student this question probabilistically by asking what the expected value of such an $n$ is. Some simple computational experiments seem to suggest these sequences stabilize relatively quickly. 


A potentially different project for a student interested in art or graphic design might be to think of a way to artistically represent these sequences and how they stabilize. 

Our work above shows that regardless of starting value $a_{0}$ the sequence $\{a_{n}\}_{n\in\N}$ will stabilize to a number $a_{N}$ where $1\leq a_{N} < 2$. Moreover, it is relatively easy to show that every number $c\in[1,2)$ there is a real number $a_{0}>0$ such that the sequence $\{a_{n}\}_{n\in\N}$ stabilizes to $c$. One question that might interest students is whether some values in this interval occur more often than others. More precisely, we might try to answer the following question

\begin{question}
What is the distribution of the 
\end{question}

Finally a project that may be out of the scope of some students, but would offer the opportunity for the students to learn some potentially interesting mathematics, would be consider the analogous problem for other fields. The jumping off point might be to consider a $p$-adic analog of Claim~\ref{claim:main} where $a_{0}\in \Q_{p}$ or more ambitiously $\overline{\Q_{p}}$. Note in this case the first hurdle would be finding appropriate ways to define the integer and decimal part of a $p$-adic number. One such approach would be to note that any number $x\in \Q_{p}$ can be written as $m+q$ where $m\in \Z_{p}$ and $|q|_{p}<1$. However, in this setting I feel like there may be other ways one might define the fractional part of a $p$-adic number. Exploring each of these different definitions, and their resulting effects on Claim~\ref{claim:main} seems interesting in it's own right. I should add the caveat that this last suggested direction feels quite speculative, and I would want to spend more time thinking about it before suggesting it to a student. 

\end{document}