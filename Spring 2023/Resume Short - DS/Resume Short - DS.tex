%%%%%%%%%%%%%%%%%%%%%%%%%%%%%%%%%%%%%%%%%
% Freeman Curriculum Vitae
% XeLaTeX Template
% Version 3.0 (September 3, 2021)
%
% This template originates from:
% https://www.LaTeXTemplates.com
%
% Authors:
% Vel (vel@LaTeXTemplates.com)
% Alessandro Plasmati
%
% License:
% CC BY-NC-SA 4.0 (https://creativecommons.org/licenses/by-nc-sa/4.0/)
%
%!TEX program = xelatex
% NOTE: this template must be compiled with XeLaTeX rather than PDFLaTeX
% due to the custom fonts used. The line above should ensure this happens
% automatically, but if it doesn't, your LaTeX editor should have a simple toggle
% to switch to using XeLaTeX.
% 
%%%%%%%%%%%%%%%%%%%%%%%%%%%%%%%%%%%%%%%%%

%----------------------------------------------------------------------------------------
%	PACKAGES AND OTHER DOCUMENT CONFIGURATIONS
%----------------------------------------------------------------------------------------

\documentclass[
	10pt, % Default font size, can be between 8pt and 12pt
]{FreemanCV}

\columnratio{0.55, 0.45} % Widths of the two columns, specified here as a ratio summing to 1 to correspond to percentages; adjust as needed for your content 

% Headers and footers can be added with the following commands: \lhead{}, \rhead{}, \lfoot{} and \rfoot{}
% Example right footer:
%\rfoot{\textcolor{headings}{\sffamily Last update: \today. Typeset with Xe\LaTeX}}

%----------------------------------------------------------------------------------------

\usepackage{enumitem}

\begin{document}

\begin{paracol}{2} % Begin two-column mode

%----------------------------------------------------------------------------------------
%	YOUR NAME AND CURRICULUM VITAE TITLE
%----------------------------------------------------------------------------------------

\parbox[][0.11\textheight][c]{\linewidth}{ % Box to hold your name and CV title; change the fixed height as needed to match the colored box to the right
	\centering % Horizontally center text
	
	{\sffamily\Huge Juliette Bruce, Ph.D.} % Your name
	
	\medskip % Vertical whitespace
	
%    	{\cursivefont\Huge\textcolor{headings}{Curriculum Vitae}}
	
	\vfill % Push content to the top of the box
}

%----------------------------------------------------------------------------------------
%	MAJOR RESEARCH PROJECT
%----------------------------------------------------------------------------------------
\vspace{-.5cm}
\section{Professional Highlights}

\textbf{I am now excited to apply my mathematical and programming skills to build models that address challenging problems. } My research in mathematics brings together computational, combinatorial, and algebraic tools to study the geometry of zero loci of systems of polynomials. 
\begin{itemize}[leftmargin=*]
\item Co-authored 15 research articles that have been, or are submitted, for publication in top peer-reviewed journals. 
%\item See  \href{https://www.juliettebruce.xyz/publications}{www.juliettebruce.xyz/publications} for a complete publication list.
\item Awarded over \$300,000 in research grants including the extremely competitive NSF Postdoctoral Research Fellowship.% and NSF Graduate Research Fellowship. 
\end{itemize}
%More specifically, I am a leading early career research in algebraic geometry, commutative algebra, and computational algebra. I have co-authored 15 research articles that have been peer-reviewed and published or are in the process of being reviewed and published.

% the use of ELW pulses from a mode-locked source array inducted through transuranic crystals to observe entanglement on supraquantum structures. Theoretical advancements included prediction of quantum resonance phenomena including the possibility of resonance cascades. I was motivated to conduct this doctoral research due to my passion for teleportation of matter and I believe I have laid the foundation for further experimental validation and development of practical outcomes.

\medskip % Extra vertical whitespace before the next section

%----------------------------------------------------------------------------------------
%	WORK EXPERIENCE
%----------------------------------------------------------------------------------------

\section{Work Experience}

% Each job is added with a \jobentry command. Below is an empty one to use as a template:

%\jobentry
%	{} % Duration
%	{} % FT/PT (full time or part time)
%	{} % Employer
%	{} % Job title
%	{} % Description

% All 5 parameters must be supplied but any can be empty if you don't need them

%------------------------------------------------

\jobentry
	{Aug. 2022 - Present} % Duration
	{FT} % FT/PT (full time or part time)
	{Brown University} % Employer
	{Postdoctoral Research Associate} % Job title
	{Established myself as a leading early-career researcher, with a strong commitment to mentorship and developing cross-field collaborations. 
	\begin{itemize}[leftmargin=*]
	\item Served as a member of the Software Presentation Committee for the International Symposium for Symbolic and Computational Algebra.
	\end{itemize}
	} % Description

%------------------------------------------------

\jobentry
	{Aug. 2020 -- Jul. 2022} % Duration
	{FT} % FT/PT (full time or part time)
	{University of California, Berkeley} % Employer
	{NSF Postdoctoral Research Fellow} % Job title
	{Developed a successful independent research program that brought together ideas from numerous areas of mathematics.
	\begin{itemize}[leftmargin=*]
	\item Mentored several graduate and undergraduate students projects resulting in multiple research articles submitted for publication.  
	\end{itemize}
	  } % Description

%------------------------------------------------

\jobentry
	{Aug. 2014 -- Jul. 2020} % Duration
	{PT} % FT/PT (full time or part time)
	{University of Wisconsin, Madison} % Employer
	{Graduate Assistant} % Job title
	{Teaching Assistant for 5 semesters and Research Assistant for 7 semesters the Mathematics Department.
	\begin{itemize}[leftmargin=*]
	\item Developed interactive teaching materials on a weekly basis, and oversaw a team of  other TA's as a head TA. \vspace{-.4em}
	\item Received the highest departmental and campus-wide awards for teaching: Capstone Teaching Award (2019) and  Teaching Assistant Award for Exceptional Service (2018). \vspace{-.4em}
	\item Received the Excellence in Mathematical Research Award (2019) for significant and substantial contributions to research.
	\end{itemize}
	} % Description


\section{Programming Experience} 

% This section is laid out using a table. A \tableentry command adds lines with the following parameters:

%\tableentry{Heading}{Content}{spaceafter}
% All 3 parameters must be supplied but any can be empty if you don't need them
% A "spaceafter" value in the third parameter will add some vertical space -- this is to be used between headings, leave it empty for no extra space

%------------------------------------------------

\begin{supertabular}{r l} % Start a table with two columns, the table will ensure everything is aligned
	
	%------------------------------------------------
	
	\tableentry{Beginner}{HTML/CSS, Pandas, SQL}{spaceafter}
	
	%------------------------------------------------
	
	\tableentry{Intermediate}{Python, Matlab}{spaceafter}
%    	\tableentry{}{Microsoft Windows}{}
%    	\tableentry{}{Computer Hardware \& Support}{spaceafter}
	
	%------------------------------------------------
	
	\tableentry{Expert}{Macaulay2, Latex}{spaceafter}
	
	%------------------------------------------------
	
\end{supertabular}
%\section{Awards}
%
%% This section is laid out using a table. A \tableentry command adds lines with the following parameters:
%
%%\tableentry{Heading}{Content}{spaceafter}
%% All 3 parameters must be supplied but any can be empty if you don't need them
%% A "spaceafter" value in the third parameter will add some vertical space -- this is to be used between headings, leave it empty for no extra space
%
%%------------------------------------------------
%
%\begin{supertabular}{r l} % Start a table with two columns, the table will ensure everything is aligned
%	
%	%------------------------------------------------
%	
%	\tableentry{2019}{\textbf{Capstone Teaching Award}}{}
%	\tableentry{}{\textit{University of Wisconsin, Madison}}{spaceafter}
%	
%	%------------------------------------------------
%	
%	\tableentry{2019}{\textbf{Excellence in Mathematical Research Award}}{}
%	\tableentry{}{\textit{TUniversity of Wisconsin, Madison}}{spaceafter}
%	
%	%------------------------------------------------
%	
%	\tableentry{2019}{\textbf{Excellence in Mathematical Research Award}}{}
%	\tableentry{}{\textit{TUniversity of Wisconsin, Madison}}{spaceafter}
%	
%	%------------------------------------------------
%\end{supertabular}

%\section{Select Projects}
%
%\subsection{Computational Algebra Packages for Macaulay2}
%
%Developed four peer-reviewed software packages extending the functionality of the open-source computer algebra software Macaulay2. These packages are  included (or are scheduled to be included) with the standard distribution of Macaulay2. 
%
%\subsection{Computing Algebraic Invariants }
%
%Led a collaborative research project that brought together tools from numerical linear algebra, high throughput computing, and homological algebra to develop novel approaches to computing certain algebraic invariants (syzygies) of geometry objects. This project has resulted in three peer-review research articles, and has
%
%\subsection{Exploring Trends in News Coverage of Science}
%
%Utilizing Python I created and analyzed a database to explore how Quanta covers different areas of science and mathematics by looking at which preprints are cited.


%----------------------------------------------------------------------------------------
%	REFERENCES
%----------------------------------------------------------------------------------------
%
%\section{References}

%\textit{References available on request} % Uncomment if you'd rather not include references and remove the section below

%------------------------------------------------

% This section is laid out using a table. A \tableentry command adds lines with the following parameters:

%\tableentry{Heading}{Content}{spaceafter}
% All 3 parameters must be supplied but any can be empty if you don't need them
% A "spaceafter" value in the third parameter will add some vertical space -- this is to be used between headings, leave it empty for no extra space

%------------------------------------------------

%\begin{supertabular}{r l} % Start a table with two columns, the table will ensure everything is aligned
%	
%	%------------------------------------------------
%	
%	\tableentry{}{\textbf{Dr. Isaac Kleiner}}{spaceafter}
%	\tableentry{Position}{Professor}{}
%	\tableentry{Employer}{\href{https://web.mit.edu/physics/}{Department of Physics}}{}
%	\tableentry{}{\href{https://web.mit.edu}{\textit{Massachusetts Institute of Technology}}}{spaceafter}
%	\tableentry{Phone}{+1 (617) 253 1000 x5322 (Work)}{}
%	\tableentry{Mobile}{+1 (232) 842-3583}{}
%	
%	%------------------------------------------------
%	
%	\\ % Additional vertical whitespace between the references
%	
%	%------------------------------------------------
%	
%	\tableentry{}{\textbf{Dr. Eli Vance}}{spaceafter}
%	\tableentry{Position}{Scientist (HL1)}{}
%	\tableentry{Employer}{\href{http://www.bmrf.us}{Black Mesa Research Facility}}{spaceafter}
%	\tableentry{Email}{\href{mailto:e.vance@bmrf.us}{e.vance@bmrf.us}}{}
%	\tableentry{Phone}{+1 (800) 786-1410 x6235 (Work)}{}
%	\tableentry{Mobile}{+1 (201) 632-3901}{}
%	
%	%------------------------------------------------
%	
%\end{supertabular}

\medskip % Extra vertical whitespace before the next section

%----------------------------------------------------------------------------------------

\switchcolumn % Switch to the second (right) column

%----------------------------------------------------------------------------------------
%	COLORED CONTACT DETAILS BOX
%----------------------------------------------------------------------------------------

\parbox[top][0.11\textheight][c]{\linewidth}{ % Box to hold the colored box; change the fixed height as needed to match the box to the left
	\colorbox{shade}{ % Create colored box and specify background color
		\begin{supertabular}{@{\hspace{3pt}} p{0.05\linewidth} | p{0.775\linewidth}} % Start a table with two columns, the table will ensure everything is aligned
%			\raisebox{-1pt}{\faHome} & 990 Duncan St G207, San Francisco, CA 94131 \\ % Address
			% Phone number
			\raisebox{-1pt}{\small\faEnvelope} & \href{mailto:juliette\_bruce1@brown.edu}{juliette\_bruce\textsf{1}@brown.edu} \\ % Email address
			\raisebox{-1pt}{\faGithub} & \href{https://github.com/juliettebruce}{https://github.com/juliettebruce} \\
			\raisebox{-1pt}{\small\faDesktop} & \href{https://www.juliettebruce.xyz}{www.juliettebruce.xyz} \\ 
			\raisebox{-1pt}{\faPhone} & +1 (810) 623-7610 % Website
			%\raisebox{-1pt}{\faGithub} & \href{https://github.com/username}{https://github.com/username} \\ % GitHub profile
			%\raisebox{-1pt}{\faLinkedinSquare} & \href{https://www.linkedin.com/in/username}{https://www.linkedin.com/in/username} \\ % LinkedIn profile
			% See fontawesome.pdf in the Fonts folder for all icons you can use
		\end{supertabular} 
	}
	\vfill % Push content to the top of the box
}

%----------------------------------------------------------------------------------------
%	EDUCATION
%----------------------------------------------------------------------------------------

\vspace{-.5cm}

%----------------------------------------------------------------------------------------
%	AWARDS
%----------------------------------------------------------------------------------------

%\section{Projects}
%
%\subsection{Computational Algebra Packages for Macaulay2}
%
%Developed four peer-reviewed software packages extending the functionality of the open-source computer algebra software Macaulay2. These packages are  included (or are scheduled to be included) with the standard distribution of Macaulay2. 
%
%\subsection{Computing Algebraic Invariants }
%
%Led a collaborative research project that brought together tools from numerical linear algebra, high throughput computing, and homological algebra to develop novel approaches to computing certain algebraic invariants (syzygies) of geometry objects. This project has resulted in three peer-review research articles, and has
%
%\subsection{Exploring Trends in News Coverage of Science}
%
%


%----------------------------------------------------------------------------------------
%	COMPUTER SKILLS
%----------------------------------------------------------------------------------------
%
%\section{Programming Experience} 
%
%% This section is laid out using a table. A \tableentry command adds lines with the following parameters:
%
%%\tableentry{Heading}{Content}{spaceafter}
%% All 3 parameters must be supplied but any can be empty if you don't need them
%% A "spaceafter" value in the third parameter will add some vertical space -- this is to be used between headings, leave it empty for no extra space
%
%%------------------------------------------------
%
%\begin{supertabular}{r l} % Start a table with two columns, the table will ensure everything is aligned
%	
%	%------------------------------------------------
%	
%	\tableentry{Beginner}{HTML/CSS}{spaceafter}
%	
%	%------------------------------------------------
%	
%	\tableentry{Intermediate}{Python, Matlab}{spaceafter}
%%    	\tableentry{}{Microsoft Windows}{}
%%    	\tableentry{}{Computer Hardware \& Support}{spaceafter}
%	
%	%------------------------------------------------
%	
%	\tableentry{Expert}{Macaulay2, Latex}{spaceafter}
%	
%	%------------------------------------------------
%	
%\end{supertabular}

%----------------------------------------------------------------------------------------
%	COMMUNICATION SKILLS
%----------------------------------------------------------------------------------------

%\section{Communication Skills}
%
%% This section is laid out using a table. A \tableentry command adds lines with the following parameters:
%
%%\tableentry{Heading}{Content}{spaceafter}
%% All 3 parameters must be supplied but any can be empty if you don't need them
%% A "spaceafter" value in the third parameter will add some vertical space -- this is to be used between headings, leave it empty for no extra space
%
%%------------------------------------------------
%
%\begin{supertabular}{r l} % Start a table with two columns, the table will ensure everything is aligned
%	
%	%------------------------------------------------
%	
%	\tableentry{Conferences}{Oral Presentation at the Annual MIT}{}
%	\tableentry{}{Theoretical Physics Conference -- 1987}{spaceafter}
%	
%	%------------------------------------------------
%	
%	\tableentry{Posters}{Poster at the Meeting of the American}{}
%	\tableentry{}{Physical Society -- 1985}{spaceafter}
%	
%	%------------------------------------------------
%	
%\end{supertabular}

%----------------------------------------------------------------------------------------
%	SKILLS DESCRIPTION
%----------------------------------------------------------------------------------------

\section{Select Projects}

\subsection{Computational Algebra Packages for Macaulay2}

Developed four peer-reviewed software packages extending the functionality of the open-source computer algebra software Macaulay2. These packages are  included (or will be included) with the standard distribution of Macaulay2. 

\subsection{Computing Algebraic Invariants }

Led a collaborative research project that brought together tools from numerical linear algebra and high throughput computing to develop novel approaches to computing syzygies. Created a website to make the data available to other researchers.  %Resulted in three peer-review articles and a public database. 

\subsection{Foundations of AI/ML in Computer Algebra}

An ongoing project to develop robust datasets to allow the development of artificial intelligence/machine learning methods in computational algebraic geometry.  

\section{Skills}

\subsection{Event Organizing}

\begin{itemize}[leftmargin=*]
\item Organized 10 research conferences ranging from narrowly focused events with 20 participants to large international conferences with over 100 participants.
\end{itemize}

\subsection{Technical \& Non-Technical Communication}

\begin{itemize}[leftmargin=*]
\item Gave over 75 invited research presentations at national and international conferences and seminars, including: Harvard, Princeton, Stanford, UC Berkeley, and UT Austin. \vspace{-.4em}
\item Gave 25 general audience talks aimed at promoting mathematics to the public. 
\end{itemize}

\subsection{Leadership}

\begin{itemize}[leftmargin=*]
\item As lead organizer (2016-2018) for the Madison Math Circle, created new programming and community initiatives that increased attendance from 25 to 250 students per year. 
 \vspace{-.4em}
\item As the inaugural president (2022) for Spectra, the association for LGBTQ+ mathematicians, and oversaw a fundraiser that raised over \$20,000. 
\end{itemize}

\section{Education} 

% Each qualification entry is added with a \qualificationentry command. Below is an empty one to use as a template:

%\qualificationentry
%	{} % Duration
%	{} % Degree
%	{} % Honors, achievements or distinctions (e.g. first class honors)
%	{} % Department
%	{} % Institution

% All 5 parameters must be supplied but any can be empty if you don't need them

%------------------------------------------------

\begin{supertabular}{r l} % Start a table with two columns, the table will ensure everything is aligned

	%------------------------------------------------
	
	\qualificationentry
		{2014 -- 2020} % Duration
		{Ph.D. in Mathematics} % Degree
		{} % Honors, achievements or distinctions (e.g. first class honors)
%		{Theoretical Physics} % Department
		{University of Wisconsin, Madison} % Institution
	
	%------------------------------------------------
	
	\qualificationentry
		{2014 - 2016} % Duration
		{M.A. in Mathematics}
		{} % Degree
%		{First Class Honors} % Honors, achievements or distinctions (e.g. first class honors)
%		{Theoretical Physics} % Department
		{University of Wisconsin,  Madison} % Institution
	
	%------------------------------------------------
	
	\qualificationentry
		{2010 -- 2014} % Duration
		{B.S. in Mathematics \& Political Science} % Degree
		{with High Honors \& Distinction} % Honors, achievements or distinctions (e.g. first class honors)
%		{Department of Physics} % Department
		{University of Michigan} % Institution
	
	%------------------------------------------------

\end{supertabular}

%\subsection{Exploring Trends in News Coverage of Science}
%
%Using Python I created and analyzed a database to explore how Quanta covers different areas of science and mathematics by looking at which preprints are cited.

%----------------------------------------------------------------------------------------
%	PUBLICATIONS
%----------------------------------------------------------------------------------------

%\section{Publications}
%
%%------------------------------------------------
%
%\textbf{Freeman, G. R.} (1996). Chemistry of Multiply Charged Negative Molecular Ions and Clusters in the Gas Phase:  Terrestrial and in Intense Galactic Magnetic Fields. \textit{The Journal of Physical Chemistry}, \textit{100}(11), 4331-4338.
%
%\medskip % Vertical whitespace
%
%Jacobsen, F. M., Gee, N., \textbf{Freeman, G. R.} (1986). Electron mobility in liquid krypton as function of density, temperature, and electric field strength. \textit{Physical Review A}, \textit{34}(3): 2329-2335.
%
%\medskip % Vertical whitespace
%
%%------------------------------------------------
%
%% As an alternative to a long-form publication list, you can create a shorter summary using only DOI values and years.
%
%% Example \doipublication{} command to add another publication:
%
%%\doipublication{Year}{DOI}{firstauthor}{spaceafter}
%
%% All four parameters are required (can be empty though)
%% A value of "firstauthor" in the third parameter will output the DOI in bold
%% A "spaceafter" value in the fourth parameter will add some vertical space -- this is to be used between years
%
%%------------------------------------------------
%
%\subsection{Publications by DOI}
%
%\begin{supertabular}{r l} % Start a table with two columns, the table will ensure everything is aligned
%	
%	%------------------------------------------------
%	
%	\doipublication{1996}{10.1021/jp951483+}{firstauthor}{spaceafter}
%	
%	%------------------------------------------------
%	
%	\doipublication{1990}{10.1139/p90-097}{firstauthor}{spaceafter}
%	
%	%------------------------------------------------
%	
%	\doipublication{1986}{10.1139/v86-297}{}{}
%	\doipublication{}{10.1103/PhysRevA.34.2329}{}{spaceafter}
%	
%	%------------------------------------------------
%	
%	& \textit{First author publications in} \textbf{bold}\\
%	
%	%------------------------------------------------
%	
%\end{supertabular}
%
%\medskip % Extra whitespace before the next section

%----------------------------------------------------------------------------------------

\end{paracol} % End two-column mode

%----------------------------------------------------------------------------------------

\end{document}
