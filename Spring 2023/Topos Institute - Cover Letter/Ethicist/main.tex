%
% brownletter_example.tex - an example latex file to illustrate brownletter.cls
%
% Copyright 2003, Nesime Tatbul (tatbul@cs.brown.edu)
%

\documentclass[11pt]{brownletter}

\usepackage{amsfonts,amsmath,amssymb,amsbsy,amstext,amsthm,mathtools}

\newcommand{\C}{\mathbb{C}}
\newcommand{\PP}{\mathbb{P}}


\name{Nesime Tatbul} % used as signature, if no signature is specified

\signature{Juliette Bruce}

\date{April 24, 2022} % if no date specified, today's date is used 

\subject{Application for Research Ethicist Position} % optional subject line

\begin{document}

\begin{letter}{Juliette Bruce,\\ 
               Brown University\\ 
               Box 1917\\ 
               Providence, RI 02912}

\opening{Dear Brendan Fong,}

I am writing to apply for the position as a Research Ethicist at the Topos Institute. Currently, I am a Postdoctoral Research Association in the Mathematics Department at Brown University, a position I have held since August 2022. I received my Ph.D. in Mathematics from the University of Wisconsin - Madison in the Summer of 2020 under the guidance of my advisor Professor Daniel Erman. Previously I was an NSF Postdoctoral Research Fellow in the Mathematics Department at the University of California, Berkeley, and for the 2020-2021 academic year, I was a Postdoctoral Fellow at the Mathematical Sciences Research Institute in Berkeley, CA. My mathematical research interests have been in algebraic geometry, commutative algebra, and arithmetic geometry. In particular, I am interested in using computational, combinatorial, and homological methods to study the geometry of zero loci of systems of polynomials (i.e. algebraic varieties).  

Despite my formal education and work having been in research mathematics, throughout my career I have been passionate about promoting inclusivity, diversity, and justice in the mathematics community and beyond. This passion has led me to work on a number of projects, initiatives, and organzaitons that seek to promote best practices within the mathematical community, support minoritized mathematicians, and explore was mathematics and mathematicians can ethically engage with the broader world. A few examples of my work include: 

\begin{itemize}
\item Working with a number of national professional societies -- including the AMS and AWM -- to develop better practices and policies for supporting LGBTQ+ and other minoritized mathematicians. This work has led a number of concrete policy changes including for example to how the AMS handles its name policies for all of its journals. 
\item Served on the organizing board for Spectra, the association for LGBTQ+ mathematicians since 2020, including serving as their inaugural president in 2022. In this position, I led the creation of a number of new outreach programs  for LGBTQ+ mathematicians and students, oversaw a fundraising effort that more than tripled the organization's funds,  and worked to establish the culture within Spectra as it moves through the early stages of growth as an organization. 
\item Organized numerous conferences and events designed to support the needs of minoritized mathematicians and students. This has included,  Trans Math Days, a yearly one-day virtual conference for transgender mathematicians and scientists, which has run since 2020 often averaging over 70 participants, Gender Equity in the Mathematical Study of Combinatorics, a multi-conference effort running since 2021 to promote gender equity in combinatorics and for ways the combinatorics community can engage in promoting justice in the broader world.  
\item Invited to speak and facilitate discussions at multiple departments, universities, and organizations on ways these groups can better serve and support minoritized communities by changing their policies, practices, and culture. 
\end{itemize}

Moreover, through this work, I have had the privilege and excitement of working informally with a number of mathematicians on ways to make their research and teaching practices more ethical and promote diversity and justice. I am now interested in using the skills and knowledge I have developed in this work to help shape the way the scientific community approaches ethics, justice, and inclusion in a larger way. With this in mind, I would be extremely excited to work to shape the ethical culture Topos Institute and by extension the ethical culture of the mathematical and scientific communities. 


With my application, I have included my curriculum vitae. In addition to the initiatives and work listed above a few highlights of my file include:
\begin{itemize}
%\item Invited to speak at the \textit{Western Algebraic Geometry Online} (April 2020), \textit{Foundations of Computing in Mathematics} (June 2020), and \textit{LGBTQ+Math} (July 2020).

\item Excellence in Mathematical Research Award (2020) - Awarded by the Department of Mathematics at the University of Wisconsin for substantial contributions to research as part of a thesis. %Elizabeth Hirschfelder Prize (2019) - Awarded by the Department of Mathematics at the University of Wisconsin to an outstanding female student who has demonstrated promise.

\item Co-organizer for seven conferences, two AMS special sessions, and numerous outreach events. Awarded approximately \$55,000 in grants to support these conferences.  

\item Capstone Teaching Award (2019) - Awarded by the Department of Mathematics at the University of Wisconsin to 1 student recognizing a record of exceptional teaching and service. Teaching Assistant Award for Exceptional Service  (2018) - Awarded by the University of Wisconsin to up to 3 teaching assistants campus-wide recognizing their exceptional service to the community. 
\end{itemize}

Finally, I will note that while currently employed by Brown University my permanent residence is in the Bay Area. Please do not hesitate to contact me with any questions, and thank you for your consideration.

\closing{Sincerely,}

% \encl{brownletter.cls}

% \ps{Please see the enclosed file.}

% \cc{J. Kirschenbaum}

\end{letter}

\end{document}