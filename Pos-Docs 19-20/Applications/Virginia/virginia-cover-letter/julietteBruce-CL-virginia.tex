\documentclass[11pt]{article}
\usepackage{amsfonts,amsmath,amssymb,amsbsy,amstext,amsthm,mathtools}

\usepackage{graphicx}
\usepackage{color}
\usepackage[hidelinks]{hyperref}
\usepackage{fullpage}
\setlength{\topmargin}{-1.0in}
\setlength{\footskip}{0.5in}
\setlength{\textwidth}{6.5in}
\setlength{\textheight}{10.0in}
\setlength{\oddsidemargin}{-0.0in}
\setlength{\evensidemargin}{-0.0in}
\usepackage{parskip}
\usepackage{fancyhdr}
\pagestyle{fancy}
\addtolength{\headheight}{\baselineskip}
\addtolength{\headheight}{1in}
\addtolength{\textheight}{-\baselineskip}
\addtolength{\textheight}{-1in}
\lhead{}
\chead{\includegraphics[height=1in]{color-center-UWlogo-print}}
\rhead{}
\lfoot{}
\cfoot{{\footnotesize
       Department of Mathematics \\
       University of Wisconsin--Madison \hspace{0.1in} 480 Lincoln Dr \hspace{0.1in} Madison, Wisconsin 53706 \\
       Phone: (608) 263--3054 \hspace{0.25in}
       Fax: (608) 263--8891 \hspace{0.25in}
       Web: http://www.math.wisc.edu
       }}
\rfoot{}
\renewcommand{\headrulewidth}{0pt}

\begin{document}

\section*{}

\noindent
\begin{minipage}{0.99\textwidth}
\begin{minipage}{0.69\textwidth}
\textcolor{white}{.}
\end{minipage}
\begin{minipage}{0.29\textwidth}
{
Juliette Bruce \\
Graduate Student \\
Department of Mathematics \\
\href{mailto:juliette.bruce@math.wisc.edu}{juliette.bruce@math.wisc.edu}
%\url{juliettebruce.github.io}
%Phone: (810)--623--7610 
}

\vspace{12pt}
\today
\end{minipage}
\end{minipage}

%\vspace{12pt}
%\noindent
%John Doe \\
%Business, Inc. \\
%1234 Address Avenue \\
%City, State ZZZIP

\vspace{12pt}
\noindent
To the Hiring Committee,

My name is Juliette Bruce, and I am a graduate student at the University of Wisconsin - Madison, working in algebraic geometry, commutative algebra, and arithmetic geometry under the guidance of my advisor Professor Daniel Erman. I expect to receive my Ph.D. in Mathematics from the University of Wisconsin - Madison in the Spring of 2020. I am writing to apply for the Postdoctoral Research Associate position at the University of Virginia. I would be especially interested in working with Professor Craig Huneke. 

As described in more depth in my research statement, I am interested in studying a number of questions in commutative algebra, algebraic geometry, and arithmetic geometry. Two projects involve studying the geometry of algebraic varieties via homological commutative algebra. A third project lies in the intersection of algebraic geometry and arithmetic geometry and seeks to extend classical results in algebraic geometry to finite fields. 

My first project, seeks to expand our understanding of syzygies of algebraic varieties from both the asymptotic and computational perspectives. In recent years substantial work has been done studying the syzygies of algebraic varieties as the positivity of the embedding grows. This project builds upon my thesis and seeks to extend these results by weakening the positivity conditions previously considered. In particular, the I will explore the asymptotic syzygies of varieties in the setting of semi-ample growth. Further, by utilizing new advances in high-performance computing and numerical linear algebra this project will generate new data regarding the syzygies of Hirzebruch surfaces. This data will be publicly disseminated via the online syzygy database \textit{syzygydata.com}, which I maintain.  

A second project will deepen our understanding of curves in toric 3-folds, for example, curves in $\mathbb{P}^1\times\mathbb{P}^2$, by generalizing existing results about the liaison theory of curves in $\mathbb{P}^3$. This project will make use of recent developments in the homological properties of varieties embedded in spaces other than $\mathbb{P}^n$. This project has the potential to yield applications to the study of toric varieties and toric vector bundles, as well as potential applications to unirationality of $\mathcal{M}_g$ or other moduli spaces (similar to~[BS15, Theorem 4.5]).

A third project I propose builds upon recent work generalizing classical Bertini Theorems in algebraic geometry to the setting of finite fields. More specifically, this project hopes to prove more general and uniform Bertini Theorems over finite fields. Such results will shed light on both the geometry and arithmetic of varieties of finite fields, and potentially prove useful in studying things like rational points on varieties. 
\\


As a graduate student at the University of Wisconsin - Madison, I served as a teaching assistant and course coordinator for Calculus I for multiple semesters, the instructor of record for math for early education majors, and the instructor of record for a course providing students from generally under-represented groups additional support during their first college math course. Additionally, for several semesters, I held a non-traditional teaching assistantship for my role as the organizer of the Madison Math Circle outreach program. My passion for promoting an interest in and excitement for math -- especially for people from generally underrepresented groups -- led me to take on teaching and outreach roles through the \textit{Girls Math Night Out} and the \textit{Wisconsin Directed Reading Program}. 

My teaching has recognized through both awards and student evaluations:
\begin{itemize}
\item In 2018, I was one of three graduate students recognized campus-wide with the Teaching Assistant Award for Exceptional Service.
\item I received two TA awards from the math department, the TA Service Award (2018) and the Capstone Teaching Award (2019), the latter of which is awarded to just one student each year, for an exceptional record of teaching excellence and service.
\item My student evaluations are generally quite high; for instance, in my most recent course, 100\% of students agreed that I was an effective teacher.
\end{itemize}
%A few highlights in my file are:
%\begin{itemize}
%\item Invited to speak at \textit{CA+} (April 2020), \textit{Foundations of Computing in Mathematics} (June 2020), and \textit{LGBTQ+Math} (July 2020).
%
%\item Elizabeth Hirschfelder Prize (2019) - Awarded by the Department of Mathematics at the University of Wisconsin to an outstanding female student who has demonstrated promise.
%
%\item Co-organizer for three conferences (\textit{M2@UW}, \textit{Geometry \& Arithmetic of Surfaces}, \textit{Graduate Workshop in Commutative Algebra for Women and Mathematicians of Minority Genders}), a AMS special session, and a Spectra panel. 
%
%\item Teaching Assistant Award for Exceptional Service  (2018) - Awarded by the University of Wisconsin to up to 3 teaching assistants campus-wide recognizing their exceptional service.
%\end{itemize}

With my application, I include the standard AMS cover sheet, a curriculum vitae, a research statement, a teaching statement, and a publication list. I will have five letters of recommendation. Four research letters: Christine Berkesch (\href{mailto:cberkesc@umn.edu}{cberkesc@umn.edu}), David Eisenbud (\href{mailto:de@msri.org}{de@msri.org}), Daniel Erman (\href{mailto:derman@math.wisc.edu}{derman@math.wisc.edu}), and Kiran Kedlaya (\href{mailto:kedlaya@ucsb.edu}{kedlaya@ucsd.edu}), and one teaching letter from Shirin Malekpour (\href{mailto:shirin.malekpour@wisc.edu}{shirin.malekpour@wisc.edu}).

Please do not hesitate to contact me with any questions, or if there is anything else I can provide, and thank you in advance for your consideration. 

\vspace{24pt}
\noindent
\begin{minipage}{0.99\textwidth}
\begin{minipage}{0.69\textwidth}
\textcolor{white}{.}
\end{minipage}
\begin{minipage}{0.29\textwidth}
Sincerely, 

\vspace{36pt}
Juliette Bruce\\
Graduate Student
\end{minipage}
\end{minipage}

% This command changes the page style to plain from this page onward.
% If your letter is 1 page long, then comment this out.
% If your letter is 2 pages long, then include this command.
% If your letter is longer than 2 pages, then you may need to place
% this command earlier in the document.
%\pagestyle{plain}

\end{document}