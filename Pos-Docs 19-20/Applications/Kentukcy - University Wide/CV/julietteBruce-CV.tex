% (c) 2002 Matthew Boedicker <mboedick@mboedick.org> (original author) http://mboedick.org
% (c) 2003-2007 David J. Grant <davidgrant-at-gmail.com> http://www.davidgrant.ca
% (c) 2008-2014 Nathaniel Johnston <nathaniel@njohnston.ca> http://www.njohnston.ca
%
% Depending on your TeX distribution, you may need to download the revnum and longtable packages for this template to work!
%
%This work is licensed under the Creative Commons Attribution-Noncommercial-Share Alike 2.5 License. To view a copy of this license, visit http://creativecommons.org/licenses/by-nc-sa/2.5/ or send a letter to Creative Commons, 543 Howard Street, 5th Floor, San Francisco, California, 94105, USA.

\documentclass[letterpaper,11pt]{article}
\newlength{\outerbordwidth}
\usepackage{amstext,amsfonts,amssymb,amscd,amsbsy,amsmath}
\pagestyle{empty}
\raggedbottom
\raggedright
\usepackage{array}
\usepackage[svgnames]{xcolor}
\usepackage{enumerate}
\usepackage{framed}
\usepackage{longtable}
\usepackage{revnum}
\usepackage{textcomp}

\usepackage[colorlinks=true,urlcolor=blue]{hyperref}
\usepackage{tocloft}
\usepackage{array}
\usepackage{marvosym} % For cool symbols.



%-----------------------------------------------------------
%Edit these values as you see fit

\setlength{\outerbordwidth}{3pt}  % Width of border outside of title bars
\definecolor{shadecolor}{gray}{0.75}  % Outer background color of title bars (0 = black, 1 = white)
\definecolor{shadecolorB}{gray}{0.93}  % Inner background color of title bars


%-----------------------------------------------------------
%Margin setup

\setlength{\evensidemargin}{-0.25in}
\setlength{\headheight}{0in}
\setlength{\headsep}{0in}
\setlength{\oddsidemargin}{-0.25in}
\setlength{\paperheight}{11in}
\setlength{\paperwidth}{8.5in}
\setlength{\tabcolsep}{0in}
\setlength{\textheight}{9.5in}
\setlength{\textwidth}{7in}
\setlength{\topmargin}{-0.3in}
\setlength{\topskip}{0in}
\setlength{\voffset}{0.1in}
\setlength\LTleft{0.2in} % needed to make longtable full-width
\setlength\LTright{0.2in}

%-----------------------------------------------------------
%Custom commands
\newcommand{\resitem}[1]{\item #1 \vspace{-2pt}}
\newcommand{\resheading}[1]{\vspace{8pt}
  \parbox{\textwidth}{\setlength{\FrameSep}{\fboxsep}
    \begin{shaded}
\setlength{\fboxsep}{0pt}\framebox[\textwidth][l]{\setlength{\fboxsep}{4pt}\fcolorbox{shadecolorB}{shadecolorB}{\textbf{\sffamily{\mbox{~}\makebox[6.762in][l]{\large #1} \vphantom{p\^{E}}}}}}
    \end{shaded}
  }\vspace{-5pt}
}

% the next four commands allow for the \ressubheading environment to be 1, 2, 3, or 4 subrows, depending on which command you use. This is admittedly hack-ish, and should probably be replaced by a single more flexible command (with optional arguments) in the future
\newcommand{\ressubheading}[4]{
\begin{tabular*}{6.5in}[t]{l@{\cftdotfill{\cftsecdotsep}\extracolsep{\fill}}r}
		\textbf{#1} & #2 \\
		\textit{#3} & \textit{#4} \\
\end{tabular*}\vspace{-6pt}}
\newcommand{\ressubheadingTalk}[2]{
\begin{tabular*}{6.5in}[t]{l@{\cftdotfill{\cftsecdotsep}\extracolsep{\fill}}r}
		#1 & #2 \\
\end{tabular*}\vspace{-6pt}}

\newcommand{\ressubheadingb}[6]{
\begin{tabular*}{6.5in}[t]{l@{\cftdotfill{\cftsecdotsep}\extracolsep{\fill}}r}
		\textbf{#1} & #2 \\
		\textit{#3} & \textit{#4} \\
		\textit{#5} & \textit{#6} \\
\end{tabular*}\vspace{-6pt}}
\newcommand{\ressubheadingc}[8]{
\begin{tabular*}{6.5in}[t]{l@{\cftdotfill{\cftsecdotsep}\extracolsep{\fill}}r}
		\textbf{#1} & #2 \\
		\textit{#3} & \textit{#4} \\
		\textit{#5} & \textit{#6} \\
		\textit{#7} & \textit{#8} \\
\end{tabular*}\vspace{-6pt}}
\newcommand\foo[9]{%
    \def\tempb{#2}%
    \def\tempc{#3}%
    \def\tempd{#4}%
    \def\tempe{#5}%
    \def\tempf{#6}%
    \def\tempg{#7}%
    \def\temph{#8}%
    \def\tempi{#9}%
    \foocontinued
}
\newcommand\foocontinued[7]{%
    % Do whatever you want with your 9+7 arguments here.
}

\newcommand{\ressubheadingd}[1]{
	\def\argten{#1}%
	\ressubheadingdb
}
\newcommand{\ressubheadingdb}[9]{
\begin{tabular*}{6.5in}[t]{l@{\cftdotfill{\cftsecdotsep}\extracolsep{\fill}}r}
		\textbf{\argten} & #1 \\
		\textit{#2} & \textit{#3} \\
		\textit{#4} & \textit{#5} \\
		\textit{#6} & \textit{#7} \\
		\textit{#8} & \textit{#9} \\
\end{tabular*}\vspace{-6pt}}
%-----------------------------------------------------------

\usepackage[lf]{Baskervaldx}

\begin{document}

{\large \begin{tabular*}{7in}{l@{\extracolsep{\fill}}r}
\textbf{\LARGE Juliette Bruce}\smallskip & \textbf{\today}\smallskip \\
Department of Mathematics & \\
University of Wisconsin & juliette.bruce@math.wisc.edu \\
480 Lincoln Dr. & \hyperref{http://math.wisc.edu/~juliettebruce}{}{}{math.wisc.edu/\textasciitilde juliettebruce} \\
Madison, WI 53706 & \\
\end{tabular*}}
\\


%%%%%%%%%%%%%%%%%%%%%%%%%%%%%%
\resheading{Education}
%%%%%%%%%%%%%%%%%%%%%%%%%%%%%%
\begin{itemize}	
\item
	\ressubheading{University of Wisconsin}{Madison, WI}{Ph.D. Mathematics}{2014 -- Present}
	\begin{itemize}
		\resitem{Advisor: Daniel Erman}
%		\resitem{Thesis: Norms and Cones in the Theory of Quantum Entanglement}
%		\resitem{Graduated with a 99.3\% average}
	\end{itemize}

\item
	\ressubheading{University of Wisconsin}{Madison, WI}{M.A. Mathematics}{2014 -- 2016}

\item
	\ressubheading{University of Michigan}{Ann Arbor, MI}{B.S. in Mathematics \& Political Science}{2010 -- 2014}
%	\begin{itemize}
%		\resitem{With High Honors and Distinction}
%	\end{itemize}
\end{itemize}

%%%%%%%%%%%%%%%%%%%%%%%%%%%%%%
\resheading{Research Interests}
%%%%%%%%%%%%%%%%%%%%%%%%%%%%%%

Algebraic Geometry, Commutative Algebra, Arithmetic Geometry, Non-linear Algebra. Specifically, homological methods in algebraic geometry, and algebraic geometry over finite fields. 

%%%%%%%%%%%%%%%%%%%%%%%%%%%%%%
\resheading{Publications}
%%%%%%%%%%%%%%%%%%%%%%%%%%%%%%

%\vspace{-0.15in}\subsection*{Peer-Reviewed Journal Articles}

\begin{revnumerate}[10]

\item
	J.~Bruce and D.~Erman. A probabilistic approach to systems of parameters and Noether normalization. {\it Algebra and Number Theory}, Accepted. E-print: \hyperref{http://arxiv.org/abs/1604.01704}{}{}{arXiv:1604.01704}.
	
\item
	J.~Bruce and W.~Li. Effective bounds on the dimensions of Jacobians covering abelian varieties.  {\it Proc. Amer. Math. Soc.}, Accepted. E-print: \hyperref{https://arxiv.org/abs/1804.11015}{}{}{arXiv:1804.11015}.
			
\item
	J.~Bruce, D.~Erman, S.~Goldstein, and J.~Yang. Conjectures and computations about Veronese syzygies.  {\it Experimental Mathematics}, To Appear. E-print: \hyperref{https://arxiv.org/abs/1711.03513}{}{}{arXiv:1711.03513}.
	
\item
	M.~Brandt, J.~Bruce, T.~Brysiewicz, R.~Krone, and E.~Robeva. The degree of $SO(n)$. {\it Combinatorial Algebraic Geometry}, 207-224, Fields Inst. Commun. \textbf{80}, (2017). E-print: \hyperref{https://arxiv.org/abs/1701.03200}{}{}{arXiv:1701.03200}.
	
\item
	J.~Bruce, M.~Logue, and R.~Walker. Monomial valuations, cusp singularities, and continued fractions. {\it Journal of Commutative Algebra}, \textbf{7} (2015) no. 4, 495-522. E-print: \hyperref{http://arxiv.org/abs/1311.6493}{}{}{arXiv:1311.6493}.

\item
	J.~Bruce, P.~Kao, E.~Nash, B.~Perez, and P.~Vermeire. Betti tables of reducible algebraic curves. {\it Proc. Amer. Math. Soc.} \textbf{142} (2014) 4039-4051. E-print: \hyperref{http://arxiv.org/abs/1210.3064}{}{}{arXiv:1210.3064}.
\end{revnumerate}

%%%%%%%%%%%%%%%%%%%%%%%%%%%%%%
\resheading{Pre-Prints}
%%%%%%%%%%%%%%%%%%%%%%%%%%%%%%

\begin{revnumerate}
\item
	J.~Bruce. The Quantitative Behavior of Asymptotic Syzygies for Hirzebruch Surfaces. {\it Submitted.} E-Print:  \hyperref{http://arxiv.org/abs/1906.07333}{}{}{arXiv:1906.07333}.
	
\item
	J.~Bruce, D.~Erman, S.~Goldstein, and J.~Yang. The SchurVeronese package in Macaulay2.  {\it Submitted}. E-print: \hyperref{https://arxiv.org/abs/1905.12661}{}{}{arXiv:1905.12661}.
	
\item
	A.~Almousa, J.~Bruce, M.~Loper, and M.~Sayrafi. The Virtual Resolutions Package for Macaulay2.  {\it Submitted.} E-print: \hyperref{http://arxiv.org/abs/1905.07022}{}{}{arXiv:1905.07022}.
	
\item
	J.~Bruce. Asymptotic Syzygies in the Setting of Semi-Ample Growth. {\it Submitted.} E-Print:  \hyperref{https://arxiv.org/abs/1904.04944}{}{}{arXiv:1904.04944}

\end{revnumerate}

%%%%%%%%%%%%%%%%%%%%%%%%%%%%%%
\resheading{Software}
%%%%%%%%%%%%%%%%%%%%%%%%%%%%%%

\begin{revnumerate}
\item
	SchurVeronese, (with D.~Erman, S.~Goldstein, and J.~Yang).  Submitted for distribution with future releases of Macaulay2, a compute algebra system focused on computations in algebraic geometry and commutative algebra.
	
\item
	VirtualResolutions, (with A.~Almousa, M.~Loper, and M.~Sayrafi). Distributed with version 1.14 of Macaulay2 (2019). 
\end{revnumerate}

%%%%%%%%%%%%%%%%%%%%%%%%%%%%%%
\resheading{Multimedia}
%%%%%%%%%%%%%%%%%%%%%%%%%%%%%%

\begin{revnumerate}
\item
	\href{https://syzygydata.com}{SyzygyData.com}, (with D.~Erman, S.~Goldstein, and J.~Yang). An online public database on large-scale syzygy computations. 

\end{revnumerate}

%%%%%%%%%%%%%%%%%%%%%%%%%%%%%%
\resheading{Grants}
%%%%%%%%%%%%%%%%%%%%%%%%%%%%%%

\begin{itemize}

\item 
	\ressubheading{Conference Grant DMS-1908799 -- \$15,000}{March 2019}{National Science Foundation}{}
	
\item 
	\ressubheading{Graduate Research Fellowship}{2015 -- 2018}{National Science Foundation}{}

\item 
	\ressubheading{Conference Grant DMS-1812462 -- \$15,000}{February 2018}{National Science Foundation}{}
	
\item 
	\ressubheading{Professional Development Grant -- \$1000}{December 2016}{Graduate School -- University of Wisconsin}{}
	
\end{itemize}

%%%%%%%%%%%%%%%%%%%%%%%%%%%%%%
\resheading{Awards \& Honors}
%%%%%%%%%%%%%%%%%%%%%%%%%%%%%%

\begin{itemize}
\item 
	\ressubheading{Excellence in Mathematical Research Award}{October 2019}{Award by the math department to a student for exceptional research in their thesis.}{}
	
\item 
	\ressubheading{Capstone Teaching Award}{October 2019}{Awarded to one student in the math dept. for an exceptional record of teaching excellence and service.}{}
	
\item 
	\ressubheading{Elizabeth Hirschfelder Prize}{October 2018}{Awarded to an outstanding female student who's demonstrated promise in their academic work.}{}

\item 
	\ressubheading{Mathematics TA Service Award}{April 2018}{Dept. of Mathematics - University of Wisconsin}{}
	
		
\item 
	\ressubheading{Teaching Assistant Award for Exceptional Service}{February 2018}{Campus-wide award recognizing TA's who perform exceptional service}{}
	
\item 
	\ressubheading{Outstanding Achievement in Mathematics}{May 2014}{Dept. of Mathematics -- University of Michigan}{}

\item 
	\ressubheading{Phi Beta Kappa}{April 2014}{University of Michigan}{}

\item 
	\ressubheading{Chancellor's Opportunity Award}{April 2014}{University of Wisconsin}{}

\end{itemize}

%%%%%%%%%%%%%%%%%%%%%%%%%%%%%%
\resheading{Seminar and Colloquium Talks}
%%%%%%%%%%%%%%%%%%%%%%%%%%%%%%
\begin{itemize}
\item
	\ressubheadingTalk{University of Michigan - Commutative Algebra Seminar}{December 2019}

\item
	\ressubheadingTalk{University of Notre Dame - Algebraic Geometry Seminar}{November 2019}

\item
	\ressubheadingTalk{DePaul University - Algebra, Combinatorics, and Number Theory Seminar}{October 2019}{}{}
	
\item
	\ressubheadingTalk{Lawerence University - Colloquium}{October 2019}{}{}

\item
	\ressubheadingTalk{University of Utah - Algebraic Geometry Seminar}{September 2019}

\item
	\ressubheadingTalk{Stanford University - Algebraic Geometry Seminar}{May 2019}
	
\item
	\ressubheadingTalk{University of Kentucky - Algebra Seminar}{April 2019}
		
\item
	\ressubheadingTalk{University of Minnesota - Commutative Algebra Seminar}{April 2019}
	
%\item
%	\ressubheading{Covering Abelian Varieties and Effective Bertini}{University of Wisconsin}{Algebra and Algebraic Geometry Seminar}{November 2018}	
	
\item
	\ressubheadingTalk{Rice University- Algebraic Geometry and Number Theory Seminar}{September 2018}{}{}
		
\item
	\ressubheadingTalk{DePaul University - Algebra, Combinatorics, and Number Theory Seminar}{March 2018}{}{}
	
%\item
%	\ressubheading{Asymptotic Syzygies in the Semi-Ample Setting}{University of Wisconsin}{Algebra and Algebraic Geometry Seminar}{February 2018}{}{}

\item
	\ressubheadingTalk{University of Michigan - Commutative Algebra Seminar}{December 2017}{}{}

%\item
%	\ressubheading{A Distributed Numerical Approach to Syzygies of $\mathbb{P}^2$ (Poster)}{UC - Berkeley}{Free Resolutions and Computations}{July 2017}
%
%\item
%	\ressubheading{Noether Normalization in Families}{University of Wisconsin}{Algebraic Geometry Seminar}{April 2016}
%
%\item
%	\ressubheading{Betti Diagrams of Graph Curves}{University of Wisconsin}{Algebraic Geometry Seminar}{December 2014}
\end{itemize}

%%%%%%%%%%%%%%%%%%%%%%%%%%%%%%
\resheading{Conference Talks}
%%%%%%%%%%%%%%%%%%%%%%%%%%%%%%

\begin{itemize}
\item
	\ressubheadingTalk{LGBTQ+Math - Fields Institute}{July 2020}

\item
	\ressubheadingTalk{Foundations of Computational Mathematics - Simon Fraser University}{June 2020}

\item
	\ressubheadingTalk{CA+ - Iowa State University}{April 2020}

\item
	\ressubheadingTalk{Joint Math Meetings - Denver, CO}{January 2020}
	
\item
	\ressubheadingTalk{Fall AMS Central Sectional - University of Wisconsin }{September 2019}
	
\item
	\ressubheadingTalk{SIAM Conference on Applied Algebraic Geometry 2019}{July 2019}
	
\item
	\ressubheadingTalk{KUMUNUjr - University of Nebraska}{March 2019}
	
\item
	\ressubheadingTalk{Spring AMS Southeastern Sectional - Auburn University}{March 2019}
	
\item
	\ressubheadingTalk{Joint Math Meetings - Baltimore, MD}{January 2019}
	
\item
	\ressubheadingTalk{Fall AMS Central Sectional - University of Michigan}{October 2018}
	
\item
	\ressubheadingTalk{Structures on Free Resolutions - Texas Tech University}{October 2017}

\item
	\ressubheadingTalk{Midwest Algebraic Geometry Graduate Conference - University of Illinois, Chicago}{April 2015}
	
\end{itemize}

%%%%%%%%%%%%%%%%%%%%%%%%%%%%%%
\resheading{Poster Talks}
%%%%%%%%%%%%%%%%%%%%%%%%%%%%%%

\begin{itemize}
\item
	\ressubheadingTalk{Summer School on Randomness and Learning in NLA - Max Plank Institute, Leipzig}{July 2019}{}{}
	
\item
	\ressubheadingTalk{2019 AWM Research Symposium - Rice University}{April 2019}

\item
	\ressubheadingTalk{AWM Poster Session - Joint Math Meetings}{January 2018}
		
\item
	\ressubheadingTalk{AGNES Poster Session - Brown University }{September 2018}
	
\item
	\ressubheadingTalk{Lectures on Arithmetic Geometry - Rice University}{February 2017}
	
\item
	\ressubheadingTalk{Introductory Workshop: Combinatorial Algebraic Geometry - Fields Institute}{August 2016}

\item
	\ressubheadingTalk{Commutative Algebra and Its Interactions with Algebraic Geometry}{July 2016}

\item
	\ressubheadingTalk{Midwest Commutative Algebra and Algebraic Geometry Conference}{May 2016}
	
\end{itemize}

%%%%%%%%%%%%%%%%%%%%%%%%%%%%%%
\resheading{Conference Organizing}
%%%%%%%%%%%%%%%%%%%%%%%%%%%%%%

\begin{itemize}
\item 
\ressubheading{Spectra Panel: Supporting Transgender and Non-binary Students}{Joint Math Meetings}{with Christopher Goff and Greg McCarthy}{January 18, 2020}

\item
\ressubheading{Special Session on Combinatorial Algebraic Geometry}{AMS Sectional}{with Daniel Erman, Chris Eur, and Lily Silverstein}{September 14-15, 2019}

\item 
\ressubheading{GWCAWMMG}{University of Minnesota}{with Christine Berkesch and Patricia Klein}{April 12-14, 2019} 

\item 
\ressubheading{Geometry \& Arithmetic of Surfaces}{University of Wisconsin}{with Wanlin Li}{February 9-10, 2019}

\item 
\ressubheading{M2@UW}{University of Wisconsin}{with Daniel Erman, Steven Sam, and Jay Yang}{April 14-17, 2018} 

\end{itemize}


%%%%%%%%%%%%%%%%%%%%%%%%%%%%%%
\resheading{Outreach Activities}
%%%%%%%%%%%%%%%%%%%%%%%%%%%%%%


\begin{itemize}

\item 
	\ressubheading{qGrads}{University of Wisconsin}{Organizer}{July 2017 -- Present}
	\begin{itemize}
		\resitem{Campus group for LGBTQ+ graduate and post-graduate students with $>350$ members.}
		\resitem{Organized a weekly coffee social hour, providing a place to relax, make friends, and discussion the challenges of being LGBTQ+.}
	\end{itemize}
	
\item 
	\ressubheading{Supporting Transgender and Non-binary Students}{MAA Panel at the JMM}{Panelist}{January 2020}
	
\item 
	\ressubheading{Undergrad Directed Reading Program}{University of Wisconsin}{Mentor}{January 2018 -- May 2019}
	\begin{itemize}
		\resitem{Lead two semester long reading projects on commutative algebra and algebraic geometry.}
		\resitem{Lead an undergraduate women on a two semester reading project, and provided guidance on applying for REU's and graduate school.}
	\end{itemize}
	
\item 
	\ressubheading{Graduate Peer Mentoring}{University of Wisconsin}{Mentor}{September 2018 -- December 2018}
	\begin{itemize}
		\resitem{Mentored 5 first year graduate students from minority genders, organizing monthly dinners where the mentees could discuss issues they were facing.}
	\end{itemize}
		
\item 
	\ressubheading{Girls Math Night Out}{University of Wisconsin}{Mentor}{September 2018 -- December 2018}
	\begin{itemize}
		\resitem{Lead 2 women from local high schools on a semesters long project about cryptography.}
	\end{itemize}

\item 
	\ressubheading{Madison Math Circle}{University of Wisconsin}{Lead Organizer}{January 2016 -- December 2018}
	\begin{itemize}
		\resitem{Lead the creation of a new outreach program, which directly visits high schools around the state of Wisconsin to better serve students from underrepresented groups.}
		\resitem{Expanded the total number of students reached per year from 25 to >250.}
	\end{itemize}

\item 
	\ressubheading{Madison Math Circle}{University of Wisconsin}{Student Volunteer}{January 2015 -- December 2018}
	
\item 
	\ressubheading{Out in STEM (oSTEM) @ UW-Madison}{University of Wisconsin}{co-Founder}{July 2017 -- Math 2018}
	\begin{itemize}
		\resitem{Founded, at the time, the only campus resource specifically for LGBTQ+ individuals in STEM, and grew the organization to over 50 members.}
		\resitem{Secured a travel grant to help 11 members (undergraduate and graduate students) attend the national oSTEM conference.}
		\end{itemize}

\item 
	\ressubheading{Out in Math: Professional Issues Facing LGBTQ Mathematicians}{MAA Panel at the JMM}{Panelist}{January 2018}

\item 
\ressubheading{Math Careers Beyond Academia}{University of Wisconsin}{Organizer}{April 14, 2017}
\begin{itemize}
		\resitem{One day conference with over 50 participants.}
		\end{itemize}

\item 
	\ressubheading{Madison Mega Math Meet}{University of Wisconsin}{Graded}{May 2015}
	
\item 
	\ressubheading{Bonding Undergraduate and Graduate Students}{University of Wisconsin}{Mentor for Undergraduate}{September 2014 -- December 2014}

\item 
	\ressubheading{Michigan Math Circle}{University of Michigan}{Organizer}{January 2013 -- June 2014}
	

\end{itemize}

%%%%%%%%%%%%%%%%%%%%%%%%%%%%%%
\resheading{Teaching Experience}
%%%%%%%%%%%%%%%%%%%%%%%%%%%%%%
\begin{itemize}
\item 
	\ressubheading{Math 221: Calculus and Analytic Geometry I}{University of Wisconsin}{Teaching Assistant}{Fall 2014/2018/2019}
	\begin{itemize}
		\resitem{Selected as a TA coordinator in 2018 and 2019, and was responsible for overseeing all other TA's and mentoring first year TA's.}
		\resitem{Average score 4.9/5.0}
	\end{itemize}
	
\item 
	\ressubheading{Math 228: Wisconsin Emerging Scholars}{University of Wisconsin}{Instructor}{Fall 2018}
	\begin{itemize}
		\resitem{Course providing students from underrepresented groups additional support.}
		\resitem{Average score: 5.0/5.0}
	\end{itemize}

\item 
	\ressubheading{Math 132: Problem Solving in Algebra, Probability \& Statistics}{University of Wisconsin}{Instructor}{Spring 2015}

\item 
	\ressubheading{Inquiry Based Learning Courses}{University of Michigan}{Course Assistant}{2012-2014}
		\begin{itemize}
		\resitem{Assisted with advanced undergraduate courses on topology, analysis, and probability. }
		\resitem{Facilitated inquiry based learning in the classroom, and responsible for office hours, grading, and review sessions.}
		\end{itemize}

\end{itemize}

%%%%%%%%%%%%%%%%%%%%%%%%%%%%%%
\resheading{Service}
%%%%%%%%%%%%%%%%%%%%%%%%%%%%%%

\begin{itemize}

\item 
	\ressubheading{AMS Graduate Student Blog}{American Mathematical Society}{Editor}{September 2015 -- September 2018}
	
\item 
	\ressubheading{AMS Graduate Student Chapter}{University of Wisconsin}{Member}{September 2014 -- September 2018}

\item 
	\ressubheading{Graduate Student Algebraic Geometry Seminar}{UW Dept. of Mathematics}{Organizer}{March 2015 -- December 2017}
		
\item 
	\ressubheading{Committee on Inclusivity and Diversity}{UW Dept. of Mathematics}{Member}{November 2016 -- August 2017}
	\begin{itemize}
		\resitem{Created policies seeking to make the department a more welcoming, inclusive, and comfortable place. This included drafting the department's statement on inclusivity, and creating similar statements for syllabi to be used throughout the department.}
	\end{itemize}

\item 
	\ressubheading{Committee on TA Pay and Performance}{UW Dept. of Mathematics}{Member}{September 2015 -- August 2017}
	\begin{itemize}
		\resitem{Developed and implemented a new system to evaluate TA performance, with the goal of creating a more transparent, useful, and non-biased system.}
	\end{itemize}
	
\item 
	\ressubheading{Instructor Excellence Program}{UW Dept. of Mathematics}{Teaching Mentor}{September 2015 -- May 2016}

\end{itemize}

%%%%%%%%%%%%%%%%%%%%%%%%%%%%%%%
%\resheading{Memberships}
%%%%%%%%%%%%%%%%%%%%%%%%%%%%%%%
%
%\begin{itemize}
%
%\item 
%	\ressubheadingTalk{Society of Industrial and Applied Mathematics}{January 2017 -- Present}{}{}	
%		
%\item 
%	\ressubheadingTalk{Association for Women in Mathematics}{January 2016 -- Present}{}{}	
%		
%\item 
%	\ressubheadingTalk{American Mathematical Society}{September 2014 -- Present}{}{}
%\end{itemize}	
%

%%%%%%%%%%%%%%%%%%%%%%%%%%%%%%
\resheading{References}
%%%%%%%%%%%%%%%%%%%%%%%%%%%%%%

	\begin{tabular}{lr}
% Referee 1
\begin{minipage}[t]{2.5in}
Prof.\ Daniel Erman\\
Department of Mathematics \\
University of Wisconsin - Madison\\
Madison, WI 53706-1325 USA\\
\Telefon\ (608)-263-3055\\
\Letter\ \href{derman@math.wisc.edu }{derman@math.wisc.edu }
\end{minipage}
&
% Referee 2
\begin{minipage}[t]{2.5in}
Prof.\ David Eisenbud\\
Department of Mathematics\\
University of California, Berkeley\\
Berkeley, CA 94720-3840 USA
\\
\Telefon\ (510)-642-0143\\
\Letter\ \href{de@math.berkeley.edu}{de@math.berkeley.edu}
\end{minipage}
\\
\\ % Additional newline for spacing.
% Referee 3
\begin{minipage}[t]{2.5in}
Prof.\ Christine Berkesch\\
Department of Mathematics\\
University of Minnesota\\
Minneapolis, MN 55455-0488 USA\\
\Telefon\ (612)-624-6424\\
\Letter\ \href{cberkesc@umn.edu}{cberkesc@umn.edu}
\end{minipage}
\end{tabular}
\end{document}