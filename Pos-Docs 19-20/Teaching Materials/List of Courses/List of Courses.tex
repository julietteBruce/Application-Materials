% (c) 2002 Matthew Boedicker <mboedick@mboedick.org> (original author) http://mboedick.org
% (c) 2003-2007 David J. Grant <davidgrant-at-gmail.com> http://www.davidgrant.ca
% (c) 2008-2014 Nathaniel Johnston <nathaniel@njohnston.ca> http://www.njohnston.ca
%
% Depending on your TeX distribution, you may need to download the revnum and longtable packages for this template to work!
%
%This work is licensed under the Creative Commons Attribution-Noncommercial-Share Alike 2.5 License. To view a copy of this license, visit http://creativecommons.org/licenses/by-nc-sa/2.5/ or send a letter to Creative Commons, 543 Howard Street, 5th Floor, San Francisco, California, 94105, USA.

\documentclass[letterpaper,11pt]{article}
\newlength{\outerbordwidth}
\usepackage{amstext,amsfonts,amssymb,amscd,amsbsy,amsmath}
\pagestyle{empty}
\raggedbottom
\raggedright
\usepackage{array}
\usepackage[svgnames]{xcolor}
\usepackage{enumerate}
\usepackage{framed}
\usepackage{longtable}
\usepackage{revnum}
\usepackage{textcomp}

\usepackage[colorlinks=true,urlcolor=blue]{hyperref}
\usepackage{tocloft}
\usepackage{array}


%-----------------------------------------------------------
%Edit these values as you see fit

\setlength{\outerbordwidth}{3pt}  % Width of border outside of title bars
\definecolor{shadecolor}{gray}{0.75}  % Outer background color of title bars (0 = black, 1 = white)
\definecolor{shadecolorB}{gray}{0.93}  % Inner background color of title bars


%-----------------------------------------------------------
%Margin setup

\setlength{\evensidemargin}{-0.25in}
\setlength{\headheight}{0in}
\setlength{\headsep}{0in}
\setlength{\oddsidemargin}{-0.25in}
\setlength{\paperheight}{11in}
\setlength{\paperwidth}{8.5in}
\setlength{\tabcolsep}{0in}
\setlength{\textheight}{9.5in}
\setlength{\textwidth}{7in}
\setlength{\topmargin}{-0.3in}
\setlength{\topskip}{0in}
\setlength{\voffset}{0.1in}
\setlength\LTleft{0.2in} % needed to make longtable full-width
\setlength\LTright{0.2in}

%-----------------------------------------------------------
%Custom commands
\newcommand{\resitem}[1]{\item #1 \vspace{-2pt}}
\newcommand{\resheading}[1]{\vspace{8pt}
  \parbox{\textwidth}{\setlength{\FrameSep}{\fboxsep}
    \begin{shaded}
\setlength{\fboxsep}{0pt}\framebox[\textwidth][l]{\setlength{\fboxsep}{4pt}\fcolorbox{shadecolorB}{shadecolorB}{\textbf{\sffamily{\mbox{~}\makebox[6.762in][l]{\large #1} \vphantom{p\^{E}}}}}}
    \end{shaded}
  }\vspace{-5pt}
}

% the next four commands allow for the \ressubheading environment to be 1, 2, 3, or 4 subrows, depending on which command you use. This is admittedly hack-ish, and should probably be replaced by a single more flexible command (with optional arguments) in the future
\newcommand{\ressubheading}[4]{
\begin{tabular*}{6.5in}[t]{l@{\cftdotfill{\cftsecdotsep}\extracolsep{\fill}}r}
		\textbf{#1} & #2 \\
		\textit{#3} & \textit{#4} \\
\end{tabular*}\vspace{-6pt}}
\newcommand{\ressubheadingTalk}[2]{
\begin{tabular*}{6.5in}[t]{l@{\cftdotfill{\cftsecdotsep}\extracolsep{\fill}}r}
		#1 & #2 \\
\end{tabular*}\vspace{-6pt}}

\newcommand{\ressubheadingb}[6]{
\begin{tabular*}{6.5in}[t]{l@{\cftdotfill{\cftsecdotsep}\extracolsep{\fill}}r}
		\textbf{#1} & #2 \\
		\textit{#3} & \textit{#4} \\
		\textit{#5} & \textit{#6} \\
\end{tabular*}\vspace{-6pt}}
\newcommand{\ressubheadingc}[8]{
\begin{tabular*}{6.5in}[t]{l@{\cftdotfill{\cftsecdotsep}\extracolsep{\fill}}r}
		\textbf{#1} & #2 \\
		\textit{#3} & \textit{#4} \\
		\textit{#5} & \textit{#6} \\
		\textit{#7} & \textit{#8} \\
\end{tabular*}\vspace{-6pt}}
\newcommand\foo[9]{%
    \def\tempb{#2}%
    \def\tempc{#3}%
    \def\tempd{#4}%
    \def\tempe{#5}%
    \def\tempf{#6}%
    \def\tempg{#7}%
    \def\temph{#8}%
    \def\tempi{#9}%
    \foocontinued
}
\newcommand\foocontinued[7]{%
    % Do whatever you want with your 9+7 arguments here.
}

\newcommand{\ressubheadingd}[1]{
	\def\argten{#1}%
	\ressubheadingdb
}
\newcommand{\ressubheadingdb}[9]{
\begin{tabular*}{6.5in}[t]{l@{\cftdotfill{\cftsecdotsep}\extracolsep{\fill}}r}
		\textbf{\argten} & #1 \\
		\textit{#2} & \textit{#3} \\
		\textit{#4} & \textit{#5} \\
		\textit{#6} & \textit{#7} \\
		\textit{#8} & \textit{#9} \\
\end{tabular*}\vspace{-6pt}}
%-----------------------------------------------------------

\usepackage[lf]{Baskervaldx}

\begin{document}

{\large \begin{tabular*}{7in}{l@{\extracolsep{\fill}}r}
\textbf{\LARGE Juliette Bruce}\smallskip & \textbf{\today}\smallskip \\
Department of Mathematics & \\
University of Wisconsin & juliette.bruce@math.wisc.edu \\
480 Lincoln Dr. & \hyperref{http://math.wisc.edu/~juliettebruce}{}{}{math.wisc.edu/\textasciitilde juliettebruce} \\
Madison, WI 53706 & \\
\end{tabular*}}
\\

%%%%%%%%%%%%%%%%%%%%%%%%%%%%%%
%%%%%%%%%%%%%%%%%%%%%%%%%%%%%%
\resheading{Teaching Experience}
%%%%%%%%%%%%%%%%%%%%%%%%%%%%%%
\begin{itemize}
\item 
	\ressubheading{Math 221: Calculus and Analytic Geometry I}{University of Wisconsin}{Teaching Assistant \& Course Coordinator}{Fall 2019}
	\begin{itemize}
		\resitem{Responsible for overseeing all other TA's and mentoring first year TA's.}
		\end{itemize}
	
\item 
	\ressubheading{Math 228: Wisconsin Emerging Scholars}{University of Wisconsin}{Instructor of Record}{Fall 2018}
	\begin{itemize}
		\resitem{Course providing students from underrepresented groups additional support.}
	\end{itemize}
	
\item 
	\ressubheading{Math 221: Calculus and Analytic Geometry I}{University of Wisconsin}{Teaching Assistant \& Course Coordinator}{Fall 2018}
		\begin{itemize}
		\resitem{Responsible for overseeing all other TA's and mentoring first year TA's.}
		\end{itemize}
	
\item 
	\ressubheading{Math 132: Problem Solving in Algebra, Statistics, \& Probability}{University of Wisconsin}{Instructor of Record}{Spring 2015}
	\begin{itemize}
		\resitem{Course for early education majors.}
	\end{itemize}

\item 
	\ressubheading{Math 221: Calculus and Analytic Geometry I}{University of Wisconsin}{Teaching Assistant}{Fall 2014}

\item 
	\ressubheading{Hex and the 4 C's}{Michigan Math and Science Scholars}{Course Assistant}{Summer 2014}
	\begin{itemize}
		\resitem{Summer program for middle and high school students interested in math and science.}
	\end{itemize}

\item 
	\ressubheading{Math 310: Explorations in Randomness}{University of Michigan}{Course Assistant}{Winter 2014}

\item 
	\ressubheading{Math 490: Introduction to Topology}{University of Michigan}{Course Assistant}{Fall 2013}

\item 
	\ressubheading{Graph Theory}{Michigan Math and Science Scholars}{Course Assistant}{Summer 2013}
	\begin{itemize}
		\resitem{Summer program for middle and high school students interested in math and science.}
	\end{itemize}
\item 
	\ressubheading{Math 351: Principals of Analysis}{University of Michigan}{Course Assistant}{Winter 2013}

	
\item 
	\ressubheading{Math 105: Data, Functions, and Graphs}{University of Michigan}{Course Assistant}{Fall 2012}

\item 
	\ressubheading{Math Lab}{University of Michigan}{Tutor}{Fall 2011 -- Winter 2013}

\end{itemize}


	
\end{document}